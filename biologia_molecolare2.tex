\documentclass[]{article}
\usepackage{lmodern}
\usepackage{amssymb,amsmath}
\usepackage{ifxetex,ifluatex}
\usepackage{fixltx2e} % provides \textsubscript
\ifnum 0\ifxetex 1\fi\ifluatex 1\fi=0 % if pdftex
  \usepackage[T1]{fontenc}
  \usepackage[utf8]{inputenc}
\else % if luatex or xelatex
  \ifxetex
    \usepackage{mathspec}
    \usepackage{xltxtra,xunicode}
  \else
    \usepackage{fontspec}
  \fi
  \defaultfontfeatures{Mapping=tex-text,Scale=MatchLowercase}
  \newcommand{\euro}{€}
\fi
% use upquote if available, for straight quotes in verbatim environments
\IfFileExists{upquote.sty}{\usepackage{upquote}}{}
% use microtype if available
\IfFileExists{microtype.sty}{%
\usepackage{microtype}
\UseMicrotypeSet[protrusion]{basicmath} % disable protrusion for tt fonts
}{}
\ifxetex
  \usepackage[setpagesize=false, % page size defined by xetex
              unicode=false, % unicode breaks when used with xetex
              xetex]{hyperref}
\else
  \usepackage[unicode=true]{hyperref}
\fi
\hypersetup{breaklinks=true,
            bookmarks=true,
            pdfauthor={},
            pdftitle={},
            colorlinks=true,
            citecolor=blue,
            urlcolor=blue,
            linkcolor=magenta,
            pdfborder={0 0 0}}
\urlstyle{same}  % don't use monospace font for urls
\setlength{\parindent}{0pt}
\setlength{\parskip}{6pt plus 2pt minus 1pt}
\setlength{\emergencystretch}{3em}  % prevent overfull lines
\setcounter{secnumdepth}{0}

\date{}

\begin{document}

\section{Tecniche di biologia
molecolare}\label{tecniche-di-biologia-molecolare}

\subsection{Enzimi di restrizione}\label{enzimi-di-restrizione}

Enzimi che apportano tagli sui filamenti di DNA vengono chiamati
\textbf{nucleasi}, e si distinguono in:

\begin{itemize}
\itemsep1pt\parskip0pt\parsep0pt
\item
  \textbf{endonucleasi}, se tagliano i legami fosfodiesterici interni in
  una molecola di DNA circolare o lineare;
\item
  \textbf{esonucleasi}, se si legano a una delle due estremità libere e
  iniziano a tagliare in maniera processiva verso l'altra estremità.
\end{itemize}

Gli enzimi di restrizione \textbf{(ER)} sono delle endonucleasi che
provocano delle rotture interne a doppio filamento sul DNA in
corrispondenza di particolari sequenze nucleotidiche.

Le endonucleasi di restrizione si legano ad una sequenza specifica di
DNA, detta \emph{``sito di riconoscimento''}, e tagliano entrambi i
filamenti dell'elica.

Gli ER vengono distinti in ER del I, del II e del III tipo. Gli
\textbf{ER del I e del III tipo} portano l'attività di restrizione e
metilazione nella stessa molecola. Filamenti di DNA \emph{non metilati}
possono essere attaccati da questi enzimi, mentre filamenti metilati ne
impediscono l'azione. Questi enzimi riconoscono una specifica sequenza
di nucleotidi, ma tagliano il DNA in posizioni non specifiche a una
certa distanza dalla sequenza riconosciuta. Poichè il sito di taglio non
corrisponde a una specifica coppia di basi, gli enzimi di restrizione
del I tipo non sono quasi mai utilizzati nelle tecnologie del DNA
ricombinante.

Gli \textbf{ER del II tipo} riconoscono anch'essi specifiche sequenze di
nucleotidi, ma tagliano il DNA all'interno di quelle specifiche
sequenze, chiamate \textbf{siti di taglio}.

La sequenza di riconoscimento per la maggioranza degli ER del II tipo è
lunga 4-6 nucleotidi, ma esistono ER chiamati \textbf{rare cutter} che
riconoscono sequenze più lunghe. La frequenza di taglio di un ER è
legata alla lunghezza della sequenza riconosciuta da quello specifico
enzima. Una sequenza si 4 basi significa una probabilità di taglio di
(1/4)\(^4\), cioè 1 volta ogni 256 nucleotidi.

ER isolati da batteri differenti ma che riconoscono e tagliano la stessa
sequenza nucleotidica sono chiamati \textbf{isoschizomeri}.

La maggior parte degli ER riconoscono sequenze sulla doppia elica del
DNA definite \textbf{palindromiche}, ovvero possiedono la stessa
sequenza sui due filamenti letti in direzione 5' \(\rightarrow\) 3' (es.
la sequenza 5'GAATTC3' riconosciuta dall'ER noto come EcoRI).

L'utilizzo degli enzimi di restrizione é molto semplice; la maggior
parte di essi funziona in semplici tamponi tra pH 7 e 8, generalmente a
37°C. Le condizioni di utilizzo sono comunque sempre specificate dai
fornitori. Per definizione \emph{un'unità di un enzima di restrizione è
la quantità di enzima richiesta per digerire completamente 1 \(\mu\)g di
DNA substrato in un'ora}. Tutti gli enzimi, in condizioni non ottimali,
danno l'\textbf{``effetto star''}, che consiste nella capacità
dell'enzima di ``confondersi'' riconoscendo e tagliando sequenze simili,
ma non identiche a quella target. Per evitare l'effetto target é
opportuno attenersi alle condizioni specificate dai fornitori, con
particolare riferimento al glicerolo e alla quantità di enzima, che non
devono essere mai in eccesso.

\subsubsection{Nomenclatura}\label{nomenclatura}

Per assegnare in modo chiaro ed univoco un codice ad un enzima si è
deciso che:

\begin{itemize}
\itemsep1pt\parskip0pt\parsep0pt
\item
  le prime \textbf{3 lettere} sono prese dal nome del batterio di
  origine. La prima lettera dal genere e le altre due dalla specie (es.
  Eco \textbf{E}scherichia \textbf{co}li);
\item
  tipi differenti dello stesso organismo sono indicati da una
  \textbf{quarta lettera} minuscola (es. Hin\textbf{d} Haemophilus
  influenzae sierotipo Rd, Hin\textbf{f} Haemophilus influenzae
  sierotipo Rf);
\item
  segue una lettere maiuscola o numero che identificano un ceppo
  particolare di batterio (Eco\textbf{R} Escherichia coli ceppo RY13);
\item
  un \textbf{numero romano} indica l'ordine di scoperta qualora dallo
  stesso batterio vengano isolati enzimi diversi.
\end{itemize}

\textbf{Esempio}

\textbf{EcoRI}:

\begin{itemize}
\itemsep1pt\parskip0pt\parsep0pt
\item
  \textbf{E} = genere Escherichia;
\item
  \textbf{co} = speie coli;
\item
  \textbf{R} = ceppo RY13;
\item
  \textbf{I} = prima endonucleasi isolata.
\end{itemize}

Sequenza riconosciuta: G/AATTC - CTTAA/G

\textbf{BamHI}:

\begin{itemize}
\itemsep1pt\parskip0pt\parsep0pt
\item
  \textbf{B} = genere Bacillus;
\item
  \textbf{am} = specie amyloliquefaciens G/GATCC;
\item
  \textbf{H} = ceppo H;
\item
  \textbf{I} = prima endonucleasi isolata.
\end{itemize}

Sequenza riconosciuta: G/GATCC - CCTAG/G

Come fa un organismo che produce una endonucleasi a proteggersi
dall'azione dell'enzima prodotto? Tramite un sistema di
\textbf{protezione per modificazione}.

Dopo aver effettuato il taglio, l'ultimo nucleotide (quello rimasto
``scoperto'') viene metilato. La metilazione non altera la normale
struttura del DNA, ma inibisce il taglio.

Il taglio tramite enzimi di restrizine può generare diversi tipi di
estremità.

Esistono ER in graod di tagliare entrambi i filamenti della doppia elica
nello stesso punto, generando delle \textbf{estremità piatte (blunt
end)}; altri ER, invece, operano un taglio obliquo lasciando delle
\textbf{estremità sporgenti}. In questo caso le estremità possono
sporgere al 5' o al 3' e vengono chiamate \textbf{sticky end}. Frammenti
di DNA generati da tagli con ER possono essere legati tra di loro o
inseriti in un vettore di clonaggio in provetta utilizzando l'enzima DNA
ligasi.

\subsection{L'elettroforesi su gel}\label{lelettroforesi-su-gel}

Molecole diverse di DNA, RNA e proteine possono essere separate modiante
elettroforesi su gel perchè differiscono nella loro carica elettrica,
dimensione e struttura.

L'elettroforesi rappresenta lo strumento classico di cui si serve un
biologo molecolare per \emph{separare}, \emph{identificare} e
\emph{isolare} frammenti di DNA o RNA.

Le prime elettroforesi, che venivano eseguite in fase liquida, sono
state soppiantate da quelle in fase solida, più riproducibili. Le
matrici solide generalmente usate sono costituite da:

\begin{itemize}
\itemsep1pt\parskip0pt\parsep0pt
\item
  \textbf{gel di agarosio}, uno zucchero solubile in acqua alla
  temperatura di ebollizione che diventa solido quando si raffredda,
  formando una matrice attraverso dei legami idrogeno tra le catene
  lineari;
\item
  \textbf{gel di poliacrilammide}, si forma tramite copolimerizzazione
  di acrilammide, un monomero solubile in acqua, e di un agente che
  forma legami trasversali così da formare un reticolo tridimensionale.
\end{itemize}

I gel più utilizzati in biologia molecolare sono quelli di agarosio con
percentuali varianti tra 0,7 e 2\%. A basse concentrazioni di agarosio
si risolvono meglio i pesi molecolari alti, mentre ad alte
concentrazioni si risolvono meglio i pesi molecolari bassi. Per
risolvere frammenti di DNA molto piccoli (tra 200 e 1 bp) si ricorre a
gel di poliacrilammide.

I due tipi di gel differiscono per le dimensioni dei pori del reticolo:
gel con pori grandi permettono la separazione di molecole di grandi
dimensioni, gel con pori piccoli separano molecole di dimensioni
limitate.

La separazione tramite elettroforesi consiste nel movimento di una
molecola carica sottoposta ad un campo elettrico. Questo campo elettrico
viene creato dal collegamento di un alimentatore di corrente elettrica
all'apparecchio per l'elettroforesi.

Poiché gli acidi nucleici sono molecole cariche negativamente, a causa
dei loro gruppi fosfato (in un tampone neutro o alcalino), migreranno
verso il polo positivo (anodo) e si separano in base alla loro massa
molecolare.

(immagine 66)

Per poter visualizzare lo spostamento dei frammenti di DNA su un gel,
altrimenti invisibili, è necessario applicare un colorante specifico
come il \emph{bromuro di etidio} (mutageno). Questa sostanza si
intercala tra i due filamenti di DNA ed emette una fluorescenza di
colore arancione quando esposta a raggi UV.

(immagine 64)

Il DNA sottoposto ad un campo elettrico migra ad una \emph{velocità
proporzionale all'inverso del log 10 del suo peso molecolare}.

\textbf{D = a-b(LogM)}

Dove:

\begin{itemize}
\itemsep1pt\parskip0pt\parsep0pt
\item
  \textbf{D} è la distanza coperta;
\item
  \textbf{M} è il peso molecolare del frammento;
\item
  \textbf{a} e \textbf{b} sono costanti.
\end{itemize}

Facendo migrare il campione in esame insieme ad uno standard di
frammenti con pesi molecolari di dimensione nota è possibile estrapolare
il peso molecolare del campione ignoto.

Utilizzando questa serie di frammenti di DNA a peso molecolare noto è
possibile costruire una \textbf{curva di taratura}, dove ogni banda sul
gel corrisponde ad un punto le cui coordinate sono costituite, in
\textbf{ascissa}, dalla \textbf{distanza in cm dal pozzetto} e in
\textbf{ordinata}, dal \textbf{Log 10 del peso molecolare}. Congiungendo
tutti i punti, riportati su un grafico semi-logarittimico, si dovrebbe
formare una linea approssimativamente retta,da cui, nota la distanza dal
pozzetto percorsa da una banda incognita, è possibile estrapolare il
peso molecolare approssimativo.

(immagine 65)

\section{RNA interference}\label{rna-interference}

L'RNA interference consiste nel processo biologico in cui molecole di
RNA inibiscono l'espressione di un gene (solitamente causando la
distruzione di specifiche molecole di mRNA).

Storicamente questo fenomeno è stato indicato anche con altri nomi,
quali:

\begin{itemize}
\itemsep1pt\parskip0pt\parsep0pt
\item
  \textbf{co-soppressione}. Il fenomeno dell'interferenza è stato per
  prima scoperto nei vegetali. Nel 1990, Jorgenson e colleghi, cercando
  di ottenere un fiore di colore scuro, avevano introdotto un
  \textbf{transgene} per la \emph{chalone synthase} (responsabile della
  sintesi del pigmento anthocyanina) in cellule di petunia.
  Sorprendetemente, ottennero un fiore variegato. L'analisi delle
  cellule bianche rivelò che in queste erano assenti i mRNAs sia del
  transgene che del gene endogeno Con co-soppressione (1990) si indica
  il fenomeno per cui l'introduzione di un gene esogeno codificante per
  l'\emph{enzima calcone sintasi} in petunia, silenzia sia il gene
  esogeno che quello endogeno;
\item
  un altro esempio di RNA interference è il \textbf{post-transcriptional
  gene silencing (PTGS)} del \emph{gene par-1} di C. elegans tramite
  l'introduzione di RNA antisenso. Lo stesso risultato venne ottenuto
  mediante l'introduzione di RNA senso usato come controllo negativo;
\item
  un altro fenomeno simile al PTGS è il \textbf{quelling}. Questo
  fenomeno è stato osservato in petunia ma avvenuto in una coltura di N.
  crassa trasformata con un plasmide contenente il \emph{gene al1}.
\end{itemize}

Nel 1998 gli studiosi Fire e Mello catalogarono tutti questi fenomeni
isolati sotto un principio comune: RNA interference (\textbf{RNAi}).

Continuando gli studi su C. elegans di Guo \& Kemphues per cercare delle
spiegazioni agli eventi da essi osservati, Fire e Mello ipotizzarono che
le preparazioni di RNA antisenso contenessero piccole quantità di RNA
senso e che la formazione di molecole ibride a doppio filamento
(\textbf{dsRNA}) fosse responsabile del silenziamento genico osservato
con l'anti-senso (RNAi).

I meccanismi responsabili del fenomeno possono essere:

\begin{itemize}
\itemsep1pt\parskip0pt\parsep0pt
\item
  il silenziamento trascrizionale del gene;
\item
  il silenziamento post-trascrizionale del gene (PTGS), l'mRNA è
  sintetizzato e poi degradato.
\end{itemize}

Nel 1995 Guo \& Kemphues, per dimostrare che avevano clonato il gene
par-1 (un gene necessario per la divisione mitotica dello zigote) di
C.elegans, condussero esperimenti di PTGS con oligonucleotidi antisenso.

Nel 1998, Fire \& Mello circonstanziarono che l'iniezione di dsRNA era
un modo molto efficiente per ottenere fenotipi mutanti in C.elegans.

L'introduzione di una soluzione contenente filamenti di RNA senso ed
antisenso era almeno dieci volte più efficace dei filamenti singoli.

Le analisi elettroforetiche mostrarono che, all'interno dell'organismo,
il materiale iniettato si conformava per la maggior parte come doppio
filamento (dsRNA, RNA double strand) infatti le molecole di ssRNA
(single strand RNA) venivano rapidamente degradate nel citoplasma.

Di contro, osservarono che un tempo prolungato tra l'iniezione del
filamento senso e di quello antisenso dava un calo drastico
dell'efficienza fino alla totale assenza di effetto.

Sorprendentemente, gli effetti di questa interferenza genica si
mostravano sia nei soggetti iniettati che nella progenie.

Che cos'è la RNA interference? Un fenomeno in cui l'espressione (o
introduzione) di un RNA a doppio filamento (dsRNA) in un diverso range
di organismi e tipi cellulari causa la degradazione di un mRNA bersaglio
complementare.

\section{Formazione e azione di siRNA e
miRNA}\label{formazione-e-azione-di-sirna-e-mirna}

\end{document}
