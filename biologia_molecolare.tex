\documentclass[]{article}
\usepackage{lmodern}
\usepackage{amssymb,amsmath}
\usepackage{ifxetex,ifluatex}
\usepackage{fixltx2e} % provides \textsubscript
\ifnum 0\ifxetex 1\fi\ifluatex 1\fi=0 % if pdftex
  \usepackage[T1]{fontenc}
  \usepackage[utf8]{inputenc}
\else % if luatex or xelatex
  \ifxetex
    \usepackage{mathspec}
    \usepackage{xltxtra,xunicode}
  \else
    \usepackage{fontspec}
  \fi
  \defaultfontfeatures{Mapping=tex-text,Scale=MatchLowercase}
  \newcommand{\euro}{€}
\fi
% use upquote if available, for straight quotes in verbatim environments
\IfFileExists{upquote.sty}{\usepackage{upquote}}{}
% use microtype if available
\IfFileExists{microtype.sty}{\usepackage{microtype}}{}
\ifxetex
  \usepackage[setpagesize=false, % page size defined by xetex
              unicode=false, % unicode breaks when used with xetex
              xetex]{hyperref}
\else
  \usepackage[unicode=true]{hyperref}
\fi
\hypersetup{breaklinks=true,
            bookmarks=true,
            pdfauthor={},
            pdftitle={},
            colorlinks=true,
            citecolor=blue,
            urlcolor=blue,
            linkcolor=magenta,
            pdfborder={0 0 0}}
\urlstyle{same}  % don't use monospace font for urls
\setlength{\parindent}{0pt}
\setlength{\parskip}{6pt plus 2pt minus 1pt}
\setlength{\emergencystretch}{3em}  % prevent overfull lines
\setcounter{secnumdepth}{0}


\begin{document}

\section{Biologia Molecolare}\label{biologia-molecolare}

\subsection{Struttura degli acidi
nucleici}\label{struttura-degli-acidi-nucleici}

\subsubsection{Struttura chimica degli acidi
nucleici}\label{struttura-chimica-degli-acidi-nucleici}

Gli acidi nucleici, DNA e RNA, sono costituiti da catene
polinucleotidiche, cioè polimeri lineari di unità chiamate
\textbf{nucleotidi}.

I nucleotidi sono molecole costituite da tre componenti:

\begin{enumerate}
\def\labelenumi{\arabic{enumi}.}
\itemsep1pt\parskip0pt\parsep0pt
\item
  uno \textbf{zucchero pentoso};
\item
  una \textbf{base azotata};
\item
  uno o più \textbf{gruppi fosfato}.
\end{enumerate}

Nel caso del DNA lo zucchero è il \textbf{deossiribosio}, mentre
nell'RNA è il \textbf{ribosio}. I due zuccheri differiscono per un
gruppo -OH presente in posizione 2 del ribosio e che manca nel
deossiribosio. Questo gruppo -OH conferisce instabilità all'RNA.

Le \emph{basi azotate} che si trovano nelgi acidi nucleici naturali sono
di due tipi:

\begin{enumerate}
\def\labelenumi{\arabic{enumi}.}
\itemsep1pt\parskip0pt\parsep0pt
\item
  \textbf{purine}, a doppio anello eterociclico (adenina e guanina);
\item
  \textbf{pirimidine}, a singolo anello (citosina, timina e uracile).
\end{enumerate}

Citosina, adenina e guanina sono comuni sia al DNA che all'RNA, mentre
la timini si torva solo nel DNA e l'uracile solo nell'RNA.

La differenza tra uracile e timina è la presenza di un gruppo metilico
in posizione 5 dell'anello pirimidinico.

(immagine Basi e zuccheri)

Quando una delle suddette basi è legata alla posizione 1 di uno
zucchero, abbiamo i \textbf{nucleosidi}.

Ai nucleosidi possono essere legati da uno a tre gruppi fosfato e, in
tal caso, prendono il nome di \textbf{nucleotidi}.

Il gruppo fosfato conferisce una valenza acida alla molecola. Il gruppo
fosfato può trovarsi legato al carbonio 5' dello zucchero oppure al
carbonio 3'.

I nucleotidi che vengono utilizzati come substrato per la sintesi del
DNA e dell'RNA hanno 3 gruppi fosfato legati in serie sulla posizione 5'
dello zucchero.

Le basi possono subire modificazioni chimiche che sono essenziali per i
processi biologici che coinvolgono il DNA e gli RNAs.

Lo scheletro della catena polinucleotidica è costituito dall'alternanza
di zuccheri e di gruppi fosfato, mentre le basi azotate sporgono
lateralmente da questo scheletro. Ciascuna base è legata alla posizione
1' di uno zucchero da un legame glicosidico che interessa l'N1 delle
pirimidine o l'N9 delle purine. Tra uno zucchero e l'altro c'è un solo
gruppo fosfato e che quindi si tratta di un polimero di nucleosidi
monofosfato. Durante la sintesi degli acidi nucleici, ciascun nucleoside
trifosfato utilizzato come substrato, perde due dei suoi tre gruppi
fosfato. Ciascun gruppo fosfato forma un \textbf{legame estere} con il
C5' di uno zucchero e un secondo legame estere con il C3' dello zucchero
successivo. Questo tipo di legame, chiamato \textbf{fosfodiesterico},
conferisce una sorta di asimmetria (o polarità) alla catena, nel senso
che questa presenta due diverse estremità: da una parte c'è un
nucleotide che ha il C5' libero, mentre il C3' è impegnato nel legame
fosfodiesterico con il nucleotide adiacente; dall'altra la molecola
polimerica termina con un nucleotide che ha il C5' impegnato nel legame
fosfodiesterico con quello adiacente, mentre il C3' è libero. La loro
valenza è comunque impegnata da gruppi -OH o da gruppi fosfato.

\subsubsection{Struttura fisica del DNA}\label{struttura-fisica-del-dna}

Il DNA è formato da due catene polinucleotidiche antiparallele avvolte
l'una sull'altra a formare una struttura a doppia elica con lo scheletro
zucchero-fosfato posto sul lato esterno e le basi impilate all'interno.

Secondo il modello proposto da Watson e Crick le coppie di basi, che con
i loro anelli esterociclici sono strutture essenzialmente piatte, sono
disposte quasi perpendicolarmente rispetto all'asse della doppia elica.
Le basi basi che si affacciano in ciascuna coppia sono sempre una
pirimidina e una purina, ed è sempre presente l'appaiamento A-T/U o C-G
(questo giustifica la regolarità del diametro della doppia elica).

Le coppie A-T sono legate da 2 legami idrogeno, mentre le coppie G-C
sono legate da 3 legami idrogeno.

La doppia elica (duplex) del DNA ha una struttura regolare, destrorsa,
compie un giro completo ogni 34 A e ha un diametro di circa 20 A. La
distanza tra due coppie di basi adiacenti è di 3,4 A e ci sono, quindi,
circa 10,4 coppie di basi per ogni giro di elica. L'elica presenta
un'asimmetria dovuta alla posizione delle molecole di deossiribosio ai
lati delle basi: come conseguenza della opposta polarità 5'$\rightarrow$
3' delle due eliche, i due zuccheri di ciascuna coppia di nucleotidi si
vengono a trovare dallo stesso lato. Questa asimmetria genera nella
doppia elica due solchi di dimensioni diverse detti \textbf{solco
maggiore} e \textbf{solco minore}, con diametro 22 A e 12 A
rispettivamente.

\paragraph{Strutture alternative del
DNA}\label{strutture-alternative-del-dna}

La struttura del DNA proposta da Watson e Crick non è l'unica possibile.
Tale forma è detta \textbf{forma B} ed è stata ottenuta da studi di
diffrazione ai raggi X condotti in condizioni di \emph{alta umidità
(95\%)} e \emph{bassa salinità (condizioni cellulari)}.

Se si riduce l'umidità relativa in cui si trova la fibra di DNA, esso
assume la \textbf{forma A}. Questa forma è destrorsa come la B, ma se ne
differenzia per vari aspetti:

\begin{itemize}
\itemsep1pt\parskip0pt\parsep0pt
\item
  il passo dell'elica è di 25 A e il diametro di 23 A (è più tozza
  rispetto alla forma B);
\item
  presenta 11 coppie di basi per ogni giro dell'elica
\item
  le coppie di basi presentano una maggiore rispetto al piano
  perpendicolare all'asse della doppia elica;
\end{itemize}

Questa forma è stata riconosciuta tramite studi di diffrazione ai raggi
X in condizioni di \emph{minore umidità (75\%)} e \emph{alta salinità}.
Questa è, in vivo, la struttura tipica dei duplex DNA-RNA e RNA-RNA. Il
2'-OH del ribosio impedisce alla molecola di assumere la forma B.

L'ultima conformazione è la \textbf{forma Z}. Questa forma è
\emph{sinistrorsa} a causa del cambiamento dell'orientamento del legame
glucosidico base-zucchero tra la guanina e il deossiribosio. Nella forma
B lo zucchero e la base sono presenti nella conformazione ``anti'',
mentre nella forma Z si presentano nella conformazione ``syn''. La forma
Z ha un passo di 46 A, con 12 coppie di basi per giro d'elica, e un
diametro di 18 A. Rispetto alla forma B (passo 34Å), questo DNA ha una
forma allungata e magra.

Questa forma è stata riconosciuta tramite studi di diffrazione a raggi X
in condizioni di \emph{alta salinità} o in presenza di alcoli. Sono
state isolate proteine che legano ad alta affinità la forma Z e prove
sperimentali dicono che una piccola percentuale del DNA in vivo si trova
in questa forma.

Esistono altre forme che non analizzeremo (C,D e E).

\subsubsection{Topologia del DNA e DNA
topoisomerasi}\label{topologia-del-dna-e-dna-topoisomerasi}

La struttura di due filamenti avvolti in una doppia elica pone die seri
problemi durante i vari processi che richiedono un'apertura dell'elica e
la separazione dei due filamenti, dove il DNA si attorciglia us se
stesso a formare strutture complesse, dette \textbf{superavvolgimenti}.
Lo stato superavvolto del DNA contiene, come in una molla, l'energia che
viene utilizzata proprio per aprire i due filamenti Le molecole di DNA
possono essere circolari o lineari, in entrambi i casi possono
presentarsi dei superavvolgimenti.

Questi superavvolgimenti causano una variazione nella
\textbf{topologia}, ovvero della conformazione nello spazio, del DNA.

Il cambiamento della topologia del DNA è un aspetto fondamentale per
tutte quelle attività funzionali che richiedono una separazione dei
filamenti (replicazione, trascrizione, ricombinazione, ecc.). Le
molecole di DNA sia batteriche che eucariotiche sono \emph{superavvolte
negativamente}. La separazione dei due filamenti è maggiormente favorita
nel DNA superavvolto negativamente che nel DNA rilassato.

Le conseguenze del superavvolgimento cambiano a seconda del verso di
attorcigliamento. Se il superavvolgimento è \emph{negativo} il DNA si
avvolge in \emph{direzione opposta} rispetto a quella della doppia
elica. In questo modo diminuisce il n° di basi per giro elica.

Se il superavvolgimento é \emph{positivo} il DNA si avvolge nella
\emph{stessa direzione} della doppia elica aumenta. In questo modo il il
n° di basi per giro d'elica aumenta.

In entrambi i casi si crea tensione all'interno del DNA.

Consideriamo due filamenti circolari chiusi: il numeor di volte che un
filamento dovrebbe passsare attraverso l'altro in maniera che essi
possano essere cioketamente separati generando due circoli in maniera
che essi possano essere completamente separati generando due circoli a
singolo filamento, si chiama \textbf{numero di legame (Lk)}.

La \emph{frequenza} (quante volte un filamento si avvolge sull'altro) e
la posizione si entrambi nello spazio tridimensionale possono essere
descritte da due grandezze:

\begin{itemize}
\itemsep1pt\parskip0pt\parsep0pt
\item
  il \textbf{twist (Tw)}, rappresenta il numero di giri della doppia
  elica rispetto all'asse centrale (definisce il grado di avvolgimento
  della doppia elica). Questo è una proprietà intrinseca alla molecola
  ed consiste nel n° totale bp/ n° bp per giro d'elica (10,4 per il DNA
  in forma B);
\item
  il \textbf{writhe (Wr)}, rappresenta il numero di volte che l'asse
  centrale della doppia elica incontra se stesso.
\end{itemize}

La somma di queste due grandezze indica il numero di volte che un
filamento si avvolge sull'altro ed è il numero di legame. Lk può dunque
essere definito dalla seguente equazione:

\textbf{Lk = Tw + Wr}

Questa equazione descrive le possibili conformazoni che il DNA assume
nello spazio tridimensionale.

Per il DNA circolare chiuso o il DNA lineare con estremità fissate, Lk è
un valore costante e non può essere modificato da nessuna deformazione
che non comporti la rottura e la riunione di uno o entrambi filamenti.

Pertanto se Wr cambia a causa di un superavvolgimento positivo o
negativo, il Tw dovrà cambiare nella direzione opposta.

Se Lk = costante

\textbf{$\Delta$Tw = - $\Delta$Wr}

Molecole circolari covalentemente chiuse di uguale lunghezza (stessa
sequenza di basi) ma che differiscono solo per il numero di legame, sono
definite \textbf{topoisomeri}.

Consideriamo ora il caso di un DNA circolare completamente rilassato
(cccDNA): il suo numero di legame Lk sarà uguale al suo twist e il
writhe sarà 0. Definiamo in questo caso \textbf{Lk = Lk$_0$}. Se
vogliamo paragonare il grado di superelicità di due DNA che hanno la
stessa lunghezza, come due diversi topoisomeri, questa differenza è
definita da:

\textbf{$\Delta$Lk = Lk - Lk$_0$}

Se il $\delta$Lk di un cccDNA è diverso da 0, la molecola conterrà della
tensione e sarà superavvolta. Con Lk \textless{} Lk$_0$ e $\Delta$Lk
\textless{} 0, il DNA è superavvolto negativamente, se Lk \textgreater{}
Lk$_0$ e $\Delta$Lk \textgreater{} 0, allora il DNA è superavvolto
positivamente

Senza introduzione di tagli, essendo Lk costante, se si forza il DNA a
cambiare Wr, la molecola compenserà cambiando Tw e viceversa.

Per modificare Lk occorre rompere la doppia elica \emph{(nick)} e far
ruotare i due filamenti l'uno rispetto all'altro in modo da aumentare o
diminuire il numero di volte che un filamento si incrocia con l'altro.

(aggiungere immagine ``superavvolgimenti'')

\textbf{Confronto fig. A-B-C} Se si riduce L di 2 unità, la molecola
introduce 2 superavvolgimenti negativi (Wr = -2) per ripristinare il
valore di TW=20 (208/10,4). L \textless{} Lk 0

\textbf{Confronto fig. D-E-F} Se si aumenta L di 2 unità, la molecola
introduce 2 superavvolgimenti positivi (Wr= +2) per ripristinare il
valore di Tw=20 (208/10,4) L \textgreater{} Lk 0

\paragraph{Le topoisomerasi}\label{le-topoisomerasi}

Lo stato topologico del DNA deve essere tneuto sotto controllo e per
fare questo la doppia elica deve essere aperta e richiusa
temporaneamente. Per catalizzare queste reazioni complesse gli organismi
contengono una classe di enzimi specializzati, chiamati \textbf{DNA
topoisomerasi}, che lavorano per mantenere adeguato il livello di
superavvolgimento del DNA.

Questi sono enzimi altamente conservati che convertono (isomerizzano) un
topoisomero in un altro modificandone il numero di legame Lk.

Questi enzimi catalizzano la rottura e la ri-unione dei filamenti di DNA
introducendo nick temporanei: creano un taglio transiente sul DNA
mediante una tirosina che si lega covalentemente allo scheletro fosfato,
con una reazione di transesterificazione fatta dal suo gruppo -OH. Dopo
la manipolazione topologica che aumenta o diminuisce Lk, il gruppo -OH
libero dle filamente tagliato, attraverso una reazione inversa, attacca
il legame tra la tirosina e il DNA, ripristinando la continuità della
doppia elica.

L'intero processo di modificazione topologica utilizza l'energia libera
contenuta nel DNA superavvolto senza richiesta di ATP aggiuntivo.

Le estremità del DNA generate dalla rottura del filamento non sono mai
libere, ma sono manipolate dentro i confini dell'enzima.

Alcune topoisomerasi possono rimuovere (e perciò rilassare)
superavvolgimenti negativi, altre sia positivi che negativi.

Le DNA topoisomerasi si differenziano tra loro per il meccanismo di
azione con il quale cambiano la topologia del DNA. Esistono due
principali strategie:

\begin{enumerate}
\def\labelenumi{\arabic{enumi}.}
\itemsep1pt\parskip0pt\parsep0pt
\item
  \textbf{rotazione controllata}, l'enzima introduce una singola rottura
  su un filamento della doppia elica e permette la rotazione su sé
  stesso del filamento intatto per un numero variabile di giri, fino a
  che l'attrito tra il DNA e l'enzima induce la rilegazione del
  filamento tagliato;
\item
  \textbf{strand passage}, consiste nel creare una rottura singola o a
  doppio filamento sul DNA, allargare l'interruzione prodotta e far
  passare l'altro filamento o la doppia elica attraverso la rottura che
  successivamente verrà risaldata.
\end{enumerate}

Le topoisomerasi vengono classificate sulla base della sequenza a.a.
osulla base del meccanismo di reazione.

In base al meccanismo di reazione si gli enzimi si suddividono in:

\begin{itemize}
\itemsep1pt\parskip0pt\parsep0pt
\item
  \textbf{topoisomerasi di tipo I}, sono in genere dei monomeri che
  tagliano un solo filamento del DNA al 5', creando in questo modo
  un'apertura mediata dalle interazioni di diversi domini dell'enzima
  con la doppia elica, attraverso la quale può passare l'altro filamento
  o una doppia elica. Questo meccanismo permette di eliminare con grande
  efficienza strutture annodate sul DNA e di rilassare esclusivamente
  superavvolgimenti negativi. Le topoisomerasi di tipo I, a seconda che
  formino un legame tirosina-5'fosfato o tirosina-3'fosfato, sono divise
  in \emph{tipo A} e \emph{tipo B} rispettivamente. Le tipo A richiedono
  anche ioni Mg$^2$$^+$ o Zn$^2$$^+$.
\end{itemize}

(immagine p41)

\begin{itemize}
\itemsep1pt\parskip0pt\parsep0pt
\item
  \textbf{topoisomerasi di tipo II}, sono in genere dei dimeri o
  multimetri che introducono un tagio su entrambi i filamenti del DNA
  con le due tirosine covalentemente legate all'estremità 5' e portano
  avanti le modificazioni topologiche facendo passare un secondo tratto
  a doppia elica attraverso la rottura. Il filamento tagliato si chiama
  segmento G (gate), il segmento che passa attraverso l'apertura si
  chiama segmento T (transport). Il meccanismo di azione della top II
  viene definito a \textbf{doppio cancello}. In questo processo il
  segmento T del DNA entra nella parte superiore dell'enzima, che si
  apre per accoglierlo; dopo l'apertura della doppia elica del segmento
  G, il segmento T attraversa tutto l'enzima, che si apre nella parte
  inferiore, rimuovendo in questa maniera due superavvolgimenti alla
  volta. Questo meccanismo richiede l'apporto di energia sotto forma di
  ATP per promuovere le modificazioni del complesso enzima-DNA (ma non
  per tagliare e risaldare i due filamenti). Queste topoisomerasi
  rilassano sia superavvolgimenti positivi che negativi.
\end{itemize}

(immagine top II)

In generale, le topoisomerasi sono coinvolte nei processi metabolici
associati alla separazione dei filamenti di DNA come la replicazione, la
trascrizione, la ricombinazione ed il riparo dell'acido nucleico. I due
filamenti del DNA, durante questi processi, devono essere separati
temporaneamente per diventare accessibili alle polimerasi o ai
componenti del complesso della trascrizione, di conseguenza il DNA
subisce degli stress torsionali che portano a dei superavvolgimenti
della molecola.

\textbf{Nomenclatura:} tutte le topoisomerasi di tipo I hanno un numero
dispari (I, III, V), mentre quelle di tipo II hanno un numero pari (II,
IV, VI). Le topoisomerasi con attività di superavvolgimento vengono
invece distinte in girasi (topoisomerasi di tipo II che introducono
superavvolgimenti negativi) e girasi inversa (topoisomerasi di tipo I
che introducono superavvolgimenti positivi).

La girasi di \emph{E. coli} è una topoisomerasi di tipo II ed è l'unica
in grado di introdurre superavvolgimenti negativi.

\subsection{La replicazione del DNA}\label{la-replicazione-del-dna}

Per trasmettere alle cellule figlie lo stesso patrimonio genetico della
cellula madre, l'informazione genetica contenuta nel DNA deve essere
duplicata con la massima fedeltà possibile, in quanto una variazione
nella sequenza nucleotidica genererebbe cambiamenti nel messaggio
genetico e dunque mutazioni.

Sulla base della struttura del DNA furono ipotizzati 3 possibili modelli
replicativi:

\begin{enumerate}
\def\labelenumi{\arabic{enumi}.}
\itemsep1pt\parskip0pt\parsep0pt
\item
  semiconservativo;
\item
  conservativo;
\item
  dispersivo.
\end{enumerate}

Il meccanismo di replicazione fu determinato nel 1958 da Meselson e
Stahl utilizzando la tecnica di \textbf{centrifugazione su gradiente di
densità}.

Inizialmente fecero crescere cellule di \emph{E. coli} in un terreno di
coltura a cui vennero aggiunti dei sali d'ammonio che contenevano
l'isotopo pesante dell'azoto $^1$$^5$N, al posto dell'isotopo normale
$^1$$^4$N. Le molecole di DNA che contengono $^1$$^5$N hanno una
``densità'' maggiore rispetto a quella di molecole di DNA contenenti
$^1$$^4$N e, dopo centrifugazione all'equilibrio in soluzioni
concentrate di cloruro di cesio, si troveranno verso il fondo della
provetta da centrifugazione.

Le cellule di \emph{E. coli} vennero fatte duplicare per molte
generazioni in un terreno contenente sali pesanti dell'azoto, affinchè
tutte le cellule avessero molecole di DNA contenenti $^1$$^5$N. A quel
punto, le cellule furono raccolte tramite centrifugazione, lavate per
togliere ogni traccia di terreno contenente $^1$$^5$N, diluite in un
terreno contenente sali d'ammonio con l'isotopo leggero $^1$$^4$N e
fatte crescere in queste condizioni per due generazioni.

Se fosse stata vera l'ipotesi semiconservatica, dopo una generazione,
tutte le cellule di \emph{E. coli} che avevano replicato il loro DNA
avrebbero dovuto avere molecole di DNA contenente un filamento parentale
marcato con $^1$$^5$N e un filamento neosintetizzato marcato con
$^1$$^4$N. Tutte le molecole avrebbero dovuto avere, quindi, una densità
intermedia rispetto a quella di un DNA a doppia elica tutto pesante o
tutto leggero.

Se invece fosse risultata vera l'ipotesi conservativa della replicazione
del DNA i ricercatori si aspettavano che alla prima generazione metà del
DNA sarebbe stato pesante e metà leggero.

(immagine 05)

\subsubsection{Il modello del replicone}\label{il-modello-del-replicone}

Un \textbf{replicone} è un'unità di DNA coinvolta nella replicazione.

Questo modello postula che l'inizio della replicazione del DNA è
geneticamente controllato da sequenze specifiche in \emph{cis} sul DNA
(chiamate \textbf{replicatori}). Queste sequenze determinano dove può
partire la replicazione del DNA interagendo con specifiche proteine
(\textbf{iniziatori}) che agiscono in \emph{trans} e collegano il
processo replicativo con la crescita e la divisione cellulare.

La replicazione del DNA necessita che i due filamenti si separino ed
inizia in punti specifici della molecola detti \textbf{origine di
replicazione}. La replicazione può essere \textbf{unidirezionale} o
\textbf{bidirezionale} a seconda del numero di forche replicative.

\emph{Unidirezionale} = una sola forca replicativa si allontana
dall'origine (rara, es. plasmide ColE1 di E. coli, fago $\Phi$ 29 e gli
adenovirus). \emph{Bidirezionale} = due forche replicative procedono
contemporaneamente allontanandosi dall'origine in direzioni opposte.

(immagine 06)

Quando il DNA in replicazione viene osservato al microscopio elettronico
la regione replicata appare come una \textbf{bolla di replicazione}
all'interno del DNA non replicato. La bolla si estende fino a quando non
conterrà l'intero replicone

In tutti i cromosomi degli eucarioti esistono numerose origini di
replicazione. L'osservazione di ``bolle'' di replicazione di dimensioni
diverse indica due caratteristiche importanti delle origini dei
cromosomi degli eucarioti:

\begin{enumerate}
\def\labelenumi{\arabic{enumi}.}
\itemsep1pt\parskip0pt\parsep0pt
\item
  l'accessione delle origini non è simultanea, ma esistono delle origini
  ``early'' che vengono accese presto durante la fase di replicazione
  del DNA, mentre altre origini ``late'' vengono attivate a tempi
  successivi;
\item
  Le forcelle di replicazione che avanzano in direzioni opposte sul
  cromosoma possono fondersi tra di loro.
\end{enumerate}

(immagine 07)

I repliconi batterici, come quello di \emph{E. coli}, sono normalmente
circolari e la replicazione procede in modo bidirezionale a partire da
una singola origine chiamata \emph{OriC}.

OriC di E. coli (genoma 4,2 Mbp) è una regione di 248 bp che contiene al
suo interno regioni che vengono riconosciute da proteine nelle fasi
iniziali della replicazione:

\begin{itemize}
\itemsep1pt\parskip0pt\parsep0pt
\item
  3 regioni di 13 bp ricche in AT;
\item
  4 regioni di 9 bp (siti di legame per l'iniziatore DnaA).
\end{itemize}

La particolarità di essere ricche in A e T è risultata, come vedremo,
una caratteristica comune alle origini di replicazione identificate in
altri organismi ed è legata al fatto che un'origine di replicazione, per
funzionare, deve potersi aprire ed è necessaria meno energia per
denaturare localmente una regione di DNA ricca in A e T piuttosto che in
C e G.

\subsubsection{Il processo replicativo}\label{il-processo-replicativo}

\paragraph{La DNA polimerasi}\label{la-dna-polimerasi}

Il modello della replicazione semiconservativa del DNA suggerisce
immediatamente l'esistenza di enzimi in grado di catalizzare la
polimerizzazione dei nucleotidi. Se durante la replicazione ciascun
filamento serve da stampo per la neosintesi di un filamento
complementare, c'è da aspettarsi che nella cellula esistano delle
attività in grado di catalizzare la formazione del legame
fosfodiesterico tra nucleotidi in modo DNA stampo-dipendente. Tale
ipotesi portò all'identificazione della prima \textbf{DNA polimerasi}.

La sintesi del DNA ha delle precise richieste biochimiche: i substrati
della reazione sono i \textbf{deossiribonucleosidi trifosfati (dNTP)}, e
un \textbf{complesso innesco-stampo} costituito da uno stampo
rappresentato da un filamento di DNA e da un innesco che può essere un
tratto più o meno lungo di DNA che porti una estremità 3'OH che la DNA
polimerasi è in grado di allungare. Tutte le DNA polimerasi non sono in
grado di iniziare la sintesi della catena nucleotidica, ma necessitano
di una estremità 3'OH da allungare.

La DNA polimerasi aggiunge un nuovo nucleotide catalizzando la
formazione del legame fosfodiesterico nel rispetto della regola della
complementarietà tra le basi con il filamento di DNA stampo.

nella formazione di ciascun legame fosfodiesterico, il fosfato in
posizione $\alpha$ del dNTP viene legato al 3'OH dell'innesco portando
alla liberazione di pirofosfato. L'energia libera di questa reazione è
piuttosto modesta, e l'energia addizionale che spinge la reazione verso
la polimerizzazione è fornita dall'idrolisi del pirofosfato da parte di
un enzima chiamato \textbf{pirofosfatasi}.

Le DNA polimerasi possiedono un'attività sintetica con polarità 5'
$\rightarrow$ 3'.

(immagine 08)

La struttura delle DNA polimerasi replicative è paragonata a quella di
una mano parzialmente chiusa dove si distinguono 3 domini: pollice, dita
e palmo. Il DNA si lega ad una grande fessura compresa tra i 3 domini.
Il sito catalitico si trova nel palmo, mentre le dita sono coinvolte nel
posizionamento corretto dello stampo nel sito attivo. Il pollice lega il
DNA mentre esce dall'enzima ed è importante per la processività
dell'enzima. L'attività esonucleasica risiede in un dominio indipendente
con un proprio sito attivo.

\paragraph{Fedeltà del processo
replicativo}\label{fedeltuxe0-del-processo-replicativo}

La fedeltà del processo replicativo è di circa 1 errore ogni
10$^9$-10$^1$$^0$ nucleotidi polimerizzati.

Un sistema basato solo sulla stereochimica di coppie di basi che
rispettino la regola della complementarietà non sarebbe in grado di
raggiungere l'accuratezza indicata sopra. La selettività delle
polimerasi è piuttosto limitata (1 errore ogni 10$^5$ circa) per la
presenza di forme tautomeriche delle basi azotate. Questi errori sono
rimosis da un'attività esonucleasica 3'$\rightarrow$ 5' che ha, quindi,
una polarità invera alla direzione di sintesi del DNA. Tale attività ha
la funzione di agire come correttore di bozze \textbf{(attività
proofreading)}. Quando la DNA polimerasi rileva la presenza di un
appaiamento non corretto tra le coppie di basi, il complesso
stampo-innesco si allontana dal sito catalitico di polimerizzazione
della DNA polimerasi e si avvicina al sito esonucleasico con funzione
proofreading. Tale attività elimina il nucleotide errato e permetta alla
polimerasi di riprendere la sintesi senza che il complesso ternario
innesco-stampo-enzima si sia dissociato.

La fedeltà della replicazione del DNA può essere dunque riassunta in:

\begin{enumerate}
\def\labelenumi{\arabic{enumi}.}
\itemsep1pt\parskip0pt\parsep0pt
\item
  fedeltà;
\item
  attività di proofreadin;
\item
  riparazione degli errori.
\end{enumerate}

\paragraph{Forcella di replicazione: sintesi del filamento continuo e
del filamento
discontinuo}\label{forcella-di-replicazione-sintesi-del-filamento-continuo-e-del-filamento-discontinuo}

Poichè tutte le DNA polimerasi possono polimerizzare il DNA soltanto in
direzione 5' $\rightarrow$ 3', ciò crea un problema nella progressione
della forcella replicativa. Soltanto uno dei due filamenti di neosintesi
può essere sintetizzato in modo continuo seguendo la direzione della
forcella replicativa: questo filamento di neosintesi viene chiamato
\textbf{filamento continuo} o \textbf{leading strand}. Per rispettare la
polarità di sintesi del DNA (5' $\rightarrow$ 3') l'altro filamento deve
essere sintetizzato in modo discontinuo e genera quello che viene
definito \textbf{filamento ritardato} o \textbf{lagging strand}. Tale
filamento è definito ritardato perchè la sua la sua sintesi può iniziare
solo dopo che il progredire della forcella di replicazione ha generato
sufficiente DNA a singolo filamento da far partire la sintesi del
filamento ``lagging''. La sintesi del filamento ritardato si realizza
attraverso la generazione di frammenti discontinui di DNA che vengono
chiamati \textbf{frammenti di Okazaki}.

(immagine 12)

La sintetizzazione discontinua del frammento lagging implica la
necessità di dell'enzima \textbf{DNA ligasi}, un enzima in grado di
saldare il 3'OH di un frammento di Okazaki con il 5'-fosfato del
frammento di Okazaki sintetizzato precedentemente.

\paragraph{Innesco della sintesi di
DNA}\label{innesco-della-sintesi-di-dna}

L'innesco della sintesi di ogni frammento di Okazaki, così come
l'innesco del filamento a un'origine di replicazione, devono prevedere
la sintesi di un iniziatore (\emph{primer}) per offrire alla DNA
polimerasi il complesso stampo-iniziatore descritto poco sopra.
L'innesco della sintesi del DNA è fornito da corte (4-12 nucleotidi)
molecole di RNA sintetizzate da un enzima, denominato \textbf{DNA
primasi}.

\paragraph{Le proteine replicative di E.
coli}\label{le-proteine-replicative-di-e.-coli}

(immagine 09)

Il primo evento molecolare consiste nel riconoscimento di \emph{ori C}
da parte della proteine \textbf{DnaA}. DnaA si attacca inizialmente alle
ripetizioni di 9 pb: la proteina si lega in modo cooperativo formando
una specie di nucleo centrale proteico intorno al quale si avvolge il
DNA di \emph{oriC}. Successivamente, DnaA si lega anche alle 3
ripetizioni di 13 pb facilitando la denaturazione localizzata del DNA in
quella regione e permettendo l'assemblaggio di altre proteine
replicative.

Il passaggio successivo consiste nel caricamento di \textbf{DnaB} e
\textbf{DnaC} a livello della bolla di denaturazione creatasi nella
regione di \emph{oriC} dando origine a un complesso proteico, denominato
\textbf{complesso di pre-innesco} della sintesi del DNA. DnaB possiede
un'attività elicasica, per cui consumando ATP è in grado di separare i
due filamenti di DNA. Lo svolgimento del DNA sia nella fase iniziale che
nel successivo processo di allungamento della sintesi del DNA genera una
tensione torsionale del DNA che è risolta da enzimi in grado di
modificare la topologia del DNA e chiamati ``DNA topoisomerasi''.

La bolla di denaturazione creatasi tende spontaneamente a rinaturare:
per evitare questo il DNA a singolo filamento originatosi viene
stabilizzato dal legame della \textbf{proteina SSB} alla regione di
ssDNA. SSB è una proteina che ha un'affinità maggiore per il DNA a
singolo filamento che a doppio filamento.

Le proteine DnaA e SSB interagiscono con il DNA in modo cooperativo:
questo sta a indicare che il legame tra una molecola di SSB all'ssDNA
facilita il legame di una seconda molecola di SSB alla stessa molecola
di DNA.

(immagine 10)

Tutte le DNA polimerasi necessitano di un'estremità 3'OH pre-formata per
poter aggiungere i nucleotidi successivi. In E. coli e negli eucarioti,
però, gli inneschi che forniscono il 3'OH che può essere allungato dalla
DNA polimerasi sono corte molecole di RNA; tali molecole funzionano da
primer sia per far partire la sintesi continua a un'origine del
filamento leading, sia per iniziare la sintesi di tutti i grammenti di
Okazaki. La \textbf{DNA primasi} di E. coli, codificata dal \textbf{gene
dnaG}, è costituita da un singolo polipeptide e la sintesi degli
\textbf{RNA primer} costituisce il primo effettivo evento di sintesi
nella replicazione del DNA.

La DNA polimerasi replicativa di E. coli è la \textbf{DNA polimerasi III
oloenzima}, una macchina proteica molto complessa formata da numerose
subunità che si assemblano in modo sequenziale a formare un dimero
catalitico. Ci sono due copie del nucleo catalitico, che è formato da 3
subunità: \textbf{$\alpha$} (attività DNA polimerizzante),
\textbf{$\varepsilon$} (attività esonucleasica correttore di bozze
3'$\rightarrow$ 5') e \textbf{$\theta$} (stimola la esonucleasi). Ci
sono anche due copie delle subunità \textbf{$\tau$}, che media la
dimerizzazione dei 2 nuclei catalitici. Questi possono assemblarsi sul
DNA solo dopo che un complesso di 5 proteine, chiamato
\textbf{``complesso $\gamma$''}, è riuscito a caricare sul DNA due copie
della subunità $\beta$; per tale processo il complesso $\gamma$, che è
una \emph{ATPasi DNA-dipendente}, richiede e consuma ATP. La subunità
\textbf{$\beta$} forma un omodimero con una forma a ciambella in grado
di abbracciare il singolo filamento di DNA che passa nel foro della
ciambella.

Il dimero $\beta$ ha una struttura a forma di anello con un diametro di
80 A ed una cavità di 35 A. Lo spazio tra l'anello proteico ed il DNA è
occupato da H$_2$O. L'anello $\beta$ è legato al DNA, ma non entra a
contatto direttamente con il DNA; scivola sul DNA stabilendo o
eliminando contatti con le molecole di H$_2$O.

I due nuclei catalitici presenti nel modello dimerico di replicazione
del DNA replicano contemporaneamente il filamento leading e il filamento
lagging della doppia elica.

L'aspetto più importante del modello è che la sintesi del filamento
leading induce la formazione, sull'altro filamento, di un'ansa che
fornisce lo stampo per la sintesi del filamento ritardato.

(Figura 6.19, p 161)

Il replisoma consiste dunque di un \textbf{dimero asimmetrico} formato
da \textbf{2 nuclei catalitici} (DNA polimerasi III) e da proteine
associate necessarie per la dimerizzazione e la funzione catalitica
della polimerasi (900 kDa):

\begin{itemize}
\itemsep1pt\parskip0pt\parsep0pt
\item
  2 copie di \textbf{nuclei catalitici} ($\alpha$, $\varepsilon$,
  $\theta$);
\item
  2 copie della proteina $\tau$, responsabile della dimerizzazione;
\item
  2 copie dell'anello \textbf{``clamp'' (pinza)}, formato da un
  omodimero della subunità $\beta$. Questo mantiene i nuclei catalitici
  sul filamento stampo, si lega al DNA ed è responsabile della
  processività della polimerasi;
\item
  1 complesso $\gamma$ o \textbf{``clamp loader''} formato da 5
  proteine, carica l'anello $\beta$ sul DNA.
\end{itemize}

Il clamp loader utilizza ATP per caricare l'anello $\beta$ sul complesso
stampo-primer. A questo punto $\beta$ cambia conformazione e lega il
nucleo polimerasico. Il dimero $\tau$ lega il nucleo polimerasico e
svolge la funzione di dimerizzazione legando un secondo nucleo
polimerasico associato ad un altro anello $\beta$. Ciascuno dei
complessi catalitici sintetizza 1 dei nuovi filamenti di DNA. Durante la
sintesi del filamento guida la polimerasi resta sempre associata allo
stampo, mentre durante la sintesi del filamento ritardato la polimerasi
si associa e si dissocia alla fine della sintesi di ciascun frammento di
Okazaki, per poi riassociarsi all'innesco sintetizzato dalla primasi
nella regione dell'ansa. Il ``clamp loader'' resta associato al nucleo
polimerasico che sintetizza il filamento ritardato.

(immagine 11)

Quando un frammento di Okazaki è completo l'anello $\beta$ viene aperto
dal complesso $\gamma$ rilasciando l'ansa. $\gamma$ è una pinza
molecolare che tramite l'utilizzo di ATO riesce a modificare la
conformazione di $\beta$. La polimerasi del filamento ritardato si
sposta da un anello $\beta$ a quello successivo senza dissociarsi dal
complesso replicativo.

E. coli presenta dunque più DNA polimerasi:

\begin{itemize}
\itemsep1pt\parskip0pt\parsep0pt
\item
  \textbf{DNA polimerasi I} (103 kDa). Formata da un'unica subunità con
  due domini. Il frammento di Klenow (68 kDa) ha attività polimerasica
  ed esonucleasica 3' $\rightarrow$ 5'. Il dominio con attività
  esonucleasica 5' $\rightarrow$ 3' (35 kDa) rimuove i primer di RNA e
  poi riempie i vuoti tra i frammenti di Okazaki. Riempie anche i vuoti
  che si formano durante la riparazione del DNA. È l'unica con attività
  esonucleasica 5' $\rightarrow$ 3';
\item
  \textbf{DNA polimerasi II}. Riempie le interruzioni in seguito a danno
  strutturale (risposta SOS inducibile);
\item
  \textbf{DNA polimerasi III}. E' resposabile dell'allungamento della
  forcella di replicazione. Svolge polimerizzazione in direzione 5'
  $\rightarrow$ 3' e attività esonucleasica in direzione 3'
  $\rightarrow$ 5'. Possiede un nucleo catalitico formato da 3 subunità
  $\alpha$ (attività polimerasica), $\varepsilon$ (esonucleasica 3'
  $\rightarrow$ 5') e $\theta$ (stimolo all'esonucleasi);
\item
  \textbf{DNA polimerasi IV e V}, sintesi di DNA translesione (bypass).
\end{itemize}

Nel DNA completamente replicato non possono venire inglobati i piccoli
frammenti di RNA sintetizzati dalla DNA primasi: questi vengono
principalmente rimossi dall'attività esonucleasica 5' $\rightarrow$ 3'
della DNA polimerasi I che, mentre elimina gli RNA primer, è in grado di
riempire i buchi o ``gap'' che essa stessa genera. I ``nick'', cioè le
interruzioni che rimangono tra i vari frammenti di Okazaki sono, infine,
saldati dall'azione della \textbf{DNA ligasi} che unisce i frammenti
legando l'estremità 3'-OH di un frammento con il 5'-fosfato del
frammento seguente.

La reazione ligasica avviene in due stadi. Inizialmente l'enzima forma
un complesso con l'AMP e si lega al 5'-fosfato del DNA formando il
\emph{complesso ligasi-AMP-5'P-DNA}. Successivamente viene formato un
legame fosfodiestere tra il 3'-OH ed il complesso ligasi-AMP-5'P-DNA.
L'enzima di E. coli utilizza NAD come cofattore in sostituzione ad ATP,
mentre la ligasi 16 del fago T4 utilizza ATP.

(immagine 13)

La terminazione della replicazione di E. coli è controllata da
ripetizioni di una breve sequenza di 23 pb, chiamata \textbf{ter}. Le
sequenze \emph{ter} sono posizionate a circa 180° rispetto a \emph{oriC}
e le diverse sequenze fanno terminare l'una o l'altra delle due forcelle
di replicazione. Per impedire il movimento delle forcelle di
replicazione le sequenze \emph{ter} vengono riconosciute da una
\textbf{controelicasi unidirezionale (Tus)} che lega una proteina
chiamata \textbf{TBP (Ter Binding Protein)} che impedisce alla forca
replicativa di procedere oltre.

Durante la terminazione della replicazione è anche necessaria l'azione
di DNA topoisomerasi per separare fisicamente le due molecole di DNA
circolari che si sono originate.

\subsubsection{Meccanismo di replicazione degli
eucarioti}\label{meccanismo-di-replicazione-degli-eucarioti}

Per molti aspetti la replicazione degli eucarioti è molto simile a
quella batterica (semiconservativa, bidirezionale e simidiscontinua).

Negli eucarioti, a causa della maggior quantità di DNA presente, le
origini di replicazioni sono numerose (+ repliconi).

Gli eucarioti presentano un maggior numero di DNA polimerasi. Le
sequenze delle origini di replicazione non sono ben definite a parte
quella del lievito S. cerevisiae (ARS).

La replicazione negli eucarioti avviene durante la fase S del ciclo
cellulare.

Il modello di studio di questo meccanismo è stato il virus delle scimmie
SV40 il quale utilizza per lo più l'apparato replicativo dell'ospite.

La velocità di replicazione negli eucarioti è di 500-3.500 basi/min,
mentre in E. coli è di 50.000 basi/min.

\paragraph{Polimerasi eucariotiche}\label{polimerasi-eucariotiche}

Le DNA polimerasi degli eucarioti si dividono in:

\begin{itemize}
\itemsep1pt\parskip0pt\parsep0pt
\item
  \textbf{replicative}, sono enzimi multimerici. Possiedono una subunità
  responsabile della catalisi e una subunità con funzioni ausiliarie
  (es. sintesi dell'innesco, processività\ldots{}). Replicano il DNA con
  alta fedeltà.
\item
  \textbf{riparative}, sono più semplici. Generalmente sono formate da
  una subunità e distinte in polimerasi riparative ad alta fedeltà e
  polimerasi inclini all'errore (sintesi di translesione).
\end{itemize}

Nella replicazione del DNA nucleare eucariote sono coinvolte 3
polimerasi:

\begin{enumerate}
\def\labelenumi{\arabic{enumi}.}
\itemsep1pt\parskip0pt\parsep0pt
\item
  \textbf{polimerasi $\alpha$} (inizia la sintesi dei nuovi filamenti);
\item
  \textbf{polimerasi $\varepsilon$} (allunga il filamento continuo);
\item
  \textbf{polimerasi $\delta$} (allunga il filamento discontinuo).
\end{enumerate}

La polimerasi $\alpha$ è atipica. Questa polimerasi è capace di iniziare
la sintesi di una nuova catena e può iniziare sia la sintesi del
filamento continuo che di quello ritardato. La pol $\alpha$ è formata da
più subunità: una subunità catalitica (180 kDa) e tre subunità
accessorie. Le subunità accessorie sono formate da una subunità B,
necessaria per l'assemblaggio del complesso, ed altre due subunità più
piccole per l'attività primasica (RNA polimerasi). Il complesso è
chiamato \textbf{polimerasi $\alpha$-primasi}.

\paragraph{Il processo replicativo negli
eucarioti}\label{il-processo-replicativo-negli-eucarioti}

Per dividersi e proliferare, tutte le cellule eucariotiche devono
eseguire correttamene un programma genetico, definito ciclo cellulare,
che sovrintende alla corretta replicazione e segregazione del materiale
ereditario nelle cellule figlie.

La selezione e preparazione delle origini di replicazione del DNA negli
eucarioti inizia quando le cellule escono dalla mitosi e prosegue nella
fase G1 del ciclo cellulare. In questo lasso di tempo si forma su
ciascuna origine un \textbf{complesso di pre-replicazione (pre-RC)}.

Il costituente principale del pre-RC è il complesso ORC. Il complesso
pre-RC comincia a formarsi quando al complesso ORC si agganciano due
proteine chiamate \textbf{Cdc6} e \textbf{Cdt1}; queste sono richieste
per il caricamento del complesso dell'elicasi, nota come
\textbf{MCM2-7}. L'attività delle proteine chinasi CDK trasforma il
pre-RC in quello che è chiamato \textbf{pre-initiation complex}, cioè
una macchina proteica più complessa che diventa competente per la
replicazione e innesca la sintesi vera e propria del DNA. Eventi di
fosforilazione catalizzati dalle CDK provocano il distacco di Cdc6 e
Cdt1, e il contemporaneo aggancio di altre proteine.

(immagine 16)

Durante la fase S viene attivata l'elicasi. In questo processo sono
coinvolte 2 proteine chinasi, \textbf{CDK (ciclin dependent kinase)} e
\textbf{DDK (Dbf4, dependent kinase)}, inattive durante la fase G1. CDK
attiva \textbf{Sld2} e \textbf{Sld3}, mentre DDK attiva Mcm2-7.

Sld2 e Sld3, quando vengono fosforilate, legano \textbf{Dpb11} ed
insieme facilitano il legame di \textbf{GINS} e \textbf{Cdc45}
all'elicasi.

(immagine 14)

Nella \emph{fase G1} Mcm2-7, non attiva, viene caricata intorno al dsDNA
come dimero (2 elicasi). Nella \emph{fase S}, dopo il legame di Cdc45 e
GINS, un filamento di DNA viene espulso dal canale centrale di ciascuna
elicasi. Le interazioni tra le due elicasi vengono eliminate.

(immagine 15)

L'inizio della sintesi di DNA sulla leading strand richiede l'azione del
\textbf{complesso DNA polimerasi $\alpha$-primasi} che porta alla
sintesi del primo RNA primer (circa 10 bp) e 20-30 basi di DNA (iDNA =
iniziatore).

A questo stadio avviene una reazione definita \textbf{scambio delle
polimerasi}. Pol $\alpha$-primasi viene sostituita dalla DNA polimerasi
$\delta$ sul 3'OH della catena nascente di DNA. Per effettuare questo
scambio intervengono due proteine proteine:

\begin{itemize}
\itemsep1pt\parskip0pt\parsep0pt
\item
  \textbf{PCNA},un trimero, simile all'anello $\beta$ di E. coli;
\item
  \textbf{RFC} (simile al clamp loader $\gamma$ di E. coli).
\end{itemize}

La pinza PCNA viene caricata sul DNA per azione di RFC, il quale
utilizza ATP per aprire l'anello PCNA.

Pol $\alpha$-primasi viene rimpiazzata da pol $\delta$ sul filamento
lento e pol $\varepsilon$ sul filamento guida.

Nella sintesi del DNA degli eucarioti la processività della pol $\delta$
e $\varepsilon$ dipende da PCNA, mentre nei procarioti dipende
dall'anello $\beta$.

(immagine 17)

Successivamente interviene \textbf{FEN1}, una esonucleasi, a rimuovere i
primer di RNA e, formando un complesso con DNA polimerasi $\delta$
sintetizza DNA in modo analogo alla DNA pol I di E. coli

(immagine 18)

Infine interviene una ligasi che lega i frammenti mediante la formazione
di un legame fosfodiestere.

(immagine 19)

Non è noto un equivalente di $\tau$ di E. coli responsabile della
coordinazione della replicazione negli eucarioti.

Una differenza tra procarioti ed eucarioti è che nei procarioti, non
appena vengono reclutate le proteine iniziatrici in corrispondenza delle
origini, vengono reclutate anche le DNA elicasi e la sintesi inizia.

Nelle cellule eucariote invece, si ha la formazione di un complesso di
pre-replicazione (abilitazione alla replicazione) che avviene in un
altro momento del ciclo (Fase G1). Una volta che la cromatina eucariota
è stata aperta, le proteine iniziatrici riconoscono la sequenza di DNA
all'origine e vi si legano formando il complesso ORC-Cdc6-Cdt1 che,
idrolizzando ATP, carica l'elicasi Mcm2-7. Quando Mcm2-7 è caricato, il
complesso Cdc6-Cdt1 non serve più. Solo quando la cellula entra nella
fase S, l'elicasi si attiva e viene caricata la pol $\alpha$-primasi ed
inizia la sintesi.

Gli eventi di abilitazione alla replicazione sono sotto il controllo
dell'attività CDK/ciclina. La formazione del complesso di
pre-replicazione avviene solo quando l'attività CDK è bassa, nella fase
G1 del ciclo. In questa fase si forma il complesso ORC-Cdc6-Cdt1 e viene
caricato Mcm2-7.

\subsubsection{La replicazione dei
telomeri}\label{la-replicazione-dei-telomeri}

Nei cromosomi eucarioti lineari è sorto un problema legato al meccanismo
di sintesi discontinua del filamento lagging. Mentre la replicazione del
filamento leading continua fino a copiare l'estremità 5', nella
replicazione del filamento lagging rimane il tratto dell'RNA primer 5'
che, una volta rimosso, lascia un segmento di DNA non replicato che non
può essere riempito da alcuna delle polimerasi conosciute.

Le regioni terminali dei cromosomi eucariotici sono note come
\textbf{telomeri} e la loro struttura è molto importante per il
controllo della stabilità del genoma. I telomeri sono strutture
specializzate per proteggere le estremità dei cromosomi lineari.

I telomeri sono costituiti da sequenze di DNA contenenti numerose
ripetizioni in tandem di brevi sequenze ricche in G che lasciano
un'estremità sporgente in 3' (nell'uomo è 5'-TTAGGG-3').

I telomeri hanno una struttura ripetitiva con un filamento ricco in G-T
che si allunga oltre un filamento ricco in C-A. L'estensione G-T rimane
a singolo filamento per 14-16 basi ed è probabilmente prodotta da una
degradazione specificatamente limitata al filamento ricco in C-A.

Il numero di ripetizioni è variabile nelle diverse specie: nell'uomo
sono lunghe 5-10 kbp, mentre nel lievito 300 bp.

I telomeri svolgono 3 funzioni:

\begin{enumerate}
\def\labelenumi{\arabic{enumi}.}
\itemsep1pt\parskip0pt\parsep0pt
\item
  proteggono le estremità del cromosoma;
\item
  permettono al telomero di essere esteso;
\item
  facilitano la riorganizzazione dei cromosomi meiotici.
\end{enumerate}

Legate ai telomeri vi sono anche due proteine, \textbf{TRF1} e
\textbf{TRF2} (nei mammiferi), che riconoscono e legano la sequenza
telomerica, proteggendo così le estremità del DNA.

E' stato osservato che le estremità dei telomeri non sono lineari, ma
assumono una struttura a forma di loop denominata \textbf{T-loop}.
Questa struttura è dovuta all'appaiamento tra le sequenze TTAGGG
dell'estremità 3' che protrude a singolo filamento e il filamento
complementare in un breve tratto denaturato nella ripetitiva adiacente
al duplex. La struttura è stabilizzata dalla formazione di complessi con
numerose proteine implicate nella funzione telomerica e nella protezione
dell'estremità del cromosoma.

Nei mammiferi le proteine sono 6: TRF1, TRF2, hRap1, TIN2, TPP1 e POT1.
Queste formano un complesso detto Schelterina che protegge i telomeri
dall'attività dei sistemi di riparo del danno al DNA.

Il mantenimento dei telomeri è assicurato dall'azione di un particolare
enzima, chiamato \textbf{telomeasi}. La telomerasi è una
ribonucleoproteina costituita da due componenti principali:

\begin{itemize}
\itemsep1pt\parskip0pt\parsep0pt
\item
  una proteina chiamata \textbf{TERT} che agisce come una trascrittasi
  inversa, essendo capace di sintetizzare DNA copiando uno stampo di
  RNA;
\item
  una molecola di RNA stampo, chiamata \textbf{TERC}.
\end{itemize}

La subunità catalitica TERT si associa con altre proteine accessorie
(chiamate Est) a formare la macchina proteica coinvolta nel mantenimento
dei telomeri.

Contrariamente a quanto ci si potrebbe aspettare, la telomerasi non
estende il filamento 5' ritardato più corto ma allunga ulteriormente il
terminale 3' sporgente. Per fare questo il terminale sporgente 3' si
appaia con il TERC della telomerasi, che ha una sequenza complementare
alla sequenza sporgente telomerica: la parte proteica della telomerasi
copia la sequenza del suo RNA allungando di una ripetizione la sequenza
3' sporgente. La successiva traslocazione e il riposizionamente della
telomerasi fanno sì che questo processo possa ripetersi più volte.

L'estensione delle sequenze ripetute da parte della telomerasi permette
di raggiungere un'omeostasi della lunghezza dei telomeri che non ne
altera la funzionalità e impedisce l'erosione dell'informazione
genetica.

Il numero di ripetizioni aggiunte è controllato da proteine ausiliarie
che funzionano come deboli inibitori ed impediscono alla telomerasi di
legare il DNA.

\subsection{Sistemi di riparazione del
DNA}\label{sistemi-di-riparazione-del-dna}

Variazioni nella sequenza nucleotidica del DNA prendono il nome di
\textbf{mutazioni}. Le mutazioni possono essere \emph{spontanee} (o
naturali) o \emph{indotte}, se dovute ad agenti mutageni esterni.

La maggior parte delle mutazioni sono definite \textbf{puntiformi}, in
quanto determinano il cambiamento di un singolo nucleotide. Le mutazioni
puntiformi sono divise in due categorie:

\begin{enumerate}
\def\labelenumi{\arabic{enumi}.}
\itemsep1pt\parskip0pt\parsep0pt
\item
  le \textbf{transizioni}, che sono cambiamenti da una purina a un'altra
  o da una pirimidina a un'altra;
\item
  le \textbf{trasversioni}, che sono cambiamenti da purina a pirimidina
  o da pirimidina a purina.
\end{enumerate}

Se durante il processo replicativo si introduce un appaiamento non
corretto tra i due filamenti del DNA e tale errore non viene riparato,
al secondo ciclo di replicazione del DNA la mutazione viene fissata nel
genoma.

Se il cambiamento nucleotidico cade nella regione di un gene codificante
per una proteina ci sono tre possibilità:

\begin{enumerate}
\def\labelenumi{\arabic{enumi}.}
\itemsep1pt\parskip0pt\parsep0pt
\item
  mutazione \textbf{silente (o sinonima)}, se la mutazione cambia la
  sequenza di una tripletta (codone) codificante per un certo
  amminoacido ma, grazie alla degenerazione del codice genetico, il
  nuovo codone codifica per lo stesso aa codificato dal codone
  originario;
\item
  mutazione \textbf{missenso}, se la mutazione puntiforme causa la
  formazione di un codone con un significato diverso dall'originario con
  conseguente cambiamento di un singolo aa. Questa mutazione può
  alterare le proprietà della proteina determinando una variazione nel
  fenotipo;
\item
  mutazioni \textbf{non senso}, se la sotituzione di un singolo
  nucleotide nella regione codificante di un gene determina la formazine
  di una delle 3 triplette non senso del codice genetico. Questo causa
  la formazione di una proteina tronca.
\end{enumerate}

Nel DNA possono verificarsi anche delle \textbf{delezioni} o
\textbf{inserzioni} di un singolo nucleotide. Questa perdita o aggiunta
di un nucleotide determina uno sfasamento del codice di lettura di un
gene codificante per una proteina. Queste mutazioni vengono chiamate
\textbf{frameshift} e alterano tutta la sequenza di amminoacidi a valle
del punto in cui è avvenuta la mutazione.

Alcune regioni del DNA hanno un'aumentata instabilità genomica a causa
della ripetizione di sequenze nucleotidiche. Se una sequenza di pochi
nucleotidi si ripete più volte in una regione di DNA, durante il
processo replciativo, le sequenze ripetute possono formare appaiamenti
strutturali alternativi con una stabilità termodinamica simile a quella
dell'appaiamento corretto. Le polimerasi possono, quindi, ``scivolare''
durante la replicazione di queste regioni, portando a una contrazione o
a un'espansione delle sequenze ripetute stesse.

Le mutazioni che causano un cambiamento del fenotipo possono essere
divise in due categorie: 1. la perdita di funzione o \textbf{loss of
function} è la conseguenza di una mutazione che abolisce o riduce
l'attività di una proteina; 2. l'acquisizione di funzione o \textbf{gain
of function} si verifica se la mutazione è in grado di conferire alla
proteina alterata codificata dal gene mutato una funzione anomala.

Le cause delle mutazioni possono essere:

\begin{itemize}
\itemsep1pt\parskip0pt\parsep0pt
\item
  errori di replicazione;
\item
  danni chimici e fisici come perdita o alterazione di basi, rottura
  dello scheletro, ecc.;
\item
  inserimento di trasposoni.
\end{itemize}

I danni causati al DNA possono essere dovuti a:

\begin{itemize}
\itemsep1pt\parskip0pt\parsep0pt
\item
  cambiamento di basi singole;
\item
  distorsioni strutturali (da raggi UV, agenti intercalanti, ecc);
\item
  danno all'ossatura del DNA (da siti abasici e da rotture del doppio
  filamento).
\end{itemize}

I processi riparativi possono essere raggruppati essenzialmente in 3
classi:

\begin{itemize}
\itemsep1pt\parskip0pt\parsep0pt
\item
  riparazione diretta del danno, in cui le basi vengono revertite
  chimicamente (es. fotoattivazione);
\item
  riparazione tramite escissione, in cui segmenti più o meno lunghi di
  un filamento di DNA danneggiato vengono eliminati con siccessiva
  sintesi riparativa utilizzando il filamento intatto come stampo;
\item
  riparazione tramite aggiramento del danno (sintesi di translesione).
\end{itemize}

Una prima opera di riparazione dei danni al DNA viene eseguita dalle DNA
polimerasi.

La \textbf{DNA polimerasi III}: questo complesso svolge, grazie alla
subunità $\varepsilon$, un'attività esonucleasica 3' $\rightarrow$ 5'.

L'inserimento di un nucleotide non corretto produce una deformazione
strutturale del filamento di DNA che induce la polimerasi a fermarsi o a
rallentare. Successivamente l'enzima retrocede e rimuove l'errore.

La \textbf{DNA polimerasi I} invece, svolge una funzione
\emph{``proofreading''}. questo enzima consiste di un'unic acatena
polipeptidica (103 kDa) che, mediante trattamento proteolitico può
essere suddivisa in due frammenti:

\begin{itemize}
\itemsep1pt\parskip0pt\parsep0pt
\item
  il maggiore, chiamato \textbf{frammento di Klenow} è dotato di
  un'attività polimerasica 5' $\rightarrow$ 3' e di un'attività
  esonucleasica 3' $\rightarrow$ 5' (correzione delle bozze). I due siti
  attivi sono separati 30 A (separazione spaziale tra il sito dove viene
  aggiunta la base da quello dove viene escissa);
\item
  il minore possiede anche un'attività esonucleasica 5' $\rightarrow$
  3', che determina l'escissione di nucleotidi a valle dell'enzima.
\end{itemize}

\subsubsection{Aggiramento della lesione: sintesi di
translesione}\label{aggiramento-della-lesione-sintesi-di-translesione}

La replicazione del DNA si blocca se trova una lesione (e.g.~dimeri di
T). La Pol III a questo punto si distacca e al suo posto entra la DNA
pol translesione. Copia il DNA con poca fedeltà, e poi si stacca per far
posto alla Pol III.

Le DNA pol inclini all'errore si sostituiscono alla replicativa e
sintetizzano il tratto di DNA utilizzando le stesse proteine ausiliarie
(sintesi di translesione).

Aggirato il danno la pol viene sostituita nuovamente dalla pol
replicativa.

Queste polimerasi introducono errori con una frequenza 10$^-$$^1$,
10$^-$$^3$, ma risolvono il grave problema dell'arresto della sintesi.

Le pol translesione in E. coli sono la DNA pol IV (dinB) e V (umuCD).

La sintesi delle DNA pol translesione è indotta in risposta al danno al
DNA. In E. coli RecA attiva DNA pol V umuD$_2$C a UmuD'$_2$C (è il
sistema SOS, vedi dopo).

(immagine 22)

\subsubsection{Reversione tramite
fotoliasi}\label{reversione-tramite-fotoliasi}

La fotoliasi è un enzima, appartenente alla classe delle liasi, che lega
specificamente i filamenti di DNA danneggiati dall'esposizione a
radiazione ultravioletta, le quali provocano la formazione di
\emph{dimeri di pirimidina} e di 6-4 fotoprodotti.

I dimeri di pirimidina si producono quando due basi azotate adiacenti
(timina e\o citosina) sullo stesso filamento di DNA vengono legate
covalentemente fra di loro. La fotoliasi ha alta affinità per queste
strutture chimiche, alle quali si lega reversibilmente e le ripara.

Questo enzima funziona come un meccanismo di riparazione del DNA quando
la luce di lunghezza d'onda compresa fra 320 e 370 nm lo colpisce
attivandolo. La reazione enzimatica prevede la rottura del dimero e la
ricostituzione della struttura corretta delle basi (fotoriattivazione).

Questo enzima è una flavoproteina che contiene due gruppi cromofori ed
agisce attraverso il trasferimento di elettroni. Nella reazione redox la
molecola FAD agisce da donatore di elettroni, mentre il dimero agisce da
accettore di elettroni.

La fotoliasi è presente e funzionante nei procarioti, è presente negli
eucarioti inferiori come il lievito dove si ritiene abbia però un ruolo
minore, ed è assente nelle cellule umane e nei mamiferi placentati in
genere.

(immagine 23)

\subsubsection{Riparazione per
escissione}\label{riparazione-per-escissione}

I meccanismi di escissione possono essere di due tipi:

\begin{enumerate}
\def\labelenumi{\arabic{enumi}.}
\itemsep1pt\parskip0pt\parsep0pt
\item
  sistemi per \textbf{escissione di nucleotidi (NER)};
\item
  sistemi per \textbf{escissione delle basi (BER)}.
\end{enumerate}

Questi due sistemi condividono le stesse fasi del meccanismo:

\begin{enumerate}
\def\labelenumi{\arabic{enumi}.}
\itemsep1pt\parskip0pt\parsep0pt
\item
  Riconoscimento;
\item
  incisione (una endonucleasi taglia il filamento su entrambi i lati del
  danno);
\item
  rimozione (un'esonucleasi rimuove il DNA);
\item
  sintesi;
\item
  unione.
\end{enumerate}

\paragraph{Sistemi NER in E. coli}\label{sistemi-ner-in-e.-coli}

Questo sistema si basa su diverse proteine chiamate globalmente
\textbf{Uvr} (UvrA, UvrB, UvrC e UvrD).

UvrA-UvrB cercano e riconoscono le distorsioni del DNA.

UvrB fonde (?) il DNA e recluta UvrC (UvrA viene rilasciato) che taglia
il DNA 8 nucleotidi a monte e 3-4 nucleotidi a valle del sito
danneggiato.

L'elicasi UvrD svolge la regione in cui è avvenuto il taglio permettendo
il rilascio del filamento tagliato.

Successivamente intervengono DNA Pol I per sintetizzare nuovamente la
porzione di DNA tagliata e ligasi.

(immagine 23)

\paragraph{Sistemi NER negli
eucarioti}\label{sistemi-ner-negli-eucarioti}

La riparazione per escissione di nucleotidi è coinvolta nella
riparazione di lesioni che provocano una distorsione della doppia elica
del DNA e sono causate da agenti chimico-fisici.

Il meccanismo del NER negli eucarioti si suddivide in due percorsi che
sono distinti nella prima fase per poi convergere in un meccanismo
comune:

\begin{itemize}
\itemsep1pt\parskip0pt\parsep0pt
\item
  il \textbf{GG-NER (Global Genome NER)}, che controlla l'intero genoma
  alla ricerca di eventi che causano una distorsione della doppia elica;
\item
  il \textbf{TCR-NER (Transcription-Couple Repair NER)}, che agisce su
  danno localizzati nelle regioni trascritte dalle RNA polimerasi.
\end{itemize}

Nel GG-NER il complesso proteico \textbf{XPC-hHR23B} più che ricercare
specifiche lesioni sul DNA, individua regioni in cui il corretto
appaiamento tra i due filamenti di DNA è alterato a causa di una
distorsione della doppia elica.

Nel TCR-NER diversi tipi di lesione nelle regioni trascritte determinano
un blocco delle RNA polimerasi che devono essere rimosse dal DNA per
permettere la riparazione dei danni.

Questo evento richiede due proteine specifiche del TCR-NER, chiamate
\textbf{CSA} e \textbf{CSB}.

I passaggi successivi sono comuni sia al GG-NER che al TCR-NER.

Il \textbf{complesso multiproteico TFIIH}, che contiene due polipeptidi
chiamati \textbf{XPB} e \textbf{XPD} con attività DNA elicasica, apre il
DNA per un tratto di circa 30 nt attorno alla lesione. La proteina
\textbf{XPA} conferma la presenza della lesione legandosi a essa,
altrimenti tutta la reazione termina a questo stadio. La regione del DNA
aperta dall'azione dell'elicasi è stabilizzata dal legame con la
proteina \textbf{RPA}, che è un fattore coinvolto anche nella
replicazione del DNA e che ha un'elevata affinità di legame per il DNA a
singolo filamento. \textbf{XPG} e \textbf{ERCC1/XPF} sono due
endonucleasi NER-specifiche che tagliano rispettivamente in 3' e in 5'
il filamento di DNA contenente la lesione, generando un frammento di
circa 30 nt con una estremità 3'OH nel filamento danneggiato. A questo
punto il filamento può essere riparato dai normali enzimi che replicano
il DNA copiando il filamento complementare che non è stato danneggiato.

(immagine 24)

Nell'uomo, lo \emph{Xeroderma pigmentosum}, patologia recessiva
responsabile della ipersensibilità alla luce solare, è dovuta ad un
difetto nelle riparazioni NER a causa della mutazione di uno dei geni
XPA-G coinvolti nel meccanismo.

\paragraph{Sistemi BER negli
eucarioti}\label{sistemi-ber-negli-eucarioti}

Il primo passaggio del processo è il riconoscimento del danno da
riparare.

Nel caso di danni causati da specie reattive dell'ossigeno \textbf{DNA
glicosilasi} specifiche rompono il legame glicosidico tra lo zucchero e
la base azotata danneggiata. Questi enzimi comprimono la doppia elica
del DNA così che la base alterata viene inglobata in una cavità presente
nella struttura tridimensionale della glicosilasi. L'enzima taglia poi
il legame glicosidico che lega la base azotata alterata al deossiribosio
sul DNA. Si crea così un sito abasico (chiamato sito AP).

Successivamente una endonucleasi chiamata APE1 riconosce il sito privo
della base e incide il legame fosfodiesterico. A questo punto il
meccanismo del BER può procedere attraverso due vie:

\begin{enumerate}
\def\labelenumi{\arabic{enumi}.}
\itemsep1pt\parskip0pt\parsep0pt
\item
  \textbf{short patch BER}, preponderante nei mammiferi. Qui la DNA pol
  $\beta$, attraverso l'attività liasica che tale polimerasi possiede,
  rimuove il deossiribosio privo della base e la stessa pol $\beta$
  riempie il buco di un nucleotide così creatosi. Successivamente il
  legame fosfodiesterico è saldato da una ligasi specializzata (DNA
  ligasi 3) con l'aiuto della proteina XRCC1;
\item
  \textbf{long patch BER}, prevede l'azione della DNA pol $\delta$ e
  $\varepsilon$ e di PCNA che allungano il 3'OH di 2-10 nt, attraverso
  una reazione di ``strand displacement''. Il filamento di DNA spelato
  via è poi tagliato da una nucleasi denominata FEN1 che riconosce la
  particolare struttura che si forma durante la reazione. La
  discontinuità sul DNA è poi saldata dall'azione della DNA ligasi 1.
\end{enumerate}

\subsubsection{Riparazione di errori replicativi: MisMatch repair
(MMR)}\label{riparazione-di-errori-replicativi-mismatch-repair-mmr}

Durante la replicaizone del DNA, l'apparato replicativo può compiere
degli errori: può inserire un nucleotide che non rispetta la
complementarietà delle basi, o può provocare delezioni/inserzioni di
nucleotidi in corrispondenza di particolari sequenze sul DNA che sono
ripetute più volte.

Tali errori replicativi sono riparati da un sistema noto come MMR.

\paragraph{MMR in E. coli}\label{mmr-in-e.-coli}

L'errore di appaiamento (mismatch) induce una distorsione della doppia
elica riconosciuta dall'omodimero \textbf{MutS} che si lega in
corrispondenza del mismatch; tale legame è stabilizzato dall'interazione
con l'omodimero \textbf{MutL}.

Un aspetto importante nel MMR è che deve essere distinto il filamento di
DNA parentale da quello di neosintesi, così che la riparazione del
mismatch avvenga su quest'ultimo.

In E. coli tale distinzione si basa sullo stato metilazione del DNA
neosintetizzato: l'adenina delle sequenze ``GATC'' del DNA di E. coli è
normalmente metilata ma, dopo la replicazione, \emph{il filamento
neosintetizzato rimane NON metilato} per una breve finestra temporale.
Il fatto che il filamento parentale sia metilato, mentre quello di
neosintesi non lo è, permette di distinguere quale filamento contenga
l'errore replicativo e debba, quindi, essere riparato.

Il complesso MutS-MutL attiva la proteina \textbf{MutH} che si lega alla
sequenza GATC metilata più vicina e, in funzione della sua attività
endonucleasica, taglia il filamento di neosintesi di fronte alla base
metilata.

L'escissione di un tratto di DNA contenente il mismatch è realizzata
tramite l'azione combinata di un'elicasi che srotola parzialmente il
filamento contenente il mismatch e la successiva azione di una
esonucleasi con la corretta polarità. Il DNA è poi riparato tramite
l'azione coordinata della DNA polimerasi III e della DNA ligasi.

(immagine 26)

\paragraph{MMR negli eucarioti}\label{mmr-negli-eucarioti}

Negli eucarioti il meccanismo del MMR è alquanto complesso e richiede,
oltre all'azione di proteine specifiche, anche la funzione di proteine
normalmente coinvolte nei meccanismi di replicazione del DNA.

Anche negli eucarioti è necessario che venga riconosciuto il filamento
neosintetizzato. Negli eucarioti la metilazione del DNA non sembra
giocare un ruolo altrettanto importante nella discriminazione dei
filamenti, ma probabilmente sono interazioni proteina-proteina tra
fattori di replicazione e proteine dell'MMR a giocare un ruolo rilevante
nella scelta del filamento che deve essere riparato. Il meccanismo di
discriminazione sembra poi essere diverso per il filamento leading e
quello lagging.

L'apparato enzimatico che agisce durante la replicazione del DNA può
compiere alcuni errori. Per esempio, sul filamento neosintetizzato può
venire inserita una base sbagliata, oppure si possono formare delle
piccole bolle dovute allo scivolamento dell'apparato replicativo in
corrispondenza di sequenze nucleotidiche ripetute.

Tali strutture anomale sono riconosciute dai complessi MutS$\alpha$ e
MutS$\beta$ che reclutano, successivamente, i complessi MutL$\alpha$ e
MutL$\beta$ (questi complessi sono omologhi di quelli di E. coli ma
hanno maggiore specificità).

Il filamento di DNA neosintetizzato contenente gli errori replicativi
viene riconosciuto e processato da una esonucleasi (Exo1) che degrada il
filamento di DNA in direzione 5' $\rightarrow$ 3'. Successivamente,
tramite interazioni con l'apparato di replicazione, viene
ri-sintetizzato il filamento di DNA corretto.

(immagine 27)

La perdita di questo meccanismo predispone al tumore non poliposo
ereditario del colon.

\subsubsection{Riparazione di rotture su entrambi i filamenti
(DSB)}\label{riparazione-di-rotture-su-entrambi-i-filamenti-dsb}

La lesione probabilemnte più pericolosa è la formazione di una rottura
su entrambi i filamenti del DNA, chiamata \textbf{DSB} o \emph{Double
Strand Break}.

La riparazione fisica dei DSB avviene principalmente mediante un
meccanismo di ricombinazione tradizionale o mediante giunzione diretta
delle estremità rotte.

Il primo meccanismo è chiamato \textbf{HR} (Homologous Recombination) e
richiede l'azione di numerose proteine ricombinative.

Nel meccanismo di HR, il complesso eterotrimerico Mre11/Rad50/Nbs1
(chiamato \textbf{MRN}) con l'ausilio di altre proteine come
\textbf{Rad51} e \textbf{Rad52}, modifica il DBS originario causando la
degradazione in direzione 5' $\rightarrow$ 3' delle due estremità 5' di
ciascun DSB.

Si generano, così, delle regioni di DNA a singolo filamento che sono
stabilizzate dal legame della \textbf{proteina RPA}. Successivamente le
proteine Rad51, Rad52 e altre proteine ricombinative, scalzano RPA e
formano un filamento nucleoproteico in cui il DNA a singolo filamento è
ricoperto da Rad51.

L'estremità 3'OH del filamento coperto da rad51, con l'aiuto di altre
proteine ricombinative, procede ora alla ricerca di sequenze omologhe di
DNA presenti sul cromosoma omologo non danneggiato, formando la
\textbf{giunzione di Holliday}.

La perdita di Rad51 all'estremità 3'OH innesca la neosintesi di DNA. La
risoluzione degli intermedi di ricombinazione da parte di proteine
specifiche, qui indicate come risolvasi, e l'azione delle DNA ligasi
genera due molecole riparate.

(immagine 28)

Il secondo meccanismo è chiamato \textbf{NHEJ}. Tale reazione è
apparentemente più semplice e richiede un numero più limitato di fattori
proteici.

Nel processo NHEJ le estremità del DSB sono riconosciute da un complesso
noto come Ku70/Ku80 che interagisce con una proteina chinasi, chiamata
DNA-PK. Può accadere che le estremità del DSB siano processate in modo
limitato localmente con il possibile intervento del complesso MRN.
Infine sono reclutate proteine XRCC4 e la DNA ligasi 4 la cui azione
porta alla saldatura diretta delle estremità rotte.

(immagine 29)

\paragraph{Modello della rottura del double
strand}\label{modello-della-rottura-del-double-strand}

Un'esonucleasi genera due estremità 3'-OH sporgenti a singolo filamento.
Queste invadono una regione omologa dell'altro duplex (donatore)
(INVASIONE DEL SINGOLO FILAMENTO) dando origine a una regione di DNA
eteroduplex.

Successivamente si ha la sintesi di nuovo DNA che sostituisce il DNA
danneggiato.

La cattura della seconda estremità mediante l'appaiamento genera una
molecola in cui i due duplex sono connessi tra loro attraverso una
regione eteroduplex e due giunzioni di Holliday (incrocio).

(immagine 30)

\subsubsection{Sistema di Post-Replication Repair (PRR) in E.
coli}\label{sistema-di-post-replication-repair-prr-in-e.-coli}

Il \textbf{sistema SOS} coinvolge la proteina RecA.

RecA, in E. coli, è la proteina chiave richiesta essenzialemente in
tutti i sotto-pathway ricombinativi per invadere la molecola di DNA
omologo e per promuovere l'appaiamento con il filamento complementare.

Durante la crescita in condizioni normali, l'espressione di geni SOS o
\emph{din} (Damage inducible) è repressa tramite il legame di una
proteina repressore, chiamata \textbf{LexA}, su una sequenza nota come
\textbf{``SOS box''} presente sul sito operatore di tali geni.

Come conseguenza a danni al DNA si accumulano regioni di DNA a singolo
filamento (ssDNA): RecA ha una grande affinità per l'ssDNA e il legame a
tali regioni determina la sua attivazione.

La forma attivata di RecA è in grado di legarsi al repressore LexA e di
indurre la sua distruzione tramite l'attività proteolitica che LexA
stessa possiede. Questo causa la depressione, cioè l'attivazione di una
serie di geni i cui prodotti sono richiesti per una corretta risposta a
danni al DNA.

\subsection{Organizzazione e impacchettamento del DNA
eucariotico}\label{organizzazione-e-impacchettamento-del-dna-eucariotico}

\subsubsection{La cromatina}\label{la-cromatina}

Nei nuclei delle cellule eucariotiche il amteriale genetico costituisce
la massa di cromatina la cui organizzazione è soggetta a vistosi
cambiamenti nel corso del ciclo cellulare.

Le molecole di DNA sono associate alle proteine istoniche a formare la
fibra cromatinica da 10 nm di diametro. Queste lunghe moleocle di DNA
sono organizzate in numerose anse della lunghezza media di 30-100 kpb
ancorate alla base su una struttura filamentosa chiamate \textbf{matrice
nucleare}. Queste anse costituiscono domini topologici indipendenti in
quanto capaci individualmente di mantenere o perdere superavvolgimenti.
Queste anse sono ancorate a una ``impalcatura'' centrale del cromosoma
costituita da proteine specifiche.

La fibra cromatinica risulta dall'interazione del DNA genomico con
proteine istoniche e con proteine non istoniche, e subisce notevoli
variazioni della compattazione nel corso del ciclo cellulare.

Nei nuclei interfasici i singoli cromosomi sono indistinguibili e il
materiale genetico nel nucleo si presenta come una rete diffusa di
filamenti (la cromatina) che si possono visualizzare con specifici
coloranti. La colorazione non omogenea della cromatina nei nuclei
interfasici suggerisce l'esistenza di due diverse organizzazioni
strutturali.

Nella maggior parte dello spazio nucleare le fibre cromatiniche appaiono
relativamente disperse, molto meno densamente compattate che nei
cromosomi mitotici, e prendono il nome di \textbf{eucromatina}.

In alcune regioni nucleari invece si osservano delle masse di cromatina
più compatta che prendono il nome di \textbf{eterocromatina}.
L'eterocromatina può essere distinta in due tipi:

\begin{enumerate}
\def\labelenumi{\arabic{enumi}.}
\itemsep1pt\parskip0pt\parsep0pt
\item
  \textbf{eteroctomatina costitutiva}, che resta sempre compatta e
  rappresenta regioni del genoma che, non avendo capacità codificante,
  non vengono mai espresse e che potrebbero avere un ruolo strutturale
  nel cromosoma;
\item
  \textbf{eterocromatina facoltativa}, che è tale solo in alcune
  situazioni, rappresenta regioni che hanno capacità codificante, che
  possono essere trascritta e che possono pertanto diventare
  eucromatina.
\end{enumerate}

Gli \textbf{istoni} sono proteine basiche che rappresentano i veri
costituenti strutturali della cromatina.

La cromatina presenta una struttura ripetitiva che appare come una
collana costituita da grani impilati du un filo, dove i grani
rappresentano i \textbf{nucleosomi}, le unità elementari della struttura
della cromatina.

Le unità monomeriche possono presentare lunghezze diverse dovute alla
variabilità delle dimensioni del DNA linker (DNA che lega un nucleosoma
all'altro, il ``filo'' della collana di perle). Le dimensioni del tratto
di DNA strettamente avvolto attorno al nucleosoma, infatti, sono
strettamente conservate, e sono lunghe 147 pb.

La forte interazione tra DNA e core istonico è mediata da circa 140
legami a idrogeno. Quasi tutti i legami si instaurano con gli atomi di
ossigeno dei legami fosfodiesterici vicini al solco minore; solo 7
legami vengono formati tra le proteine e le basi attraverso il solco
maggiore. Da qui la conclusione che il legame tra DNA e nucleosoma è
forte ma non è sequenza-specifico.

La natura basica degli istoni maschera le cariche negative dei fosfati,
permettendone un avvicinamento fisico, determinato dalla curvatura del
DNA, altrimenti impossibile.

Il nucleosoma è costituito da un tratto di DNA di circa 200 pb associato
con un ottamero di proteine istoniche, che consiste di due copie di
ciascuno degli istoni \textbf{H2A}, \textbf{H2B}, \textbf{H3} e
\textbf{H4}.

L'ottamero di istoni costituisce la struttura centrale o ``nucleo''
(core) del nucleosoma, a forma di cilindro appiattito del diametro di
circa 10 nm. All'intorno del cilindro la doppia elica di DNA compie
quasi due giri.

A questo nucleosoma, ma esternamente rispetto ad esso, si trova
associata una molecola dell'\textbf{istone H1}, che quindi è presente in
quantità pari alla metà di quella degli altri istoni.

L'istone H1 interagisce con il tratto di DNA tra due nucleosomi (DNA
linker) producendo una maggiore adesione del DNA all'ottamero istonico.
Il legame di H1 aumenta il compattamento del DNA sul nucleosoma,
imponendo una costrizione sugli angoli di entrata e uscita del DNA dal
nucleosoma.

Le proteine istoniche, che interagiscono con il DNA carico
negativamente, sono fortemente basiche (oltre 20\% degli aa sono Arg e
Lys). Presentano una struttura conservata costituita da tre regioni ad
$\alpha$-elica denominata \textbf{histone-fold} e da una coda
N-terminale.

Tutti gli istoni sono soggetti a modificazioni covalenti; la maggior
parte delle quali interessano la regione N-terminale. Modifiche
transitorie a livello di numerosi siti di acetilazione, fosforilazione e
metilazione.

In condizioni di bassa forza ioni e in assenza dell'istone H1 si osserva
la struttura meno condensata della cromatina, la ``collana di perle'',
chiamata anche \textbf{``fibra da 10 nm''}. Se analizziamo la cromatina
in condizioni più fisiologiche, essa appare come una \textbf{``fibra da
30 nm''}, chiamata anche \emph{solenoide}, in cui i nucleosomi appaiono
organizzati in una struttura elicoidale con sei nucleosomi per giro.

Il modello strutturale della cromatina che presenta il maggiore
impaccamento vede la formazione di anse bloccate alla base da una
struttura proteica, la \textbf{matrice nucleare} o l'\textbf{impalcatura
(scaffold) del cromosoma}. La \textbf{topoisomerasi II e le }proteine
SMC** \emph{(Structural Maintenance of Chromosome)} sono componenti
essenziali di quese strutture proteiche.

\subsection{La trascrizione}\label{la-trascrizione}

Un gene è una sequenza di DNA contenente le informazioni per la sintesi
di un prodotto indipendente (rRNA, tRNA, RNA regolatori, mRNA
$\rightarrow$ proteina).

Si definisce \textbf{``espressione genica''} il meccanismo attraverso
cui l'informazione presente in una sequenza di DNA viene utilizzata per
produrre un RNA, o un polipeptide, mediante trascrizione e traduzione.
La relazione tra la sequenza di DNA e la sequenza aa è detta
\textbf{codice genetico}. Un gene contiene una serie di codoni che
vengono letti in maniera sequenziale a partire da un sito di inizio fino
a un sito di terminazione.

Mentre nei batteri la trascrizione è contestuale alla traduzione, dato
che non presentano una compartimentazione cellulare, negli eucarioti la
trascrizione avviene nel nucleo mentre le traduzione nel citoplasma.
L'mRNA va incontro a maturazione e solo successivamente viene tradotto.

Il processo di sintesi dell'RNA su uno stampo di DNA è chiamato
\textbf{trascrizione} e l'attore primario dell'intero meccanismo è un
enzima chiamato \textbf{RNA polimerasi}.

La RNA pol è una complessa macchina molecolare formata da varie subunità
che dopo essersi legata al DNA lo apre, creando una zona denaturata,
chiamata \textbf{bolla di trascrizione}, che fornisce all'enzima il
filamento stampo da cui è diretta la sintesi dell'RNA. La bolla si muove
con l'enzima e il DNA si apre via via che l'enzima procede e si richiude
posteriormente.

Il processo di trascrizione comprende tre diverse fasi: inizio,
allungamento e terminazione.

L'RNA prodotto non rimane appaiato al DNA, ma si stacca dallo stampo a
una distanza di pochi nucleotidi da dove è stato aggiunto l'ultimo
ribonucleotide alla catena.

Quando la doppia elica del DNA viene aperta, l'RNA pol usa uno solo dei
due filamenti come stampo.

Il filamento di DNA che funge da stampo è chiamato \textbf{filamento
stampo} ed è complementare al messaggero (RNA), mentre l'altro
filamento, che viene chiamato \textbf{fiamento non stampo} o
\textbf{codificante}, ha la stessa sequenza del messaggero, con le T al
posto delle U.

L'RNA è una catena polinucleotidica a \emph{singolo filamento}. Lo
zucchero è rappresentato dal \textbf{ribosio} anzichè 2'-deossiribosio.
Il residuo 2'-OH rende questa molecola molto reattiva (funzioni
catalitiche). Tra le basi dell'RNA troviamo l'uracile al posto della
timina. Questo filamento assorbe la luce a una lunghezza d'onda di 260
nm.

La sintesi dell'RNA è la stessa sia nei procarioti che negli eucarioti,
ma la regolazione del processo è molto più complessa negli eucarioti.

La trascrizione dei geni sia nei procarioti che negli eucarioti è
regolata da sequenze specifiche di DNA che costituiscono delle regioni
di controllo della trascrizione.

L'RNA pol sintetizza l'RNA in direzione 5' $\rightarrow$ 3'.

L'RNA pol si lega a particolari sequenze sul DNA che si chiamano
\textbf{promotori} e che sono situate all'inizio del gene. Il promotore
contiene anche il nucleotide da cui inizia la sintesi dell'RNA, che
viene definito \textbf{sito d'inizio della trascrizione (TSS)}. La
trascrizione procede poi fino a una particolare sequenza chiamata
\textbf{terminatore} e si definisce \textbf{unità di trascrizione} il
tratto di DNA che va dal promotore fino al terminatore, espresso come
una singola molecola di RNA.

L'RNA pol, a differenza della DNA pol, non ha bisogno di un innesco per
sintetizzare un nuovo filamento di RNA. Dopo il suo posizionamento, il
gruppo 3'OH del primo nucleotide reagisce con il successivo nucleoside
5' trifosfato il cui fosfato in posizione $\alpha$ è usato per formare
il legame fosfodiesterico, mentre i fosfati $\beta$ e $\gamma$ sono
rilasciati come una molecola di pirofosfato.

La bolla di trascrizione è lunga circa 25 pb, ma il tratto che forma un
ibrido tra DNA e RNA è lungo 8-9 pb.

\subsubsection{L'RNA polimerasi}\label{lrna-polimerasi}

La chimica della sintesi degli RNA è la stessa per tutti i tipi di RNA;
catalizza la formazione del legame fosfodiesterico utilizzando ioni
Mg$^2$$^+$ quali cofattori. L'enzima è completamente processivo, per cui
un trascritto completo viene sintetizzato da un'unica RNA polimerasi.
Una volta iniziata la sintesi dell'RNA, la RNA polimerasi si sposta
unidirezionalmente lungo la catena stampo di DNA trascrivendo l'RNA in
direzione 5' $\rightarrow$ 3'. Non necessitano di primer di innesco e
non sono dotate di attività di correzione dei trascritti.

Le RNA polimerasi sono capaci di individuare e trascrivere
selettivamente i geni interagendo, con l'ausilio di altre proteine
(elementi trans), con siti specifici del DNA (elementi cis) presenti
nella regione del promotore della trascrizione. Una volta riconosciuto
un elemento cis, la RNA polimerasi inizia la sintesi dell'RNA.

\paragraph{L'RNA polimerasi di E.
coli}\label{lrna-polimerasi-di-e.-coli}

Il peso molecolare del nucleo (core) dell'enzima è circa di 400 kDa.

Questo oloenzima è costituito da 5 subunità:

\begin{itemize}
\itemsep1pt\parskip0pt\parsep0pt
\item
  2 subunità \textbf{$\alpha$}.
\end{itemize}

Queste subunità sono responsabili dell'assemblaggio del complesso e
contengono due domini che hanno funzioni diverse: il dominio C-terminale
(chiamato $\alpha$CTD) si lega a una zona del promotore chimata
\textbf{UP-element} (elemento a monte), mentre il dominio N-terminale è
responsabile dell'interazione con le altre subunità dell'enzima;

\begin{itemize}
\itemsep1pt\parskip0pt\parsep0pt
\item
  1 subunità \textbf{$\beta$}.
\end{itemize}

Questa contiene il sito catalitico, che sintetizza l'RNA, di cui fanno
parte anche due atomi di Mg$^2$$^+$ essenziali per la sintesi. Uno di
questi atomi è sempre presente nel sito attivo, l'altro viene
trasportato nel complesso dai nucleotidi in entrata;

\begin{itemize}
\itemsep1pt\parskip0pt\parsep0pt
\item
  1 subunità \textbf{$\beta$'}.
\end{itemize}

Questa subunità si lega al DNA in maniera non specifica;

\begin{itemize}
\itemsep1pt\parskip0pt\parsep0pt
\item
  1 subunità \textbf{$\omega$}.
\end{itemize}

Questa subunità ha la funzione di promuovere e mantenere stabile il
complesso.

Per legarsi in maniera specifica al promotore, la polimerasi ha bisogno
di una sesta subunità, chiamata \textbf{$\sigma$}.

L'enzima completo \textbf{$\alpha$$_2$$\beta$$\beta$'$\omega$$\sigma$}
(o \textbf{oloenzima}) ha un peso molecolare di circa 480 kDa.

Il fattore $\sigma$ è costituito da 4 domini, distribuiti sul nucleo
enzimatico, in parte verso l'esterno per riconoscere e legare il
promotore e in parte nella regione interna tra le due parti della
``pinza'', formata dalle sub $\beta$ e $\beta$', occupando parzialmente
il canale dove si posiziona il DNA.

Il fattore $\sigma$ funziona come un vero e proprio fattore di
trascrizione. Con $\sigma$ legato si riduce la capacità globale della
pol di legarsi al DNA, ma aumenta moltissimo la sua affinità per il
promotore. Questo permette all'RNA pol di essere generalmente sempre
legata al DNA (senza $\sigma$) e poter riconoscere con molta precisione
i promotori (quando c'è $\sigma$).

Al suo interno, l'RNA pol, ha un solco lungo circa 55 A con una
larghezza di circa 25 A. Esso può accogliere un tratto di DNA lungo
circa 15-16 bp di doppia elica.

Il DNA è costretto dalla struttura a fare una piegatura di quasi 90°
nella parte posteriore dell'enzima che viene definita ``muro''. Nella
parte superiore si trova il foro di uscita dell'RNA e più a destra il
canale da dove esce il DNA, che si riappaia alla fine della bolla; al
centro c'è una struttura chiamata ``timone'', che contribuisce a tenere
aperta la bolla. Nella parte inferiore c'è unas truttura a ``imbuto''
che permette l'entrata dei ribonucleosidi trifosfati.

(immagine 31)

\paragraph{I promotori e il fattore sigma in
E.coli}\label{i-promotori-e-il-fattore-sigma-in-e.coli}

Per trascrivere un determinato gene l'enzima dve essere in grado di
riconoscerne il sito di inizio con grande precisione.

Questo problema è risolto da due diversi elementi: la sequenza specifica
del DNA chiamta \textbf{promotore} e la **subunità $\sigma$ che si
associa al core della polimerasi. Il fattore $\sigma$ riconosce il
promotore, costituito da sequenze conservate che si trovano a monte del
sito di inizio della trascrizione, ma solo quando è parte integrante
della RNA pol oloenzima.

In E. coli il fattore $\sigma$ più usato è $\sigma$$^7$$^0$ (peso
molecolare di 70 kDa), che riconosce un promotore costituito da due
sequenze conservate, lunghe 6 nucleotidi, che si trovano in posizione
\textbf{-10} (TATA box) e \textbf{-35}.

La sequenza consenso dell'elemento -10 è, sul filamento codificante,
TATAAT. L'elemento -35 ha come consenso la sequenza TTGACA. Il fatto che
questi elementi siano ricchi in A e T, li rende facilmente denaturabili
per formare il complesso aperto.

La forza del promotore è correlata alla maggiore o minore similitudine
delle sequenze che sono localizzate a -10 e -35. Un promotore forte
viene trascritto ogni 2 secondi, mentre uno debole viene trascritto ogni
2 minuti. La sequenza del tratto ``spacer'' (tra la regione -10 e -35)
non è importante.

Il fattore $\sigma$ è diviso in 4 regioni, o domini, chiamati:
\textbf{$\sigma$$_1$, $\sigma$$_2$, $\sigma$$_3$ e $\sigma$$_4$}.

La regione $\sigma$$_2$ riconosce l'elemento -10, mentre la regione
$\sigma$$_4$ riconosce l'elemento -35. $\sigma$$_4$ è costituita da due
$\alpha$-eliche che formano un dominio elica-giro-elica
(helix-turn-helix) che è tipico di molti fattori che si legano al DNA.
Un'elica interagisce con le basi esposte nel solco maggiore della doppia
elica e l'altra si posiziona sopra al solco interagendo con lo scheletro
zucchero-fosfato. L'interazione forte di $\sigma$$_4$ con la regione -35
àncora specificamente l'oloenzima al DNA, mentre la regione $\sigma$$_2$
ha il compito di promuovere l'apertura del DNA.

La regione N-terminale di $\sigma$ ha un ruolo di regolazione molto
importante: se si elimina questo dominio, la proteina tronca è in grado
di legarsi specificamente ai promotori anche in assenza di RNA pol
``core''. Questo dato suggerisce che la regione N-terminale mascheri le
regioni $\sigma$$_2$ e $\sigma$$_4$ quando il fattore non è legato
all'enzima ``core'' e quando non è in grado di occupare promotori in
assenza delle subunità responsabili dell'attività di sintesi dell'RNA.
Una volta che si forma l'oloenzima, un cambiamento conformazionale
espone queste regioni rendendole in grado di legarsi al DNA.

Il nucleo enzimatico ha una certa capacità di legare il DNA in maniera
non specifica grazie all'attrazione elettrostatica fra proteina basica e
acido nucleico. Questo legame è definito sito di legame debole. Il
fattore $\sigma$ riduce la capacità di legare i siti deboli e conferisce
la capacità di legare i promotori. L'oloenzima riconosce un promotore
circa 10$^-$$^7$ volte meglio di una sequenza non specifica.

\subsubsection{Le 3 fasi della
trascrizione}\label{le-3-fasi-della-trascrizione}

\begin{enumerate}
\def\labelenumi{\arabic{enumi}.}
\itemsep1pt\parskip0pt\parsep0pt
\item
  La \textbf{fase di inizio}
\end{enumerate}

In questa fase avviene il riconoscimento del promotore e il DNA si apre
formando la bolla di trascrizione per iniziare la sintesi dell'RNA;

\begin{enumerate}
\def\labelenumi{\arabic{enumi}.}
\setcounter{enumi}{1}
\itemsep1pt\parskip0pt\parsep0pt
\item
  La \textbf{fase di allungamento}
\end{enumerate}

In questa fase la polimerasi e la bolla si muovono lungo il DNA
estendendo la catena di RNA;

\begin{enumerate}
\def\labelenumi{\arabic{enumi}.}
\setcounter{enumi}{2}
\itemsep1pt\parskip0pt\parsep0pt
\item
  La \textbf{fase di terminazione}
\end{enumerate}

In questa fase l'RNA pol si arresta al terminatore, il trascritto di RNA
si dissocia dalla polimerasi e la bolla si richiude.

Sono necessari due enzimi: la topoisomerasi, che rilascia i
superavvolgimenti negativi, e la girasi, che introduce i
superavvolgimenti negativi, per rettificare la situazione che si viene a
creare davanti e dietro la polimerasi.

Il legame che si forma inizialmente tra l'RNA pol e il promotore viene
definito \textbf{complesso chiuso} perchè il DNA non è stato ancora
aperto per formare la bolla.

Una volta che l'enzima è stabilmente legato al promotore, una serie di
cambiamenti conformazionali al suo interno promuovono l'apertura della
bolla e la formazione del \textbf{complesso aperto}. Questa apertura
rende disponibile il filamento stampo al riconoscimento complementare
dei nucleotidi.

Inizialmente l'RNA polimerasi sintetizza, senza distaccarsi dal
promotore, un frammento di RNA che viene rilasciato prima di raggiungere
i 9 nt di lunghezza. Questa \emph{sintesi abortiva} è ripetuta più volte
fino a quando, superata la lunghezza di 9 nt, si forma un ibrido DNA-RNA
(complesso ternario) sufficientemente stabile. La polimerasi può a
questo punto rilasciare il promotore e proseguire nella fase di
allungamento.

La transizione dalla fase di inizio abortivo a quella di allungamento
(non chiara) comporta un cambiamento strutturale.

Durante la fase di allungamento l'enzima si muove lungo il filamento
stampo in direzione 3' $\rightarrow$ 5' sintetizzando la catena di RNA
dal 5' al 3' e muove con sé la bolla di trascrizione; il DNA si apre
nella direzione della sintesi e si richiude alle sue spalle, mantenendo
la bolla di una lunghezza costante.

La terminazione è l'ultima fase della trascrizione in cui l'RNA
polimerasi rilascia il filamento di RNA prodotto e si dissocia dal DNA.

\paragraph{Passaggio complesso aperto-chiuso in E.
coli}\label{passaggio-complesso-aperto-chiuso-in-e.-coli}

Nel \textbf{complesso chiuso}: + il DNA si lega alla superficie con
$\alpha$ e $\sigma$; + $\sigma$$_2$ e $\sigma$$_4$ sono esposti
all'esterno e legano le zone -10 e -35; + l'interno del canale è
occupato da $\sigma$$_1.1$.

Nel \textbf{complesso aperto} avviene l'isomerizzazione dell'RNA:

\begin{itemize}
\itemsep1pt\parskip0pt\parsep0pt
\item
  il DNA è costretto ad una piegatura di 90° che permette allo stampo di
  raggiungere il sito catalitico dove ci sono 2 Mg$^2$$^+$;
\item
  avviene l'apertura dei filamenti di DNA;
\item
  $\sigma$$_1.1$ si sposta;
\item
  $\sigma$$_3$ e $\sigma$$_4$ bloccano il canale di uscita.
\end{itemize}

Infine, nel \textbf{complesso ternario} formato da DNA, RNA ed
oloenzima;

\begin{itemize}
\itemsep1pt\parskip0pt\parsep0pt
\item
  $\sigma$$_3$ e $\sigma$$_4$ liberano il canale di uscita;
\item
  $\sigma$$_4$ si stacca parzialmente dall'elemento -35;
\item
  l'RNA esce dal canale.
\end{itemize}

L'RNA polimerasi cambia dimensione nelle diverse fasi.

Il complesso di inizio contiene sigma e copre 75-80 bp (copre una
lunghezza da -55 a +20 sul DNA). La forma più estesa dell'enzima
potrebbe coprire solo 50 bp, per questo il DNA deve essere curvato (per
coprire una zona più estesa).

Nella fase di allungamento il complesso diminuisce le sue dimensioni
(parliamo della fase iniziale in cui l'RNA ha una lunghezza di circa di
10 bp). Può perdere $\sigma$ e perde i contatti da -55 a -35 (servono
solo per il riconoscimento).

Infine, nella fase allungamento (RNA di 15-20b): l'enzima copre 30-40
basi.

\subsubsection{Terminazione della
trascrizione}\label{terminazione-della-trascrizione}

La trascrizione termina quando l'RNA pol raggiunge delle sequenze
specifiche sul DNA chiamate \textbf{terminatori}.

L'arresto impedisce l'aggiunta di nuovi nucleotidi, l'enzima si dissocia
dal DNA ed è pronto per un nuovo ciclo. E' necessario che tutti i legami
che mantengono uniti i due filamenti nel tratto ibrido DNA-RNA vengano
rotti per pemettere il ri-appaiamento tra i filamenti complementari del
DNA e la dissociazione del complesso.

In E. coli esistono due meccanismi diversi per terminare la
trascrizione: in un primo caso, sequenze denominate \textbf{terminatori
intrinseci} (\textbf{Rho indipendenti}) inducono la polimerasi a
staccarsi dal complesso e a rilasciare la catena di RNA prodotto, senza
l'ausilio di alcun fattore aggiuntivo. Nel secondo caso, denominato
\textbf{terminazione Rho-dipendente}, la sequenza di terminazione non è
in grado di promuovere da sola la dissociazione del complesso, ma
necessita dell'azione della proteina Rho.

\paragraph{I terminatori
Rho-indipendenti}\label{i-terminatori-rho-indipendenti}

Questi terminatori sono costituiti da una corta sequenza palindromica
ricca in G-C seguite da un tratto di 8-9 nucleotidi ricco in A e T.
L'RNA trascritto forma, nella regione palindromica ricca in GC, una
struttura a forcina che destabilizza, insieme al tratto di 8 U, il
complesso trascrizionale.

L'interferenza provocata dalla struttura stem-loop unita alla debolezza
del legame A:U favorisce il distacco del trascritto dallo stampo.

\paragraph{I terminatori
Rho-dipendenti}\label{i-terminatori-rho-dipendenti}

Questo meccanismo è mediato da una proteina chiamata *``Rho* che si
trova in tutti i batteri.

Rho si associa all'RNA nascente in una regione ricca di citosine lunga
circa 40 nt, denominata \textbf{sito di utilizzo di Rho (rut)}. Questo
sito si trova a valle della regione codificante e quindi Rho non
incontrerà sul suo cammino un ribosoma che sta sintetizzando. Una volta
legato, Rho agisce come una elicasi ATP-dipendente che si muove in
direzione 5' $\rightarrow$ 3', per traslocare verso il sito della
trascrizione e promuovere il distacco della polimerasi.

Rho ha una struttura a omoesamero e assume una forma ad anello.

Appena Rho si lega sul trascritto crescente di RNA, il singolo filamento
interagisce con il foro dell'esamero (sito di legame per l'ssDNA) e con
domini della proteina che lo circondano. Una volta formato il complesso,
il legame con l'ATP e la sua idrolisi provocano la traslocazione di Rho
lungo l'RNA, fino a che raggiunge la polimerasi e dissocia il complesso
e l'ibrido DNA-RNA. Oltre ai ``rut'', Rho utilizza per terminare la
trascrizione delle sequenze di terminazione molto simili ai terminatori
intrinseci, ma con una forcina più breve.

Rho lega l'RNA dall'estremità 3' sulla faccia esterna dei domini N-term,
mentre l'estremità 5' è legata da un sito secondario posto all'interno
dell'esamero.

Nei batteri la trascrizione è contestuale alla traduzione. L'accesso di
Rho al sito rut è mascherato dai ribosomi. Una mutazione non senso che
fa staccare i ribosomi prematuramente, fa entrare Rho che blocca la
trascrizione dei geni distali (stesso meccanismo di prima).

(immagine 31)

\subsubsection{Differenza tra mRNA mono e
policistronico}\label{differenza-tra-mrna-mono-e-policistronico}

L'mRNA si dice \textbf{monocistronico} quando porta l'informazione per
un solo gene, ed è una caratteristica tipica degli eucarioti mentre nei
procarioti l'mRNA è molto spesso \textbf{policistronico} e cioè porta
l'informazione per più geni (il trascritto di mRNA corrispondente è in
grado di tradurre per più catene polipeptidiche diverse, in sequenza).

Un \textbf{cistrone} è una sequenza di basi nucleotidiche compresa tra
una tripletta di inizio AUG e una di stop. Il termine deriva dal fatto
che due mutazioni puntiformi diverse possono complementare (ossia dare
un fenotipo normale) solo se sono in cis, ossia nello stesso filamento
(per cui l'altro cistrone è selvatico), e non se sono in trans, in
quanto entrambi i polipeptidi risultano anormali.

La tipica organizzazione policistronica dell'mRNA dei procarioti è
dovuta alla caratteristica organizzazione dei geni in \emph{operoni},
ovvero una serie di geni disposti uno accanto all'altro lungo il
cromosoma che codificano enzimi con funzione correlata tra loro
(solitamente coinvolti nello stesso pathway) controllati da un unico
induttore e che sono trascritti su un'unica molecola di mRNA.

\subsubsection{La trascrizione negli
eucarioti}\label{la-trascrizione-negli-eucarioti}

\end{document}
