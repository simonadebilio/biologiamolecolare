\documentclass[]{article}
\usepackage{lmodern}
\usepackage{amssymb,amsmath}
\usepackage{ifxetex,ifluatex}
\usepackage{fixltx2e} % provides \textsubscript
\ifnum 0\ifxetex 1\fi\ifluatex 1\fi=0 % if pdftex
  \usepackage[T1]{fontenc}
  \usepackage[utf8]{inputenc}
\else % if luatex or xelatex
  \ifxetex
    \usepackage{mathspec}
    \usepackage{xltxtra,xunicode}
  \else
    \usepackage{fontspec}
  \fi
  \defaultfontfeatures{Mapping=tex-text,Scale=MatchLowercase}
  \newcommand{\euro}{€}
\fi
% use upquote if available, for straight quotes in verbatim environments
\IfFileExists{upquote.sty}{\usepackage{upquote}}{}
% use microtype if available
\IfFileExists{microtype.sty}{\usepackage{microtype}}{}
\ifxetex
  \usepackage[setpagesize=false, % page size defined by xetex
              unicode=false, % unicode breaks when used with xetex
              xetex]{hyperref}
\else
  \usepackage[unicode=true]{hyperref}
\fi
\hypersetup{breaklinks=true,
            bookmarks=true,
            pdfauthor={},
            pdftitle={},
            colorlinks=true,
            citecolor=blue,
            urlcolor=blue,
            linkcolor=magenta,
            pdfborder={0 0 0}}
\urlstyle{same}  % don't use monospace font for urls
\setlength{\parindent}{0pt}
\setlength{\parskip}{6pt plus 2pt minus 1pt}
\setlength{\emergencystretch}{3em}  % prevent overfull lines
\setcounter{secnumdepth}{0}


\begin{document}

\section{Biologia Molecolare}\label{biologia-molecolare}

\subsection{Struttura degli acidi
nucleici}\label{struttura-degli-acidi-nucleici}

\subsubsection{Struttura chimica degli acidi
nucleici}\label{struttura-chimica-degli-acidi-nucleici}

Gli acidi nucleici, DNA e RNA, sono costituiti da catene
polinucleotidiche, cioè polimeri lineari di unità chiamate
\textbf{nucleotidi}.

I nucleotidi sono molecole costituite da tre componenti:

\begin{enumerate}
\def\labelenumi{\arabic{enumi}.}
\itemsep1pt\parskip0pt\parsep0pt
\item
  uno \textbf{zucchero pentoso};
\item
  una \textbf{base azotata};
\item
  uno o più \textbf{gruppi fosfato}.
\end{enumerate}

Nel caso del DNA lo zucchero è il \textbf{deossiribosio}, mentre
nell'RNA è il \textbf{ribosio}. I due zuccheri differiscono per un
gruppo -OH presente in posizione 2 del ribosio e che manca nel
deossiribosio. Questo gruppo -OH conferisce instabilità all'RNA.

Le \emph{basi azotate} che si trovano nelgi acidi nucleici naturali sono
di due tipi:

\begin{enumerate}
\def\labelenumi{\arabic{enumi}.}
\itemsep1pt\parskip0pt\parsep0pt
\item
  \textbf{purine}, a doppio anello eterociclico (adenina e guanina);
\item
  \textbf{pirimidine}, a singolo anello (citosina, timina e uracile).
\end{enumerate}

Citosina, adenina e guanina sono comuni sia al DNA che all'RNA, mentre
la timini si torva solo nel DNA e l'uracile solo nell'RNA.

La differenza tra uracile e timina è la presenza di un gruppo metilico
in posizione 5 dell'anello pirimidinico.

(immagine Basi e zuccheri)

Quando una delle suddette basi è legata alla posizione 1 di uno
zucchero, abbiamo i \textbf{nucleosidi}.

Ai nucleosidi possono essere legati da uno a tre gruppi fosfato e, in
tal caso, prendono il nome di \textbf{nucleotidi}.

Il gruppo fosfato conferisce una valenza acida alla molecola. Il gruppo
fosfato può trovarsi legato al carbonio 5' dello zucchero oppure al
carbonio 3'.

I nucleotidi che vengono utilizzati come substrato per la sintesi del
DNA e dell'RNA hanno 3 gruppi fosfato legati in serie sulla posizione 5'
dello zucchero.

Le basi possono subire modificazioni chimiche che sono essenziali per i
processi biologici che coinvolgono il DNA e gli RNAs.

Lo scheletro della catena polinucleotidica è costituito dall'alternanza
di zuccheri e di gruppi fosfato, mentre le basi azotate sporgono
lateralmente da questo scheletro. Ciascuna base è legata alla posizione
1' di uno zucchero da un legame glicosidico che interessa l'N1 delle
pirimidine o l'N9 delle purine. Tra uno zucchero e l'altro c'è un solo
gruppo fosfato e che quindi si tratta di un polimero di nucleosidi
monofosfato. Durante la sintesi degli acidi nucleici, ciascun nucleoside
trifosfato utilizzato come substrato, perde due dei suoi tre gruppi
fosfato. Ciascun gruppo fosfato forma un \textbf{legame estere} con il
C5' di uno zucchero e un secondo legame estere con il C3' dello zucchero
successivo. Questo tipo di legame, chiamato \textbf{fosfodiesterico},
conferisce una sorta di asimmetria (o polarità) alla catena, nel senso
che questa presenta due diverse estremità: da una parte c'è un
nucleotide che ha il C5' libero, mentre il C3' è impegnato nel legame
fosfodiesterico con il nucleotide adiacente; dall'altra la molecola
polimerica termina con un nucleotide che ha il C5' impegnato nel legame
fosfodiesterico con quello adiacente, mentre il C3' è libero. La loro
valenza è comunque impegnata da gruppi -OH o da gruppi fosfato.

\subsubsection{Struttura fisica del DNA}\label{struttura-fisica-del-dna}

Il DNA è formato da due catene polinucleotidiche antiparallele avvolte
l'una sull'altra a formare una struttura a doppia elica con lo scheletro
zucchero-fosfato posto sul lato esterno e le basi impilate all'interno.

Secondo il modello proposto da Watson e Crick le coppie di basi, che con
i loro anelli esterociclici sono strutture essenzialmente piatte, sono
disposte quasi perpendicolarmente rispetto all'asse della doppia elica.
Le basi basi che si affacciano in ciascuna coppia sono sempre una
pirimidina e una purina, ed è sempre presente l'appaiamento A-T/U o C-G
(questo giustifica la regolarità del diametro della doppia elica).

Le coppie A-T sono legate da 2 legami idrogeno, mentre le coppie G-C
sono legate da 3 legami idrogeno.

La doppia elica (duplex) del DNA ha una struttura regolare, destrorsa,
compie un giro completo ogni 34 A e ha un diametro di circa 20 A. La
distanza tra due coppie di basi adiacenti è di 3,4 A e ci sono, quindi,
circa 10,4 coppie di basi per ogni giro di elica. L'elica presenta
un'asimmetria dovuta alla posizione delle molecole di deossiribosio ai
lati delle basi: come conseguenza della opposta polarità 5'$\rightarrow$
3' delle due eliche, i due zuccheri di ciascuna coppia di nucleotidi si
vengono a trovare dallo stesso lato. Questa asimmetria genera nella
doppia elica due solchi di dimensioni diverse detti \textbf{solco
maggiore} e \textbf{solco minore}, con diametro 22 A e 12 A
rispettivamente.

\paragraph{Strutture alternative del
DNA}\label{strutture-alternative-del-dna}

\end{document}
