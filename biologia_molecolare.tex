\documentclass[]{article}
\usepackage{lmodern}
\usepackage{amssymb,amsmath}
\usepackage{ifxetex,ifluatex}
\usepackage{fixltx2e} % provides \textsubscript
\ifnum 0\ifxetex 1\fi\ifluatex 1\fi=0 % if pdftex
  \usepackage[T1]{fontenc}
  \usepackage[utf8]{inputenc}
\else % if luatex or xelatex
  \ifxetex
    \usepackage{mathspec}
    \usepackage{xltxtra,xunicode}
  \else
    \usepackage{fontspec}
  \fi
  \defaultfontfeatures{Mapping=tex-text,Scale=MatchLowercase}
  \newcommand{\euro}{€}
\fi
% use upquote if available, for straight quotes in verbatim environments
\IfFileExists{upquote.sty}{\usepackage{upquote}}{}
% use microtype if available
\IfFileExists{microtype.sty}{\usepackage{microtype}}{}
\ifxetex
  \usepackage[setpagesize=false, % page size defined by xetex
              unicode=false, % unicode breaks when used with xetex
              xetex]{hyperref}
\else
  \usepackage[unicode=true]{hyperref}
\fi
\hypersetup{breaklinks=true,
            bookmarks=true,
            pdfauthor={},
            pdftitle={},
            colorlinks=true,
            citecolor=blue,
            urlcolor=blue,
            linkcolor=magenta,
            pdfborder={0 0 0}}
\urlstyle{same}  % don't use monospace font for urls
\setlength{\parindent}{0pt}
\setlength{\parskip}{6pt plus 2pt minus 1pt}
\setlength{\emergencystretch}{3em}  % prevent overfull lines
\setcounter{secnumdepth}{0}


\begin{document}

\section{Biologia Molecolare}\label{biologia-molecolare}

\subsection{Struttura degli acidi
nucleici}\label{struttura-degli-acidi-nucleici}

\subsubsection{Struttura chimica degli acidi
nucleici}\label{struttura-chimica-degli-acidi-nucleici}

Gli acidi nucleici, DNA e RNA, sono costituiti da catene
polinucleotidiche, cioè polimeri lineari di unità chiamate
\textbf{nucleotidi}.

I nucleotidi sono molecole costituite da tre componenti:

\begin{enumerate}
\def\labelenumi{\arabic{enumi}.}
\itemsep1pt\parskip0pt\parsep0pt
\item
  uno \textbf{zucchero pentoso};
\item
  una \textbf{base azotata};
\item
  uno o più \textbf{gruppi fosfato}.
\end{enumerate}

Nel caso del DNA lo zucchero è il \textbf{deossiribosio}, mentre
nell'RNA è il \textbf{ribosio}. I due zuccheri differiscono per un
gruppo -OH presente in posizione 2 del ribosio e che manca nel
deossiribosio. Questo gruppo -OH conferisce instabilità all'RNA.

Le \emph{basi azotate} che si trovano nelgi acidi nucleici naturali sono
di due tipi:

\begin{enumerate}
\def\labelenumi{\arabic{enumi}.}
\itemsep1pt\parskip0pt\parsep0pt
\item
  \textbf{purine}, a doppio anello eterociclico (adenina e guanina);
\item
  \textbf{pirimidine}, a singolo anello (citosina, timina e uracile).
\end{enumerate}

Citosina, adenina e guanina sono comuni sia al DNA che all'RNA, mentre
la timini si torva solo nel DNA e l'uracile solo nell'RNA.

La differenza tra uracile e timina è la presenza di un gruppo metilico
in posizione 5 dell'anello pirimidinico.

(immagine Basi e zuccheri)

Quando una delle suddette basi è legata alla posizione 1 di uno
zucchero, abbiamo i \textbf{nucleosidi}.

Ai nucleosidi possono essere legati da uno a tre gruppi fosfato e, in
tal caso, prendono il nome di \textbf{nucleotidi}.

Il gruppo fosfato conferisce una valenza acida alla molecola. Il gruppo
fosfato può trovarsi legato al carbonio 5' dello zucchero oppure al
carbonio 3'.

I nucleotidi che vengono utilizzati come substrato per la sintesi del
DNA e dell'RNA hanno 3 gruppi fosfato legati in serie sulla posizione 5'
dello zucchero.

Le basi possono subire modificazioni chimiche che sono essenziali per i
processi biologici che coinvolgono il DNA e gli RNAs.

Lo scheletro della catena polinucleotidica è costituito dall'alternanza
di zuccheri e di gruppi fosfato, mentre le basi azotate sporgono
lateralmente da questo scheletro. Ciascuna base è legata alla posizione
1' di uno zucchero da un legame glicosidico che interessa l'N1 delle
pirimidine o l'N9 delle purine. Tra uno zucchero e l'altro c'è un solo
gruppo fosfato e che quindi si tratta di un polimero di nucleosidi
monofosfato. Durante la sintesi degli acidi nucleici, ciascun nucleoside
trifosfato utilizzato come substrato, perde due dei suoi tre gruppi
fosfato. Ciascun gruppo fosfato forma un \textbf{legame estere} con il
C5' di uno zucchero e un secondo legame estere con il C3' dello zucchero
successivo. Questo tipo di legame, chiamato \textbf{fosfodiesterico},
conferisce una sorta di asimmetria (o polarità) alla catena, nel senso
che questa presenta due diverse estremità: da una parte c'è un
nucleotide che ha il C5' libero, mentre il C3' è impegnato nel legame
fosfodiesterico con il nucleotide adiacente; dall'altra la molecola
polimerica termina con un nucleotide che ha il C5' impegnato nel legame
fosfodiesterico con quello adiacente, mentre il C3' è libero. La loro
valenza è comunque impegnata da gruppi -OH o da gruppi fosfato.

\subsubsection{Struttura fisica del DNA}\label{struttura-fisica-del-dna}

Il DNA è formato da due catene polinucleotidiche antiparallele avvolte
l'una sull'altra a formare una struttura a doppia elica con lo scheletro
zucchero-fosfato posto sul lato esterno e le basi impilate all'interno.

Secondo il modello proposto da Watson e Crick le coppie di basi, che con
i loro anelli esterociclici sono strutture essenzialmente piatte, sono
disposte quasi perpendicolarmente rispetto all'asse della doppia elica.
Le basi basi che si affacciano in ciascuna coppia sono sempre una
pirimidina e una purina, ed è sempre presente l'appaiamento A-T/U o C-G
(questo giustifica la regolarità del diametro della doppia elica).

Le coppie A-T sono legate da 2 legami idrogeno, mentre le coppie G-C
sono legate da 3 legami idrogeno.

La doppia elica (duplex) del DNA ha una struttura regolare, destrorsa,
compie un giro completo ogni 34 A e ha un diametro di circa 20 A. La
distanza tra due coppie di basi adiacenti è di 3,4 A e ci sono, quindi,
circa 10,4 coppie di basi per ogni giro di elica. L'elica presenta
un'asimmetria dovuta alla posizione delle molecole di deossiribosio ai
lati delle basi: come conseguenza della opposta polarità 5'$\rightarrow$
3' delle due eliche, i due zuccheri di ciascuna coppia di nucleotidi si
vengono a trovare dallo stesso lato. Questa asimmetria genera nella
doppia elica due solchi di dimensioni diverse detti \textbf{solco
maggiore} e \textbf{solco minore}, con diametro 22 A e 12 A
rispettivamente.

\paragraph{Strutture alternative del
DNA}\label{strutture-alternative-del-dna}

La struttura del DNA proposta da Watson e Crick non è l'unica possibile.
Tale forma è detta \textbf{forma B} ed è stata ottenuta da studi di
diffrazione ai raggi X condotti in condizioni di \emph{alta umidità
(95\%)} e \emph{bassa salinità (condizioni cellulari)}.

Se si riduce l'umidità relativa in cui si trova la fibra di DNA, esso
assume la \textbf{forma A}. Questa forma è destrorsa come la B, ma se ne
differenzia per vari aspetti:

\begin{itemize}
\itemsep1pt\parskip0pt\parsep0pt
\item
  il passo dell'elica è di 25 A e il diametro di 23 A (è più tozza
  rispetto alla forma B);
\item
  presenta 11 coppie di basi per ogni giro dell'elica
\item
  le coppie di basi presentano una maggiore rispetto al piano
  perpendicolare all'asse della doppia elica;
\end{itemize}

Questa forma è stata riconosciuta tramite studi di diffrazione ai raggi
X in condizioni di \emph{minore umidità (75\%)} e \emph{alta salinità}.
Questa è, in vivo, la struttura tipica dei duplex DNA-RNA e RNA-RNA. Il
2'-OH del ribosio impedisce alla molecola di assumere la forma B.

L'ultima conformazione è la \textbf{forma Z}. Questa forma è
\emph{sinistrorsa} a causa del cambiamento dell'orientamento del legame
glucosidico base-zucchero tra la guanina e il deossiribosio. Nella forma
B lo zucchero e la base sono presenti nella conformazione ``anti'',
mentre nella forma Z si presentano nella conformazione ``syn''. La forma
Z ha un passo di 46 A, con 12 coppie di basi per giro d'elica, e un
diametro di 18 A. Rispetto alla forma B (passo 34Å), questo DNA ha una
forma allungata e magra.

Questa forma è stata riconosciuta tramite studi di diffrazione a raggi X
in condizioni di \emph{alta salinità} o in presenza di alcoli. Sono
state isolate proteine che legano ad alta affinità la forma Z e prove
sperimentali dicono che una piccola percentuale del DNA in vivo si trova
in questa forma.

Esistono altre forme che non analizzeremo (C,D e E).

\subsubsection{Topologia del DNA e DNA
topoisomerasi}\label{topologia-del-dna-e-dna-topoisomerasi}

La struttura di due filamenti avvolti in una doppia elica pone die seri
problemi durante i vari processi che richiedono un'apertura dell'elica e
la separazione dei due filamenti, dove il DNA si attorciglia us se
stesso a formare strutture complesse, dette \textbf{superavvolgimenti}.
Lo stato superavvolto del DNA contiene, come in una molla, l'energia che
viene utilizzata proprio per aprire i due filamenti Le molecole di DNA
possono essere circolari o lineari, in entrambi i casi possono
presentarsi dei superavvolgimenti.

Questi superavvolgimenti causano una variazione nella
\textbf{topologia}, ovvero della conformazione nello spazio, del DNA.

Il cambiamento della topologia del DNA è un aspetto fondamentale per
tutte quelle attività funzionali che richiedono una separazione dei
filamenti (replicazione, trascrizione, ricombinazione, ecc.). Le
molecole di DNA sia batteriche che eucariotiche sono \emph{superavvolte
negativamente}. La separazione dei due filamenti è maggiormente favorita
nel DNA superavvolto negativamente che nel DNA rilassato.

Le conseguenze del superavvolgimento cambiano a seconda del verso di
attorcigliamento. Se il superavvolgimento è \emph{negativo} il DNA si
avvolge in \emph{direzione opposta} rispetto a quella della doppia
elica. In questo modo diminuisce il n° di basi per giro elica.

Se il superavvolgimento é \emph{positivo} il DNA si avvolge nella
\emph{stessa direzione} della doppia elica aumenta. In questo modo il il
n° di basi per giro d'elica aumenta.

In entrambi i casi si crea tensione all'interno del DNA.

Consideriamo due filamenti circolari chiusi: il numeor di volte che un
filamento dovrebbe passsare attraverso l'altro in maniera che essi
possano essere cioketamente separati generando due circoli in maniera
che essi possano essere completamente separati generando due circoli a
singolo filamento, si chiama \textbf{numero di legame (Lk)}.

La \emph{frequenza} (quante volte un filamento si avvolge sull'altro) e
la posizione si entrambi nello spazio tridimensionale possono essere
descritte da due grandezze:

\begin{itemize}
\itemsep1pt\parskip0pt\parsep0pt
\item
  il \textbf{twist (Tw)}, rappresenta il numero di giri della doppia
  elica rispetto all'asse centrale (definisce il grado di avvolgimento
  della doppia elica). Questo è una proprietà intrinseca alla molecola
  ed consiste nel n° totale bp/ n° bp per giro d'elica (10,4 per il DNA
  in forma B);
\item
  il \textbf{writhe (Wr)}, rappresenta il numero di volte che l'asse
  centrale della doppia elica incontra se stesso.
\end{itemize}

La somma di queste due grandezze indica il numero di volte che un
filamento si avvolge sull'altro ed è il numero di legame. Lk può dunque
essere definito dalla seguente equazione:

\textbf{Lk = Tw + Wr}

Questa equazione descrive le possibili conformazoni che il DNA assume
nello spazio tridimensionale.

Per il DNA circolare chiuso o il DNA lineare con estremità fissate, Lk è
un valore costante e non può essere modificato da nessuna deformazione
che non comporti la rottura e la riunione di uno o entrambi filamenti.

Pertanto se Wr cambia a causa di un superavvolgimento positivo o
negativo, il Tw dovrà cambiare nella direzione opposta.

Se Lk = costante

\textbf{$\Delta$Tw = - $\Delta$Wr}

Molecole circolari covalentemente chiuse di uguale lunghezza (stessa
sequenza di basi) ma che differiscono solo per il numero di legame, sono
definite \textbf{topoisomeri}.

Consideriamo ora il caso di un DNA circolare completamente rilassato
(cccDNA): il suo numero di legame Lk sarà uguale al suo twist e il
writhe sarà 0. Definiamo in questo caso \textbf{Lk = Lk$_0$}. Se
vogliamo paragonare il grado di superelicità di due DNA che hanno la
stessa lunghezza, come due diversi topoisomeri, questa differenza è
definita da:

\textbf{$\Delta$Lk = Lk - Lk$_0$}

Se il $\delta$Lk di un cccDNA è diverso da 0, la molecola conterrà della
tensione e sarà superavvolta. Con Lk \textless{} Lk$_0$ e $\Delta$Lk
\textless{} 0, il DNA è superavvolto negativamente, se Lk \textgreater{}
Lk$_0$ e $\Delta$Lk \textgreater{} 0, allora il DNA è superavvolto
positivamente

Senza introduzione di tagli, essendo Lk costante, se si forza il DNA a
cambiare Wr, la molecola compenserà cambiando Tw e viceversa.

Per modificare Lk occorre rompere la doppia elica \emph{(nick)} e far
ruotare i due filamenti l'uno rispetto all'altro in modo da aumentare o
diminuire il numero di volte che un filamento si incrocia con l'altro.

(aggiungere immagine ``superavvolgimenti'')

\textbf{Confronto fig. A-B-C} Se si riduce L di 2 unità, la molecola
introduce 2 superavvolgimenti negativi (Wr = -2) per ripristinare il
valore di TW=20 (208/10,4). L \textless{} Lk 0

\textbf{Confronto fig. D-E-F} Se si aumenta L di 2 unità, la molecola
introduce 2 superavvolgimenti positivi (Wr= +2) per ripristinare il
valore di Tw=20 (208/10,4) L \textgreater{} Lk 0

\paragraph{Le topoisomerasi}\label{le-topoisomerasi}

Lo stato topologico del DNA deve essere tneuto sotto controllo e per
fare questo la doppia elica deve essere aperta e richiusa
temporaneamente. Per catalizzare queste reazioni complesse gli organismi
contengono una classe di enzimi specializzati, chiamati \textbf{DNA
topoisomerasi}, che lavorano per mantenere adeguato il livello di
superavvolgimento del DNA.

Questi sono enzimi altamente conservati che convertono (isomerizzano) un
topoisomero in un altro modificandone il numero di legame Lk.

Questi enzimi catalizzano la rottura e la ri-unione dei filamenti di DNA
introducendo nick temporanei: creano un taglio transiente sul DNA
mediante una tirosina che si lega covalentemente allo scheletro fosfato,
con una reazione di transesterificazione fatta dal suo gruppo -OH. Dopo
la manipolazione topologica che aumenta o diminuisce Lk, il gruppo -OH
libero dle filamente tagliato, attraverso una reazione inversa, attacca
il legame tra la tirosina e il DNA, ripristinando la continuità della
doppia elica.

L'intero processo di modificazione topologica utilizza l'energia libera
contenuta nel DNA superavvolto senza richiesta di ATP aggiuntivo.

Le estremità del DNA generate dalla rottura del filamento non sono mai
libere, ma sono manipolate dentro i confini dell'enzima.

Alcune topoisomerasi possono rimuovere (e perciò rilassare)
superavvolgimenti negativi, altre sia positivi che negativi.

Le DNA topoisomerasi si differenziano tra loro per il meccanismo di
azione con il quale cambiano la topologia del DNA. Esistono due
principali strategie:

\begin{enumerate}
\def\labelenumi{\arabic{enumi}.}
\itemsep1pt\parskip0pt\parsep0pt
\item
  \textbf{rotazione controllata}, l'enzima introduce una singola rottura
  su un filamento della doppia elica e permette la rotazione su sé
  stesso del filamento intatto per un numero variabile di giri, fino a
  che l'attrito tra il DNA e l'enzima induce la rilegazione del
  filamento tagliato;
\item
  \textbf{strand passage}, consiste nel creare una rottura singola o a
  doppio filamento sul DNA, allargare l'interruzione prodotta e far
  passare l'altro filamento o la doppia elica attraverso la rottura che
  successivamente verrà risaldata.
\end{enumerate}

Le topoisomerasi vengono classificate sulla base della sequenza a.a.
osulla base del meccanismo di reazione.

In base al meccanismo di reazione si gli enzimi si suddividono in:

\begin{itemize}
\itemsep1pt\parskip0pt\parsep0pt
\item
  \textbf{topoisomerasi di tipo I}, sono in genere dei monomeri che
  tagliano un solo filamento del DNA al 5', creando in questo modo
  un'apertura mediata dalle interazioni di diversi domini dell'enzima
  con la doppia elica, attraverso la quale può passare l'altro filamento
  o una doppia elica. Questo meccanismo permette di eliminare con grande
  efficienza strutture annodate sul DNA e di rilassare esclusivamente
  superavvolgimenti negativi. Le topoisomerasi di tipo I, a seconda che
  formino un legame tirosina-5'fosfato o tirosina-3'fosfato, sono divise
  in \emph{tipo A} e \emph{tipo B} rispettivamente. Le tipo A richiedono
  anche ioni Mg$^2$$^+$ o Zn$^2$$^+$.
\end{itemize}

(immagine p41)

\begin{itemize}
\itemsep1pt\parskip0pt\parsep0pt
\item
  \textbf{topoisomerasi di tipo II}, sono in genere dei dimeri o
  multimetri che introducono un tagio su entrambi i filamenti del DNA
  con le due tirosine covalentemente legate all'estremità 5' e portano
  avanti le modificazioni topologiche facendo passare un secondo tratto
  a doppia elica attraverso la rottura. Il filamento tagliato si chiama
  segmento G (gate), il segmento che passa attraverso l'apertura si
  chiama segmento T (transport). Il meccanismo di azione della top II
  viene definito a \textbf{doppio cancello}. In questo processo il
  segmento T del DNA entra nella parte superiore dell'enzima, che si
  apre per accoglierlo; dopo l'apertura della doppia elica del segmento
  G, il segmento T attraversa tutto l'enzima, che si apre nella parte
  inferiore, rimuovendo in questa maniera due superavvolgimenti alla
  volta. Questo meccanismo richiede l'apporto di energia sotto forma di
  ATP per promuovere le modificazioni del complesso enzima-DNA (ma non
  per tagliare e risaldare i due filamenti). Queste topoisomerasi
  rilassano sia superavvolgimenti positivi che negativi.
\end{itemize}

(immagine top II)

In generale, le topoisomerasi sono coinvolte nei processi metabolici
associati alla separazione dei filamenti di DNA come la replicazione, la
trascrizione, la ricombinazione ed il riparo dell'acido nucleico. I due
filamenti del DNA, durante questi processi, devono essere separati
temporaneamente per diventare accessibili alle polimerasi o ai
componenti del complesso della trascrizione, di conseguenza il DNA
subisce degli stress torsionali che portano a dei superavvolgimenti
della molecola.

\textbf{Nomenclatura:} tutte le topoisomerasi di tipo I hanno un numero
dispari (I, III, V), mentre quelle di tipo II hanno un numero pari (II,
IV, VI). Le topoisomerasi con attività di superavvolgimento vengono
invece distinte in girasi (topoisomerasi di tipo II che introducono
superavvolgimenti negativi) e girasi inversa (topoisomerasi di tipo I
che introducono superavvolgimenti positivi).

La girasi di \emph{E. coli} è una topoisomerasi di tipo II ed è l'unica
in grado di introdurre superavvolgimenti negativi.

\subsection{La replicazione del DNA}\label{la-replicazione-del-dna}

Per trasmettere alle cellule figlie lo stesso patrimonio genetico della
cellula madre, l'informazione genetica contenuta nel DNA deve essere
duplicata con la massima fedeltà possibile, in quanto una variazione
nella sequenza nucleotidica genererebbe cambiamenti nel messaggio
genetico e dunque mutazioni.

Sulla base della struttura del DNA furono ipotizzati 3 possibili modelli
replicativi:

\begin{enumerate}
\def\labelenumi{\arabic{enumi}.}
\itemsep1pt\parskip0pt\parsep0pt
\item
  semiconservativo;
\item
  conservativo;
\item
  dispersivo.
\end{enumerate}

Il meccanismo di replicazione fu determinato nel 1958 da Meselson e
Stahl utilizzando la tecnica di \textbf{centrifugazione su gradiente di
densità}.

Inizialmente fecero crescere cellule di \emph{E. coli} in un terreno di
coltura a cui vennero aggiunti dei sali d'ammonio che contenevano
l'isotopo pesante dell'azoto $^1$$^5$N, al posto dell'isotopo normale
$^1$$^4$N. Le molecole di DNA che contengono $^1$$^5$N hanno una
``densità'' maggiore rispetto a quella di molecole di DNA contenenti
$^1$$^4$N e, dopo centrifugazione all'equilibrio in soluzioni
concentrate di cloruro di cesio, si troveranno verso il fondo della
provetta da centrifugazione.

Le cellule di \emph{E. coli} vennero fatte duplicare per molte
generazioni in un terreno contenente sali pesanti dell'azoto, affinchè
tutte le cellule avessero molecole di DNA contenenti $^1$$^5$N. A quel
punto, le cellule furono raccolte tramite centrifugazione, lavate per
togliere ogni traccia di terreno contenente $^1$$^5$N, diluite in un
terreno contenente sali d'ammonio con l'isotopo leggero $^1$$^4$N e
fatte crescere in queste condizioni per due generazioni.

Se fosse stata vera l'ipotesi semiconservatica, dopo una generazione,
tutte le cellule di \emph{E. coli} che avevano replicato il loro DNA
avrebbero dovuto avere molecole di DNA contenente un filamento parentale
marcato con $^1$$^5$N e un filamento neosintetizzato marcato con
$^1$$^4$N. Tutte le molecole avrebbero dovuto avere, quindi, una densità
intermedia rispetto a quella di un DNA a doppia elica tutto pesante o
tutto leggero.

Se invece fosse risultata vera l'ipotesi conservativa della replicazione
del DNA i ricercatori si aspettavano che alla prima generazione metà del
DNA sarebbe stato pesante e metà leggero.

(immagine 05)

\subsubsection{Il modello del replicone}\label{il-modello-del-replicone}

Un \textbf{replicone} è un'unità di DNA coinvolta nella replicazione.

Questo modello postula che l'inizio della replicazione del DNA è
geneticamente controllato da sequenze specifiche in \emph{cis} sul DNA
(chiamate \textbf{replicatori}). Queste sequenze determinano dove può
partire la replicazione del DNA interagendo con specifiche proteine
(\textbf{iniziatori}) che agiscono in \emph{trans} e collegano il
processo replicativo con la crescita e la divisione cellulare.

La replicazione del DNA necessita che i due filamenti si separino ed
inizia in punti specifici della molecola detti \textbf{origine di
replicazione}. La replicazione può essere \textbf{unidirezionale} o
\textbf{bidirezionale} a seconda del numero di forche replicative.

\emph{Unidirezionale} = una sola forca replicativa si allontana
dall'origine (rara, es. plasmide ColE1 di E. coli, fago $\Phi$ 29 e gli
adenovirus). \emph{Bidirezionale} = due forche replicative procedono
contemporaneamente allontanandosi dall'origine in direzioni opposte.

(immagine 06)

Quando il DNA in replicazione viene osservato al microscopio elettronico
la regione replicata appare come una \textbf{bolla di replicazione}
all'interno del DNA non replicato. La bolla si estende fino a quando non
conterrà l'intero replicone

In tutti i cromosomi degli eucarioti esistono numerose origini di
replicazione. L'osservazione di ``bolle'' di replicazione di dimensioni
diverse indica due caratteristiche importanti delle origini dei
cromosomi degli eucarioti:

\begin{enumerate}
\def\labelenumi{\arabic{enumi}.}
\itemsep1pt\parskip0pt\parsep0pt
\item
  l'accessione delle origini non è simultanea, ma esistono delle origini
  ``early'' che vengono accese presto durante la fase di replicazione
  del DNA, mentre altre origini ``late'' vengono attivate a tempi
  successivi;
\item
  Le forcelle di replicazione che avanzano in direzioni opposte sul
  cromosoma possono fondersi tra di loro.
\end{enumerate}

(immagine 07)

I repliconi batterici, come quello di \emph{E. coli}, sono normalmente
circolari e la replicazione procede in modo bidirezionale a partire da
una singola origine chiamata \emph{OriC}.

OriC di E. coli (genoma 4,2 Mbp) è una regione di 248 bp che contiene al
suo interno regioni che vengono riconosciute da proteine nelle fasi
iniziali della replicazione:

\begin{itemize}
\itemsep1pt\parskip0pt\parsep0pt
\item
  3 regioni di 13 bp ricche in AT;
\item
  4 regioni di 9 bp (siti di legame per l'iniziatore DnaA).
\end{itemize}

La particolarità di essere ricche in A e T è risultata, come vedremo,
una caratteristica comune alle origini di replicazione identificate in
altri organismi ed è legata al fatto che un'origine di replicazione, per
funzionare, deve potersi aprire ed è necessaria meno energia per
denaturare localmente una regione di DNA ricca in A e T piuttosto che in
C e G.

\subsubsection{Il processo replicativo}\label{il-processo-replicativo}

\paragraph{La DNA polimerasi}\label{la-dna-polimerasi}

Il modello della replicazione semiconservativa del DNA suggerisce
immediatamente l'esistenza di enzimi in grado di catalizzare la
polimerizzazione dei nucleotidi. Se durante la replicazione ciascun
filamento serve da stampo per la neosintesi di un filamento
complementare, c'è da aspettarsi che nella cellula esistano delle
attività in grado di catalizzare la formazione del legame
fosfodiesterico tra nucleotidi in modo DNA stampo-dipendente. Tale
ipotesi portò all'identificazione della prima \textbf{DNA polimerasi}.

La sintesi del DNA ha delle precise richieste biochimiche: i substrati
della reazione sono i \textbf{deossiribonucleosidi trifosfati (dNTP)}, e
un \textbf{complesso innesco-stampo} costituito da uno stampo
rappresentato da un filamento di DNA e da un innesco che può essere un
tratto più o meno lungo di DNA che porti una estremità 3'OH che la DNA
polimerasi è in grado di allungare. Tutte le DNA polimerasi non sono in
grado di iniziare la sintesi della catena nucleotidica, ma necessitano
di una estremità 3'OH da allungare.

La DNA polimerasi aggiunge un nuovo nucleotide catalizzando la
formazione del legame fosfodiesterico nel rispetto della regola della
complementarietà tra le basi con il filamento di DNA stampo.

nella formazione di ciascun legame fosfodiesterico, il fosfato in
posizione $\alpha$ del dNTP viene legato al 3'OH dell'innesco portando
alla liberazione di pirofosfato. L'energia libera di questa reazione è
piuttosto modesta, e l'energia addizionale che spinge la reazione verso
la polimerizzazione è fornita dall'idrolisi del pirofosfato da parte di
un enzima chiamato \textbf{pirofosfatasi}.

Le DNA polimerasi possiedono un'attività sintetica con polarità 5'
$\rightarrow$ 3'.

(immagine 08)

La struttura delle DNA polimerasi replicative è paragonata a quella di
una mano parzialmente chiusa dove si distinguono 3 domini: pollice, dita
e palmo. Il DNA si lega ad una grande fessura compresa tra i 3 domini.
Il sito catalitico si trova nel palmo, mentre le dita sono coinvolte nel
posizionamento corretto dello stampo nel sito attivo. Il pollice lega il
DNA mentre esce dall'enzima ed è importante per la processività
dell'enzima. L'attività esonucleasica risiede in un dominio indipendente
con un proprio sito attivo.

\paragraph{Fedeltà del processo
replicativo}\label{fedeltuxe0-del-processo-replicativo}

La fedeltà del processo replicativo è di circa 1 errore ogni
10$^9$-10$^1$$^0$ nucleotidi polimerizzati.

Un sistema basato solo sulla stereochimica di coppie di basi che
rispettino la regola della complementarietà non sarebbe in grado di
raggiungere l'accuratezza indicata sopra. La selettività delle
polimerasi è piuttosto limitata (1 errore ogni 10$^5$ circa) per la
presenza di forme tautomeriche delle basi azotate. Questi errori sono
rimosis da un'attività esonucleasica 3'$\rightarrow$ 5' che ha, quindi,
una polarità invera alla direzione di sintesi del DNA. Tale attività ha
la funzione di agire come correttore di bozze \textbf{(attività
proofreading)}. Quando la DNA polimerasi rileva la presenza di un
appaiamento non corretto tra le coppie di basi, il complesso
stampo-innesco si allontana dal sito catalitico di polimerizzazione
della DNA polimerasi e si avvicina al sito esonucleasico con funzione
proofreading. Tale attività elimina il nucleotide errato e permetta alla
polimerasi di riprendere la sintesi senza che il complesso ternario
innesco-stampo-enzima si sia dissociato.

La fedeltà della replicazione del DNA può essere dunque riassunta in:

\begin{enumerate}
\def\labelenumi{\arabic{enumi}.}
\itemsep1pt\parskip0pt\parsep0pt
\item
  fedeltà;
\item
  attività di proofreadin;
\item
  riparazione degli errori.
\end{enumerate}

\paragraph{Forcella di replicazione: sintesi del filamento continuo e
del filamento
discontinuo}\label{forcella-di-replicazione-sintesi-del-filamento-continuo-e-del-filamento-discontinuo}

Poichè tutte le DNA polimerasi possono polimerizzare il DNA soltanto in
direzione 5' $\rightarrow$ 3', ciò crea un problema nella progressione
della forcella replicativa. Soltanto uno dei due filamenti di neosintesi
può essere sintetizzato in modo continuo seguendo la direzione della
forcella replicativa: questo filamento di neosintesi viene chiamato
\textbf{filamento continuo} o \textbf{leading strand}. Per rispettare la
polarità di sintesi del DNA (5' $\rightarrow$ 3') l'altro filamento deve
essere sintetizzato in modo discontinuo e genera quello che viene
definito \textbf{filamento ritardato} o \textbf{lagging strand}. Tale
filamento è definito ritardato perchè la sua la sua sintesi può iniziare
solo dopo che il progredire della forcella di replicazione ha generato
sufficiente DNA a singolo filamento da far partire la sintesi del
filamento ``lagging''. La sintesi del filamento ritardato si realizza
attraverso la generazione di frammenti discontinui di DNA che vengono
chiamati \textbf{frammenti di Okazaki}.

La sintetizzazione discontinua del frammento lagging implica la
necessità di dell'enzima \textbf{DNA ligasi}, un enzima in grado di
saldare il 3'OH di un frammento di Okazaki con il 5'-fosfato del
frammento di Okazaki sintetizzato precedentemente.

\paragraph{Innesco della sintesi di
DNA}\label{innesco-della-sintesi-di-dna}

L'innesco della sintesi di ogni frammento di Okazaki, così come
l'innesco del filamento a un'origine di replicazione, devono prevedere
la sintesi di un iniziatore (\emph{primer}) per offrire alla DNA
polimerasi il complesso stampo-iniziatore descritto poco sopra.
L'innesco della sintesi del DNA è fornito da corte (4-12 nucleotidi)
molecole di RNA sintetizzate da un enzima, denominato \textbf{DNA
primasi}.

\paragraph{Le proteine replicative di E.
coli}\label{le-proteine-replicative-di-e.-coli}

(immagine 09)

Il primo evento molecolare consiste nel riconoscimento di \emph{ori C}
da parte della proteine \textbf{DnaA}. DnaA si attacca inizialmente alle
ripetizioni di 9 pb: la proteina si lega in modo cooperativo formando
una specie di nucleo centrale proteico intorno al quale si avvolge il
DNA di \emph{oriC}. Successivamente, DnaA si lega anche alle 3
ripetizioni di 13 pb facilitando la denaturazione localizzata del DNA in
quella regione e permettendo l'assemblaggio di altre proteine
replicative.

Il passaggio successivo consiste nel caricamento di \textbf{DnaB} e
\textbf{DnaC} a livello della bolla di denaturazione creatasi nella
regione di \emph{oriC} dando origine a un complesso proteico, denominato
\textbf{complesso di pre-innesco} della sintesi del DNA. DnaB possiede
un'attività elicasica, per cui consumando ATP è in grado di separare i
due filamenti di DNA. Lo svolgimento del DNA sia nella fase iniziale che
nel successivo processo di allungamento della sintesi del DNA genera una
tensione torsionale del DNA che è risolta da enzimi in grado di
modificare la topologia del DNA e chiamati ``DNA topoisomerasi''.

La bolla di denaturazione creatasi tende spontaneamente a rinaturare:
per evitare questo il DNA a singolo filamento originatosi viene
stabilizzato dal legame della \textbf{proteina SSB} alla regione di
ssDNA. SSB è una proteina che ha un'affinità maggiore per il DNA a
singolo filamento che a doppio filamento.

Le proteine DnaA e SSB interagiscono con il DNA in modo cooperativo:
questo sta a indicare che il legame tra una molecola di SSB all'ssDNA
facilita il legame di una seconda molecola di SSB alla stessa molecola
di DNA.

(immagine 10)

Tutte le DNA polimerasi necessitano di un'estremità 3'OH pre-formata per
poter aggiungere i nucleotidi successivi. In E. coli e negli eucarioti,
però, gli inneschi che forniscono il 3'OH che può essere allungato dalla
DNA polimerasi sono corte molecole di RNA; tali molecole funzionano da
primer sia per far partire la sintesi continua a un'origine del
filamento leading, sia per iniziare la sintesi di tutti i grammenti di
Okazaki. La \textbf{DNA primasi} di E. coli, codificata dal \textbf{gene
dnaG}, è costituita da un singolo polipeptide e la sintesi degli
\textbf{RNA primer} costituisce il primo effettivo evento di sintesi
nella replicazione del DNA.

La DNA polimerasi replicativa di E. coli è la \textbf{DNA polimerasi III
oloenzima}, una macchina proteica molto complessa formata da numerose
subunità che si assemblano in modo sequenziale a formare un dimero
catalitico. Ci sono due copie del nucleo catalitico, che è formato da 3
subunità: \textbf{$\alpha$} (attività DNA polimerizzante),
\textbf{$\varepsilon$} (attività esonucleasica correttore di bozze
3'$\rightarrow$ 5') e \textbf{$\theta$} (stimola la esonucleasi). Ci
sono anche due copie delle subunità \textbf{$\tau$}, che media la
dimerizzazione dei 2 nuclei catalitici. Questi possono assemblarsi sul
DNA solo dopo che un complesso di 5 proteine, chiamato
\textbf{``complesso $\gamma$''}, è riuscito a caricare sul DNA due copie
della subunità $\beta$; per tale processo il complesso $\gamma$, che è
una \emph{ATPasi DNA-dipendente}, richiede e consuma ATP. La subunità
\textbf{$\beta$} forma un omodimero con una forma a ciambella in grado
di abbracciare il singolo filamento di DNA che passa nel foro della
ciambella. I due nuclei catalitici presenti nel modello dimerico di
replicazione del DNA replicano contemporaneamente il filamento leading e
il filamento lagging della doppia elica.

L'aspeytto più importante del modello è che la sintesi del filamento
leading induce la formazione, sull'altro filamento, di un'ansa che
fornisce lo stampo per la sintesi del filamento ritardato.

(Figura 6.19, p 161)

\end{document}
