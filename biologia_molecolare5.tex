\documentclass[]{article}
\usepackage{lmodern}
\usepackage{amssymb,amsmath}
\usepackage{ifxetex,ifluatex}
\usepackage{fixltx2e} % provides \textsubscript
\ifnum 0\ifxetex 1\fi\ifluatex 1\fi=0 % if pdftex
  \usepackage[T1]{fontenc}
  \usepackage[utf8]{inputenc}
\else % if luatex or xelatex
  \ifxetex
    \usepackage{mathspec}
    \usepackage{xltxtra,xunicode}
  \else
    \usepackage{fontspec}
  \fi
  \defaultfontfeatures{Mapping=tex-text,Scale=MatchLowercase}
  \newcommand{\euro}{€}
\fi
% use upquote if available, for straight quotes in verbatim environments
\IfFileExists{upquote.sty}{\usepackage{upquote}}{}
% use microtype if available
\IfFileExists{microtype.sty}{%
\usepackage{microtype}
\UseMicrotypeSet[protrusion]{basicmath} % disable protrusion for tt fonts
}{}
\ifxetex
  \usepackage[setpagesize=false, % page size defined by xetex
              unicode=false, % unicode breaks when used with xetex
              xetex]{hyperref}
\else
  \usepackage[unicode=true]{hyperref}
\fi
\hypersetup{breaklinks=true,
            bookmarks=true,
            pdfauthor={},
            pdftitle={},
            colorlinks=true,
            citecolor=blue,
            urlcolor=blue,
            linkcolor=magenta,
            pdfborder={0 0 0}}
\urlstyle{same}  % don't use monospace font for urls
\setlength{\parindent}{0pt}
\setlength{\parskip}{6pt plus 2pt minus 1pt}
\setlength{\emergencystretch}{3em}  % prevent overfull lines
\setcounter{secnumdepth}{0}

\date{}

\begin{document}

\section{La trasposizione}\label{la-trasposizione}

La trasposizione è una forma specifica di ricombinazione genica che
sposta alcuni elementi genici da un sito all'altro. Questi elementi
genici mobili sono detti elementi trasponibili o trasposoni.

Lo spostamento avviene mediante un evento di ricombinazione tra sequenze
di DNA poste alle estremità dell'elemento trasponibile ed una sequenza
presente sul DNA della cellula ospite. La trasposizione può avvenire con
o senza la duplicazione dell'elemento.

Talvolta la reazione di ricombinazione passa attraverso un intermedio
transitorio a RNA.

Quando gli elementi trasponibili si muovono, mostrano una scarsa
selettività nella scelta della sequenza del sito di inserzione.

Il risultato è che i trasposoni si possono inserire all'interno di geni,
spesso distruggendone completamente la funzione. Si possono anche
inserire in regioni regolatrici di un gene, in questo caso la loro
presenza può causare un cambio della modalità di espressione del gene
stesso. Parliamo appunto di scarsa selettività del sito di inserzione.
Possiamo poi distinguere parlando dei trasposoni in ``elementi
autonomi'' e ``elementi non autonomi''. I primi codificano proteine per
la loro mobilità (vale a dire che possiedono tutto ciò che serve per
spostarsi), mentre i secondi sono in grado di muoversi solo se nella
cellula è presente un elemento autonomo. Questi infatti contengono solo
le sequenze cis (estremità ripetute e invertite) necessarie per la
trasposizione e non i geni codificanti le proteine per la loro motilità.
In assenza dell'elemento autonomo rimangono bloccati.

IMMAGINE

Nell'uomo sono importante causa di malattie e di mutazioni genetiche
(emofilia B e coriodermia sono alcuni esempi). Per lo stesso motivo sono
usati in biologia sperimentale come agenti mutageni e vettori per il
trasporto di DNA.

Barbara McClintock negli anni cinquanta, con esperimenti sulle
pannocchie di granoturco scopre l'esistenza dei trasposoni. Per questa
scoperta vinse il premio Nobel per la medicina nel 1983.

La ricercatrice inizialmente noto la capacità degli elementi
trasponibili di rompere i cromosomi e trovò che alcuni ceppi subivano
con particolare frequenza tali rotture cromosomiche. Chiamò gli elementi
genetici responsabili di queste rotture Ds (dissociator). Usa un ceppo
di mais con chicchi non pigmentati per la presenza dell'elemento Ds.
Mediante l'inserimento di Ac (activator) in questo ceppo, si ottiene la
destabilizzazione del fenotipo. Compaiono infatti zone pigmentate.

Ciò dimostra che nelle cellule non pigmentate Ds è un elemento mobile
che sopprime l'espressione di un gene della pigmentazione che era
adiacente a Ds stesso. Con l'inserimento di Ac il gene invece può essere
trasferito e la pigmentazione non è più repressa.

Si scoprì poi che il gene Ds è un trasposone NON AUTONOMO mentre Ac è
AUTONOMO.

Tutti gli esseri viventi contengono elementi trasponibili nel loro
genoma. Ma tendenzialmente all'aumentare della complessità di un
organismo, decresce la densità genica ed aumentano le sequenze ripetute.
Per esempio in Saccharomyces Cerevisiae il genoma occupato da elemneti
trasponibili è il 3\%, mentre in Arabidopsis thaliana è il 14\%, in
Drosophila melanogaster il 15\%, in Homo sapiens l 44\% e in Zea mais il
60\%.

I trasposoni si dividono anche in trasposoni semplici e complessi.

I trasposoni semplici contengono solo sequenze necessarie per la loro
trasposizione e i geni codificanti le proteine che regolano il processo,
mentre quelli complessi contengono uno o più geni che possono conferire
vantaggi alla cellula ospite (come per esempio la resistenza agli
antibiotici).

Ci sono tre classi principali di elementi trasponibili che esistono sia
in forma autonoma che non autonoma: 1. \textbf{trasposoni a DNA}; 2.
\textbf{retrotrasposoni simili a virus (retrotrasposoni LTR)} -- i
retrovirus fanno parte di questa categoria; 3. \textbf{retrotrasposoni
poli-A} (sono anche chiamati retrotrasposoni non virali).

I trasposoni a DNA rimangono a DNA per tutto il ciclo ricombinativo,
mentre gli altri formano un intermedio a RNA. Questi elementi si muovono
usando dei meccanismi di taglio e riunione di filamenti di DNA in
maniera analoga agli elementi che si muovono per ricombinazione
conservativa sito-specifica.

IMMAGINE

\subsection{Trasposoni a DNA}\label{trasposoni-a-dna}

Riportano sia delle sequenze di DNA che servono come siti per la
ricombinazione, sia geni che codificano proteine che partecipano alla
ricombinazione, sia geni che codificano proteine che partecipano alla
ricombinazione. Sono formati da:

\begin{itemize}
\itemsep1pt\parskip0pt\parsep0pt
\item
  \textbf{siti per la ricombinazione} si trovano all'estremità
  dell'elemento e sono organizzati come sequenze ripetute ed invertite
  che variano in lunghezza da circa 25 ad alcune centinaia di coppie di
  basi. Questi tratti portano le sequenze di riconoscimento della
  \textbf{trasposasi};
\item
  \textbf{gene trasposasi} per la motilità (nei trasposoni autonomi);
\item
  \textbf{geni aggiuntivi} che codificano proteine che regolano la
  trasposizione o che conferiscono alla cellula funzioni utili. Molti
  trasposoni batterici a DNA portano dei geni che codificano delle
  proteine che inducono resistenza a uno o più antibiotici;
\item
  a monte e a valle dei trasposoni vi è una \textbf{sequenza duplicata}
  che non fa parte del trasposone.
\end{itemize}

IMMAGINE

\subsection{Retrotrasposoni simili a
virus}\label{retrotrasposoni-simili-a-virus}

Contengono:

\begin{itemize}
\itemsep1pt\parskip0pt\parsep0pt
\item
  \textbf{sequenze ripetute e invertite} come siti per il legame e
  l'azione della ricombinasi inserite in sequenze ripetute più lunghe
  dette LTR. Queste sequenze si ritrovano all'estremità dell'elemento
  con un'organizzazione del tipo ``ripetizione diretta'' e per questo
  vengono dette lunghe sequenze terminali ripetute;
\item
  \textbf{geni codificanti per l'integrasi};
\item
  \textbf{geni codificanti per la trascrittasi inversa} DNA polimerasi
  capace di usare uno stampo ad RNA per sintetizzare DNA. Fondamentali
  poiché per l'integrazione di questi trasposoni è indispensabile un
  intermedio ad RNA;
\item
  a monte e a valle dei trasposoni vi è una \textbf{sequenza duplicata}
  che non fa parte del trasposone.
\end{itemize}

La differenza tra i trasposoni simili a virus ed i retrovirus è che il
restrovirus fugge dalla cellula e può infettarne un'altra. Il trasposone
può solo spostarsi in nuove posizioni all'interno della stessa cellula
senza mai lasciarla.

IMMAGINE

\subsection{Retrotrasposoni poli-A}\label{retrotrasposoni-poli-a}

Questi:

\begin{itemize}
\itemsep1pt\parskip0pt\parsep0pt
\item
  \textbf{non} hanno \textbf{sequenze ripetute ed invertite};
\item
  le due estremità dell'elemento sono caratterizzate da sequenze
  diverse: un'estremità è detta \textbf{5'UTR} mentre l'altra ha una
  zona chiamata \textbf{3'UTR} seguita da una serie di coppie di basi
  del tipo A-T ossia la \textbf{sequenza poli-A}. Questi elementi sono
  anche fiancheggiati da corte duplicazioni del sito bersaglio;
\item
  due geni \textbf{ORF1} e \textbf{ORF2}. ORF1 codifica una proteina che
  lega L'RNA, ORF2 un fattore caratterizzato da una attività sia di
  trascrittasi inversa sia endonucleasica. Questa proteina, anche se
  diversa dalle trasposasi e dalle integrasi codificate dagli altri tipi
  di elementi trasponibili ha un ruolo fondamentale nel processo di
  ricombinazione;
\item
  a monte e a valle dei trasposoni vi è una \textbf{sequenza duplicata}
  che non fa parte del trasposone.
\end{itemize}

IMMAGINE

\subsection{Trasposasi ed integrasi}\label{trasposasi-ed-integrasi}

Le DNA trasposasi e le integrasi retrovirali fanno parte della stessa
superfamiglia proteica. La conservazione del meccanismo di
ricombinazione si riflette nella struttura delle proteine trasposasi e
integrasi.

Esse hanno un dominio catalitico caratterizzato da una comune
conformazione tridimensionale.

Il dominio catalitico contiene tre amminoacidi conservati
nell'evoluzione: due aspartati (D) e un glutammato (E). per questo
motivo si parla delle ricombinasi di questa classe come delle
trasposasi/integrasi del motivo DDE.

Questi amminoacidi fanno parte del sito catalitico e coordinano gli ioni
metallici bivalenti (come Mg\(^2\)\(^+\) e Mn\(^2\)\(^+\)) necessari
alla loro attività.

Una caratteristica inusuale di queste proteine è quella di usare lo
stesso sito attivo per catalizzare sia il taglio del DNA che il
trasferimento del filamento, invece di usare due siti attivi ciascuno
specializzato per una reazione chimica.

A differenza della struttura altamente conservata dei domini catalitici,
le restanti regioni delle proteine appartenenti a questa famiglia non
sono conservate.

Queste regioni codificano dei domini di legame al DNA sito-specifici e
delle regioni coinvolte in interazioni proteina-proteina necessarie per
formare il complesso nucleoproteico specifico per ciascun singolo
elemento.

Questi domini unici fanno sì che le trasposasi e le integrasi
catalizzino la ricombinazione esattamente sull'elemento che le codifica
o su un elemento strutturalmente molto simile.

Il sequenziamento dei genomi ha permesso di identificare:

\begin{itemize}
\itemsep1pt\parskip0pt\parsep0pt
\item
  12 superfamiglie di trasposoni a DNA di cui 7 strettamente correlate
  ai trasposoni trovati nei batteri. Questo ci fa comprendere come
  questi elementi siano comparsi prima della separazione tra eucarioti e
  batteri. In ogni caso la composizione e la percentuale relativa dei
  diversi elementi trasponibili presenti nel genoma variano enormemente
  da specie a specie;
\item
  nell'uomo gli elementi trasponibili costituiscono il 44\% del genoma
  (che è di circa 3000 Mpb). Tra il totale degli elementi trasponibili
  abbiamo un 75,1 \% di retrotrasposoni poli-A, un 18,6\% di
  retrotrasposoni LTR e un 6,3\% di trasposoni a DNA;
\item
  in Saccharomyces cerevisiae vi sono solo retrotrasposoni LTR per un
  totale del 3\% del genoma;
\item
  Ecc ecc
\end{itemize}

\subsection{Retrotrasposoni poli-A}\label{retrotrasposoni-poli-a-1}

Tra i retrotrasposoni poli-A possiamo distinguere due classi differenti:
i \textbf{LINE} e i \textbf{SINE}.

I \textbf{LINE} sono \textbf{lunghi elementi nucleari interspersi}.

La più comune nel genoma dell'uomo è L1 (lunga 6000bp circa) presente in
circa 500'000 copie e rappresenta il 17\% circa del genoma umano. È un
elemento autonomo.

In particolare gli elementi LINEs o trasposoni non-LTR hanno una
lunghezza di circa 6-7 kb.

Contengono un promotore per l'RNA polimerasi II, una o due ORF e un
segnale di polladenilazione all'estremità 3'.

\begin{itemize}
\itemsep1pt\parskip0pt\parsep0pt
\item
  \textbf{ORF1} codifica per una proteina a funzione ignota (lega
  l'RNA?)
\item
  \textbf{ORF2} codifica per un enzima che possiede sia un'attività di
  trascrittasi inversa (RT) simile a quella dei retrovirus e dei
  retrotrasposoni virali, che un'attività di DNA endonucleasi (EN)
\end{itemize}

Vi sono tre famiglie principali di elementi LINEs:

\begin{itemize}
\itemsep1pt\parskip0pt\parsep0pt
\item
  \textbf{L1}, in cui sono incluse 60-100 copie tuttora attive e
  moltissime copie inattive troncate all'estremità 5';
\item
  \textbf{L2-L3}, inattive.
\end{itemize}

Le copie attive inserendosi in punti critici del genoma possono
inattivare dei geni con conseguente insorgenza di patologie.

Le LINEs si inseriscono preferibilmente nelle regioni eucromatiche
ricche in A+T.

IMMAGINE

I \textbf{SINE} sono \textbf{corti elementi nucleari interspersi}.

Hanno lunghezza minore di 500bp. Il trasposone Alu (lungo circa 300bp) è
presente il 106 copie e rappresenta circa il 10\% del genoma umano. È un
elemento non autonomo.

Gli elementi SINEs sono elementi non automi, hanno una lunghezza
compresa tra 0,1 e 0,4 kb.

Hanno un promotore all'interno per l'RNA polimerasi III (derivano da
trascritti dell'RNA polimerasi III infatti) e una regione ricca in A
all'estremità 3' ma non contengono un segnale di poliadenilazione.

Gli elementi SINEs non contengono alcuna ORF codificante per una
trascrittasi inversa sintetizzata da altri retroelementi (trasposizione
LINEs-dipendente).

Gli elementi SINEs sono distribuiti ad alta densità nelle regioni richhe
in CG del genoma (isocore H) perché hannp un più elevato contenuto G+C
(circa 57\%) rispetto gli elementi LINE's (40\%).

Nel genoma dei primati sono presenti tre differenti famiglie di elementi
SINEs: l'\textbf{elemento Alu}, ancora attivo; e gli elementi inattivi
\textbf{MIR} e \textbf{Ther2/MIR3}.

L'elemento Alu, il più comune nei primati, è lungo 0,3kb, è presente in
circa 1.200.000 copie nel genoma umano e rappresenta quindi oltre il
10\% di tutto il genoma. Presenta una regione ricca in A/T all'estremità
3', coinvolta nel meccanismo di retrotrasposizione.

Le sequenze Alu sono localizzate a monte o a valle dei geni, negli
introni, nelle regioni 5' e 3' non tradotte dell'mRNA. Non è noto il
loro ruolo funzionale, nonostante siano molto diffuse nel genoma di
tutti i primati.

IMMAGINE

\subsection{Meccanismi di
trasposizione}\label{meccanismi-di-trasposizione}

Esistono vari meccanismi di trasposizione:

\begin{enumerate}
\def\labelenumi{\arabic{enumi}.}
\item
  \textbf{Taglia-e-cuci}. I trasposoni a DNA, i retrotrasposoni simili a
  virus e i retrovirus usano un meccanismo simile di ricombinazione per
  inserire il proprio DNA nel nuovo sito. Questo modo di ricombinare
  prevede l'escissione del trasposone dalla sua posizione iniziale nel
  DNA ospite, seguita dall'integrazione del trasposone escisso in un
  nuovo sito di DNA. Per questo motivo si parla di trasposizione taglia
  e incolla.

  \begin{itemize}
  \itemsep1pt\parskip0pt\parsep0pt
  \item
    La trasposasi si lega alle estremità ripetute ed invertite del
    trasposone.
  \item
    L'enzima una volta che ha riconosciuto queste sequenze avvicina le
    due estremità del trasposone dando luogo ad un complesso
    nucleoproteico stabile detto \textbf{complesso sinaptico} o
    trasposoma. Questo complesso contiene un multimero di trasposasi
    generalmente composto da due o quattro subunità e le due estremità
    del DNA. Il suo compito è quello di assicurare che le azioni di
    taglio e riunione del DNA necessarie allo spostamento del trasposone
    avvengano comtemporaneamente alle due estremità del DNA. Inoltre
    durante la ricombinazione protegge le estremità del DNA dagli enzimi
    cellulari.
  \item
    Escissione del trasposone dalla sua originaria posizione nel genoma.
    Le subunità della trasposasi all'interno del trasposoma iniziano con
    il tagliare un filamento ad ogni estremità del trasposone,
    esattamente alla giunzione tra il trasposone e la sequenza
    dell'ospite il cui esso è inserito (regione detta DNA fiancheggiante
    dell'ospite). L'enzima taglia il DNA in modo che la sequenza del
    trasposone termini, ad ogni estremità dell'elemento con dei gruppi
    3'-OH liberi. Per terminare la reazione di escissione, anche il
    secondo filamento ad ogni estremità dell'elemento a DNA deve essere
    tagliato. I diversi trasposoni usano diversi meccanismi per tagliare
    questi ``secondi filamenti'' di DNA.
  \item
    Le estremità 3'OH del trasposone a DNA attaccano i legami
    fosfodiesterici della doppia elica, nel sito di nuova inserzione.
    Questo segmento di DNA attaccato è chiamato \textbf{DNA bersaglio}.
    Il DNA bersaglio può essere costituito da qualsiasi sequenza.
    L'attacco consiste nel legame covalente tra il trasposone ed il DNA
    nel sito bersaglio. Si ha un taglio sfalsato del DNA bersaglio. Nel
    DNA bersaglio viene anche prodotta un'interruzione (nick).
  \item
    La reazione di saldatura avviene mediante un passaggio di
    transesterificazione noto come trasferimento del filamento a DNA.
  \item
    Si ha sintesi del DNA di riparazione per riempire le rotture
  \item
    Saldatura delle incisioni con duplicazione del nuovo sito bersaglio
  \end{itemize}
\end{enumerate}

IMMAGINE

I siti di taglio sul DNA bersaglio sono sfalsati di pochi nucleotidi
(spesso 2, 5, 9) e dunque si ha la formazione a fianco del trasposone
trasferito di corte interruzioni di DNA a singolo filamento che vengono
poi riempite da una DNA polimerasi di riparazione. Si parla infatti di
duplicazione dei siti bersaglio. Il DNA bersaglio è tagliato in modo da
generare gruppi 3'-OH liberi che vengono usati come ancore per la DNA
polimerasi nella sintesi di riparazione.

IMMAGINE

Una DNA ligasi salda poi i filamenti.

Come è stato descritto, l'enzima trasposasi taglia l'estremità 3'
dell'elemento a DNA e promuove il trasferimento del filamento per
catalizzare la trasposizione del tipo taglia e cuci.

Tuttavia i trasposoni che si muovono in questa maniera, hanno anche
bisogno di tagliare l'altro filamento del trasposone, quello che termina
con l'estremità 5' e che vengono chiamati \textbf{filamenti non
trasferiti}. I diversi trasposoni possono usare meccanismi diversi per
catalizzare la reazione del taglio del secondo filamento.

Riconosciamo 3 meccanismi:

\begin{itemize}
\item
  \textbf{Meccanismo a -- caso di Tn7} Una proteina diversa dalla
  trasposasi, nello specifico una endonucleasi, taglia l'estremità 5'.
  In particolare una specifica proteina codificata dal trasposone Tn7,
  detta TnsA, si assembla con la trasposasi codificata da Tn7, ossia la
  proteina TnsB, escindendo il trasposone dal suo sito originario. La
  TnsA lavora sul filamento con estremità 5' mentre a TnsB sul filamento
  con estremità 3';
\item
  \textbf{Meccanismo b -- caso di Tn5 e Tn10} La stessa trasposasi
  promuove il taglio del filamento che termina in 5' con un meccanismo
  di transesterificazione del DNA simile al trasferimento del filamento.
  I trasposoni Tn5 e Tn10 tagliano il filamento non trasferito formando
  una struttura nota come ``forcina''. Nella formazione della forcina la
  trasposasi usa l'estremità 3'-OH del trasposone a DNA precedentemente
  tagliata per attaccare un legame fosfodiesterico direttamente
  attraverso la doppia elica sull'altro filamento. Questa reazione
  taglia il filamento di DNA e attaccato e unisce covalentemente
  l'estremità 3' del trasposone ad un lato della rottura. Ne deriva che
  i due filamenti di DNA sono uniti covalentemente da un'estremità che
  forma un'ansa che ricorda la forma di una forcina. La forcina di DNA
  viene quindi tagliata ovvero aperta dalla trasposasi producendo una
  normale rottura a doppio filamento nel DNA. La reazione di apertura
  avviene su entrambe le estremità del trasposone;
\item
  \textbf{Meccanismo c -- caso di Hermes} Anche questo meccanismo si
  avvale della struttura a ``forcina'' ma l'ordine delle reazioni di
  taglio del filamento e di transesterificazione è diverso e quindi le
  forcine si formano sul DNA della cellula ospite e non sull'estremità
  dell'elemento trasponibile. Prima viene tagliato il filamento non
  trasferito e poi (estremità 5') e poi l'estremità 3'.
\end{itemize}

\begin{enumerate}
\def\labelenumi{\arabic{enumi}.}
\setcounter{enumi}{1}
\itemsep1pt\parskip0pt\parsep0pt
\item
  \textbf{Replicativa} Secondo questo meccanismo l'elemento di DNA
  trasponibile viene replicato ogni ciclo di trasposizione.

  \begin{itemize}
  \itemsep1pt\parskip0pt\parsep0pt
  \item
    Il primo passaggio nella trasposizione consiste nell'assemblaggio
    della trasposasi su entrambe le estremità del trasposone a
    costituire un trasposoma.
  \item
    Successivamente si tagliano le estremità del trasposone in una
    reazione catalizzata dalla trasposasi all'interno del trasposoma.
    L'enzima introduce un'incisione nel DNA su entrambe le giunzioni
    (tra le sequenze del trasposone e quelle fiancheggianti del DNA
    dell'ospite). Questo taglio libera le due estremità 3'-OH sulla
    sequenza del trasposone. In questo passaggio non si ha la completa
    escissione del trasposone dal genoma dell'ospite come invece avviene
    durante il meccanismo taglia e cuci.
  \item
    Le estremità 3'OH del trasposone sono unite al sito bersaglio dalla
    reazione di trasferimento del filamento con un meccanismo identico a
    quello che abbiamo precedentemente visto. In questo caso però
    l'intermedio che si viene a formare con il trasferimento del
    filamento è una molecola di DNA con una doppia ramificazione. Le
    estremità 3' del trasposone sono unite covalentemente al nuovo sito
    bersaglio, mentre le estremità 5' del trasposone restano unite al
    vecchio filamento di DNA fiancheggiante.Le due braccia di DNA
    sull'intermedio hanno la struttura di una forca replicativa. Dopo il
    trasferimento del filamento, le proteine replicative della cellula
    ospite possono assemblarsi su queste forche.
  \item
    Solo una delle due forche viene replicata e l'estremità 3'-OH viene
    utilizzata come innesco per la sintesi di DNA. La replicazione
    procede lungo la sequenza del trasposone e termina sulla seconda
    forca. Si generano così due copie del trasposone, ciascuna delle
    quali è fiancheggiata da una breve ripetizione diretta del sito
    bersaglio.
  \end{itemize}
\end{enumerate}

IMMAGINE

La trasposizione replicativa spesso provoca delle inversioni e delezioni
sul cromosoma che possono essere dannose per la cellula ospite. Forse
per questo motivo questo è un meccanismo di inserzione che in quanto
svantaggioso non è molto comune.

\begin{enumerate}
\def\labelenumi{\arabic{enumi}.}
\setcounter{enumi}{2}
\itemsep1pt\parskip0pt\parsep0pt
\item
  \textbf{Con un intermedio ad RNA} I retrotrasposoni simili ai virus e
  i retrovirus si inseriscono in nuovi siti nel genoma della cellula
  ospite utilizzando gli stessi passaggi di taglio del DNA e
  trasferimento del filamento che è stato descritto per i trasposoni a
  DNA. A differenza di questi ultimi però la ricombinazione per questi
  retroelementi passa attraverso un intermedio a RNA.

  \begin{itemize}
  \itemsep1pt\parskip0pt\parsep0pt
  \item
    Si ha per prima cosa la trascrizione della sequenza di DNA di un
    retrotrasposone (o di un retrovirus) in RNA da parte di una RNA
    polimerasi cellulare. La trascrizione inizia su una LTR e prosegue
    sull'elemento per portare alla formazione di una copia a RNA quasi
    completa dell'elemento a DNA.
  \item
    L'RNA viene retrotrascritto a DNA a doppio filamento ossia cDNA (DNA
    copiato), libera di ogni sequenza fiancheggiante l'ospite.
  \item
    Il cDNA viene riconosciuto da un'integrasi che si assembla
    all'estremità del cDNA, elimina alcuni nucleotidi all'estremità 3' e
    catalizza l'inserimento dell'estremità 3' tagliate nel sito
    bersaglio del genoma della cellula ospite (come nel meccanismo
    taglia e cuci e in quello replicativo). Il sio bersaglio può avere
    praticamente qualsiasi sequenza.
  \item
    le proteine della riparazione della cellula ospite riempono
    l'interruzione che si è venuta a formare sul sito bersaglio.
    L'innesco della sintesi del DNA a partire da RNA varia da una classe
    all'altra dei retrotrasposoni LTR. Spesso è un tRNA cellulare con
    estremità 3' complementare che si ibrida con l'estremità del
    trasposone o un altro RNA.
  \end{itemize}
\end{enumerate}

Dal momento che la trascrizione dell'intermedio a RNA inizia all'interno
di una delle due LTR, questo RNA non porterà l'intera sequenza dell'LTR,
infatti è privo della sequenza che va dal sito di inizio della
trascrizione all'estremità dell'LTR. Per riformare l'intera sequenza
dell'elemento durante la retrotrascrizione è richiesto un meccanismo
speciale. Questo si basa su due eventi di innesco interni e due scambi
del filamento.

Questi eventi di scambio portano alla duplicazione delle sequenze alle
estremità del cDNA. A questo punto il cDNA ha delle sequenze LTR
complete che sostituiscono le regioni perse durante la trascrizione.
Sono essenziali affinché l'integrasi riconosca e quindi ricombini il
DNA.

IMMAGINE

\subsubsection{Meccanismo di formazione del cDNA -- Attività della
trascrittasi
inversa}\label{meccanismo-di-formazione-del-cdna-attivituxe0-della-trascrittasi-inversa}

Ogni LTR è formata da tre elementi di sequenza detti: + U3, estremità
3'; + R, ripetizione; + U5, estremità 5'.

La trascrizione della copia integrata del genoma retrovirale produce
l'RNA virale con una sequenza R ad ogni estremità.

Durante il processo di retrotrascrizione è necessario dunque provvedere
alla sintesi anche delle regioni U3 e U5.

IMMAGINE

Nei retrotrasposoni non LTR:

\begin{itemize}
\itemsep1pt\parskip0pt\parsep0pt
\item
  una RNA polimerasi trascrive l'elemento a DNA integrato;
\item
  l'RNA viene trasportato nel citoplasma e tradotto per produrre ORF 1 e
  ORF2 (con attività endonucleasica e di trascrittasi inversa) che
  rimangono associate al trascritto che e ha codificate. In questo modo
  l'elemento promuove la propria trasposizione e non fornisce alcuna
  proteina a elementi che con esso competono;
\item
  il complesso ribonucleoproteico rientra ora nel nucleo dove si associa
  con il DNA cellulare;
\item
  l'endonucleasi inizia la reazione di integrazione introducendo un
  taglio nel DNA cromosomico. Le sequenze ricche di T costituiscono dei
  siti di taglio preferenziali. La presenza di queste T sul sito di
  taglio permette al DNA di appaiarsi con la coda di poli-A
  dell'elemento a RNA. L'estremità 3'-OH che si viene a formare serve
  ora come innesco per la retrotrascrizione dell'elemento ad RNA;
\item
  i passaggi finali, anche se non ancora ben compresi, includono la
  sintesi del secondo filamento di cDNA, la riparazione del sito di
  inserzione e la ligazione.
\end{itemize}

\subsection{Regolazione della
trasposizione}\label{regolazione-della-trasposizione}

La trasposizione è regolata in modo da cercare di stabilire una
coesistenza pacifica con la cellula ospite.

Questa coesistenza è fondamentale per la sopravvivenza dell'elemento in
quanto i trasposoni non possono esistere senza un organismo ospite.

Dall'altro lato i trasposoni possono causare gravi danni alla cellula
come per esempio mutazioni per inserzione, modificazioni
dell'espressione genica e l'induzione nel DNA di riarrangiamenti su
larga scala.

Due tipi di regolazione della trasposizione sembrano essere ricorrenti:

\begin{enumerate}
\def\labelenumi{\arabic{enumi}.}
\itemsep1pt\parskip0pt\parsep0pt
\item
  \textbf{Controllo del numero di copie}, in questo modo limitano il
  loro impatto deleterio sul genoma della cellula ospite;
\item
  \textbf{Controllo della scelta del sito bersaglio}

  \begin{itemize}
  \itemsep1pt\parskip0pt\parsep0pt
  \item
    alcuni elementi si inseriscono preferenzialmente in regioni del
    cromosoma che tendono a non essere pericolose per la cellula ospite.
    Queste regioni sono dette \textbf{rifugi di sicurezza} per i
    trasposoni;
  \item
    Alcuni trasposoni evitano specificamente di trasporsi nel proprio
    DNA. Questo fenomeno è detto \textbf{immunità dalla trasposizione
    del sito bersaglio}.
  \end{itemize}
\end{enumerate}

\subsubsection{Controllo del numero di copie -- Caso del trasposone
batterico
Tn10}\label{controllo-del-numero-di-copie-caso-del-trasposone-batterico-tn10}

IMMAGINE

Tn10 è un elemento compatto di 9 kb che contiene al suo interno il gene
per la propria trasposasi e i geni che conferiscono la resistenza
all'antibiotico tetraciclina.

Questo trasposone utilizza il meccanismo di tipo taglia e cuci usando la
strategia della forcina a DNA per tagliare i filamenti non trasferiti.

È un trasposone composito organizzato in tre moduli funzionali. I due
elementi più esterni sono \textbf{IS10L} (left) e \textbf{IS10R} (right)
che fungono da mini trasposoni. \textbf{IS} sta per \emph{``inserzione
di sequenza''}. IS10R è autonomo mentre IS10L non lo è. Entrambi gli
elementi tuttavia si trovano nei genomi anche non associati a Tn10.

Tn10 limita il proprio numero di copie all'interno della cellula con
delle strategie che riducono la frequenza di trasposizione.

Un meccanismo consiste nell'utilizzare un RNA antisenso per controllare
l'espressione dle gene della trasposasi.

Vicino all'estremità IS10R si trovano due promotori che dirigono la
sintesi dell'RNA da parte della RNA polimerasi della cellula ospite.

Il promotore che dirige la sintesi dell'RNA verso l'interno, detto PIN,
è responsabile dell'espressione del gene della trasposasi.

Il promotore che dirige la trascrizione verso l'esterno,
P\(_O\)\(_U\)\(_T\), serve a regolare l'espressione della trasposasi
mediante un RNA antisenso.

P\(_O\)\(_U\)\(_T\) si trova 36 basi a sinistra di P\(_I\)\(_N\) e
dunque le gli RNA sintetizzati dai promotori si sovrappongono per 36
basi formando legami a idrogeno tra queste regioni di sovrapposizione
complementari.

Il trascritto P\(_O\)\(_U\)\(_T\) ha un'emivita maggiore di
P\(_I\)\_N\$.

L'appaiamento impedisce ai ribosomi di legarsi al trascritto che deriva
dal promotore PIN impedendo quindi la sintesi di trasposasi.

Le cellule con un numero di copie più alto di Tn10 trascriveranno una
maggiore quantità di RNA antisenso che quindi limiterà l'espressione del
gene della trasposasi riducendo in questo modo il numero di copie
dell'elemento. Qui l'appaiamento RNA-RNA p frequente.

Se nella cellula c'è una sola copia di Tn10 il livello di RNA antisenso
sarà basso, la sintesi della proteina che permette la trasposizione sarà
alta e quindi la trasposizione avverrà con un'elevata frequenza.
L'appaiamento RNA-RNA è raro.

IMMAGINE

\subsection{Immunità della trasposizione del sito bersaglio -- Caso del
fago
Mu}\label{immunituxe0-della-trasposizione-del-sito-bersaglio-caso-del-fago-mu}

Il fago Mu è un batteriofago lisogenico ed un grosso trasposone a DNA.

Questo fago durante l'inserzione usa la trasposizione per inserire il
proprio DNA nel genoma della cellula ospite e in questo è molto simile
ai retrovirus.

Il nome Mu è l'abbreviazione di Mutator e deriva dalla capacità di
questo fago di trasporsi in modo promiscuo.

Il genoma di Mu è di circa 40 kb e contiene più di 35 geni dei quali
solo due codificano per proteine coinvolte nella trasposizione. Questi
sono i geni A e B che codificano per le proteine MuA e MuB. \textbf{MuA}
è una trasposasi membro della superfamiglia delle proteine DDE
\textbf{MuB} è un'ATPasi che stimola l'attività di MuA e controlla la
scelta del sito di DNA bersaglio. L'interazione tra MuA e MuB stimola
MuB a idrolizzare ATP e a staccarsi dal DNA.

Siccome Mu ha scarsa selettività di sequenza per il proprio sito
bersaglio, risolve il problem,a grazie ad un processo di immunità della
trasposizione del sito bersaglio.

Le sequenze che circondano la copia dell'elemento Mu e anche il DNA
dell'elemento stesso diventano bersagli molto improbabili per un nuovo
evento di trasposizione.

Nel caso di Mu le sequenze distanti circa 15 kb da una precedente
inserzione sono immuni a nuove inserzioni. L'immunità del bersaglio
protegge un elemento dalla trasposizione in se stesso o dalla
possibilità di avere un'altra copia di un elemento dello stesso tipo
inserita nel proprio genoma.

Questo tipo di regolazione della scelta del sito bersaglio è una forza
propulsiva per gli elementi nella ricerca di nuove posizioni lontane da
quelle in cui erano inizialmente inseriti, una cartteristica che
potrebbe essere vantaggiosa alla loro propagazione e sopravvivenza.

IMMAGINE

Quattro subunità della trasposasi MuA si assemblano alle estremità del
DNA di Mu. MuB lega l'ATP e poi lega qualunque sequenza di DNA.

Un'interazione proteina-proteina tra MuA e MuB porta il complesso di MuA
legato al DNA del trasposoma su un nuovo sito del bersaglio. MuB non è
più mostrata nel disegno finale perché, dopo il trasferimento del
filamento, non è più necessaria e probabilmente lascia il complesso.

IMMAGINE

Le interazioni tra MuA e MuB impediscono a MuB di legarsi al DNA in
prossimità de legame di MuA.

MuA inibisce il legame di MuB ai siti di DNA posti nelle immediate
vicinanze. Questa inibizione richiede l'idrolisi di ATP.

MuB aiuta MuA a trovare un sito bersaglio per la trasposizione:

\begin{itemize}
\itemsep1pt\parskip0pt\parsep0pt
\item
  \textbf{regione naturale} Quando Mu si prepara a trasporsi, MuB legata
  all'ATP si legherà al DNA avvalendosi della sua attività di legame non
  specifico alla doppia elica. Contemporaneamente la trasposasi MuA
  porterà alla formazione di un trasposoma sul DNA Mu. La proteina MuA
  presente nel trasposoma può ora instaurare delle interazioni proteina
  -- proteina con il complesso MuB-DNA sulla regione naturale. Il
  risultato di questa interazione è che MuB fornisce a MuA la regione di
  DNA da usare come sito bersaglio.
\item
  \textbf{regione immune} Sia MuA che MuB si legano al DNA nella regione
  immune. MuA interagisce in modo specifico con i siti di legame del
  genoma Mu già presenti. Anche MutB-ATP si lega grazie alla propria
  affinità per qualunque sequenza di DNA. Quando sia MuA che MuB saranno
  legate a questa regione, interagiranno.
\end{itemize}

Ne consegue che MuA stimola MuB a idrolizzare ATP e MuB si dissocia da
questa sequenza. Pertanto MuB non si accumula su questa regione di DNA
immune.

In questo modo le proteine per la trasposizione Mu usano l'energia
accumulata nell'ATP per proteggere il genoma Mu dalla possibilità di
divenire bersaglio della trasposizione. È sufficiente la presenza di un
singolo sito di legame per MuA in una molecola di DNA per conferire
l'immunità per la trasposizione del sito bersaglio.

\subsection{Rifugi di sicurezza -- caso dell'elemento Ty del
lievito}\label{rifugi-di-sicurezza-caso-dellelemento-ty-del-lievito}

Gli elementi Ty (transposons in yeast -- trasposoni nel lievito) sono
dei trasposoni simili a virus.

Ci sono molti tipi di elementi Ty e ciascuna classe di questi elementi
Ty induce la propria mobilità ma non è capace di rendere mobili gli
elementi appartenenti ad una classe diversa.

Gli elementi Ty si integrano preferenzialmente in specifiche regioni del
cromosoma. Per esempio gli elementi Ty1 e Ty3 si traspongono quasi
sempre in un DNA che abbia a monte (rispettivamente a circa 200 bp e a
circa 2 bp) un sito di inizio per la trascrizione della RNA polimerasi
III che trascrive specificamente i geni del tRNA (la maggior parte delle
inserzioni Ty1 avviene in prossimità di questi geni).

Al contrario Ty5 preferisce integrarsi in regioni del genoma che sono
silenti dal punto di vista trascrizionale come ad esempio i telomeri. Il
meccanismo con cui avviene la selezione della regione contenente il sito
bersaglio richiede la formazione di specifici complessi
proteina-proteina tra l'integrasi dell'elemento e specifiche proteine
dell'ospite legate a questi siti cromosomici.

Per esempio l'integrasi di Ty5 forma un complesso specifico con la
proteina necessaria per il silenziamento del DNA Sir4.

È stato ipotizzato che la specificità del bersaglio permetta ai
trasposoni di conservarsi nell'organismo ospite, in quanto concentrano
la propria inserzione in regioni lontane da siti non direttamente
coinvolti nella sintesi di proteine.

\end{document}
