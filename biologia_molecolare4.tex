\documentclass[]{article}
\usepackage{lmodern}
\usepackage{amssymb,amsmath}
\usepackage{ifxetex,ifluatex}
\usepackage{fixltx2e} % provides \textsubscript
\ifnum 0\ifxetex 1\fi\ifluatex 1\fi=0 % if pdftex
  \usepackage[T1]{fontenc}
  \usepackage[utf8]{inputenc}
\else % if luatex or xelatex
  \ifxetex
    \usepackage{mathspec}
    \usepackage{xltxtra,xunicode}
  \else
    \usepackage{fontspec}
  \fi
  \defaultfontfeatures{Mapping=tex-text,Scale=MatchLowercase}
  \newcommand{\euro}{€}
\fi
% use upquote if available, for straight quotes in verbatim environments
\IfFileExists{upquote.sty}{\usepackage{upquote}}{}
% use microtype if available
\IfFileExists{microtype.sty}{%
\usepackage{microtype}
\UseMicrotypeSet[protrusion]{basicmath} % disable protrusion for tt fonts
}{}
\ifxetex
  \usepackage[setpagesize=false, % page size defined by xetex
              unicode=false, % unicode breaks when used with xetex
              xetex]{hyperref}
\else
  \usepackage[unicode=true]{hyperref}
\fi
\hypersetup{breaklinks=true,
            bookmarks=true,
            pdfauthor={},
            pdftitle={},
            colorlinks=true,
            citecolor=blue,
            urlcolor=blue,
            linkcolor=magenta,
            pdfborder={0 0 0}}
\urlstyle{same}  % don't use monospace font for urls
\setlength{\parindent}{0pt}
\setlength{\parskip}{6pt plus 2pt minus 1pt}
\setlength{\emergencystretch}{3em}  % prevent overfull lines
\setcounter{secnumdepth}{0}

\date{}

\begin{document}

\section{Il clonaggio del DNA}\label{il-clonaggio-del-dna}

Il termine clonaggio di un segmento di DNA indica làintroduzione di un
gene )o un suo segmento= allòinterno di una cellula, in modo che tale
frammento venga copiato quando la cellula si replica, determinando la
produzione di numerose cellule tutte contenenti almeno una copia di quel
frammento.

Per gli esperimenti di clonaggio viene generata una molecola di
\textbf{DNA ricombinante}, cioè isolata dal suo contesto e introdotta in
un \textbf{replicone} (una unità di DNA capace di replicarsi detta
vettore).

La molecola di DNA ricombinante viene introdotta in un sistema cellulare
appropriato. Tutti i discendenti di questa singola cellula, detta
\textbf{clone}, conterranno la medesima molecola di DNA ricombinante
originariamente isolata, permettendo di produrne quantità praticamente
illimitate.

Il clonaggio molecolare di un frammento di DNA parte d auna miscela
complessa di frammenti generati dalla rottura meccanica o dal taglio con
ER (enzimi di restrizione) del genoma di un certo organismo e usa poi
cellule viventi per riprodurre molte copie di uno specifico frammento
che è stato inserito all'interno di ``trasportatori specializzati''
chiamati \textbf{vettori di clonaggio}.

Esistono diversi tipi di vettori di clonaggio.

La principale considerazione da fare è relativa alle \textbf{dimensioni}
dell'inserto di DNA che ogni vettore può accettare.

Dai plasmidi batterici naturali sono derivati \textbf{vettori di
clonaggio plasmidici}. Tra i vari vettori disponibili, i plasmidi hanno
acquistato un posto privilegiato per la loro \emph{affidabilità} e
\emph{facilità di manipolazione}.

\subsection{Caratteristiche ideali di un
vettore}\label{caratteristiche-ideali-di-un-vettore}

Per svolgere la sua funzione, un vettore plasmidico \emph{deve
possedere} le seguenti proprietà:

\begin{itemize}
\itemsep1pt\parskip0pt\parsep0pt
\item
  un \textbf{origine di replicazione} (\emph{ori}) che gli permetta di
  replicarsi autonomamente una volta introdotto all'interno della
  cellula ospite;
\item
  uno o più \textbf{marcatori genetici} o \textbf{``marcatori di
  selezione''} che permettano di individuare le cellule che contengono
  il vettore e qualsiasi frammento di DNA in esso inserito. I marcatori
  genetici possono essere \emph{resistenze ad antibiotici} o
  \emph{marcatori nutrizionali} (es. auxotrofie). In questo ultimo caso
  si devono usare ceppi batterici con mutazioni adatte;
\item
  contenere il maggior numero possibile di \textbf{siti di restrizione
  unici}, cioè presenti una sola volta nel loro DNA (siti di taglio che
  presentano il più alto numero possibili di enzimi di restrizione.
  Questi siti devono stare \textbf{al di fuori} delle regioni essenziali
  (replicone e marcatore).
\end{itemize}

Non è essenziale, ma molto utile:

\begin{itemize}
\itemsep1pt\parskip0pt\parsep0pt
\item
  \emph{conoscere l'intera sequenza nucleotidica del vettore}, in modo
  da poter analizzare al computer tutti gli elementi di interesse (siti
  di taglio, prodotti di digestione..);
\item
  che il \emph{sito di clonaggio sia fiancheggiato da sequenze per i}
  \textbf{primer universali} usati nella PCR e nel sequenziamento;
\item
  che il sito di clonaggio sia fiancheggiato da elementi di controllo
  che consentano la trascrizione dell'inserto di DNA in RNA
  (\textbf{vettori di trascrizione}) o la sua espressione in cellule di
  procarioti o di eucarioti (\textbf{vettori di espressione}).
\end{itemize}

\subsubsection{I plasmidi come vettori}\label{i-plasmidi-come-vettori}

I plasmidi sono degli \textbf{elementi genetici extracrosomali} che si
\emph{replicano autonomamente} e che \emph{segregano autonomamente}
rispetto al DNA cromosomale batterico.

Variano da 1 a 200 Kb e sono molto diffusi tra i procarioti. Esempi di
plasmidi batterici naturali sono i \textbf{plasmidi pcolE1} di E.coli
(colina), i \textbf{plasmidi F} di E.coli (coniugazione), i
\textbf{plasmidi Ti} o \textbf{Ri} di Agrobacterium (galla del colletto
nelle dicotiledoni)\ldots{}

I plasmidi possono essere \emph{lineari} o \emph{integrati} nel
cromosoma batterico ma, nella maggior parte dei casi, sono molecole di
DNA \emph{circolari}.

Nell'ospite batterico i plasmidi si presentano come \textbf{molecole
circolari superavvolte} che, durante le manipolazioni sperimentali,
possono rilassarsi o linearizzarsi in seguito a rotture a singolo o a
doppio filamento.

La caratteristica più importante dei plasmidi è quella di essere dei
\textbf{repliconi}, cioè molecole capaci di replicazione autonoma. Un
replicone è costituito da un **origine di replicazione, chiamata ori e
da elementi di controllo.

Si conoscono circa 30 repliconi, ma la maggior parte dei plasmidi di
clonaggio possiede il replicone di \textbf{pMB1} (un plasmide naturale)
che è \emph{identico al replicone di pcolE1}.

I plasmidi si replicano per replicazione \(\theta\) (uni o
bi-direzionale) o per circolo rotante.

Richiedono proteine plasmidiche e/o dell'ospite batterico.

Oltre ad essere essenziale per la replicazione, l'origine di
replicazione controlla:

\begin{itemize}
\itemsep1pt\parskip0pt\parsep0pt
\item
  il numero di copie del plasmide;
\item
  la specificità d'ospite.
\end{itemize}

\paragraph{I marcatori di selezione}\label{i-marcatori-di-selezione}

I plasmidi naturali a volte codificano per \emph{funzioni non
essenziali}, mentre a volte \emph{conferiscono un vantaggio selettivo}
in alcune situazioni. Per esempio possono codificare per le tossine
batteriche o per geni di resistenza agli antibiotici.

In alcuni casi, tuttavia, nessun vantaggio competitivo sembra essere
associato alla presenza di geni di resistenza.

Tutti i vettori di clonaggio includono \emph{almeno} un marcatore di
selezione. Lo scopo essenziale è di \textbf{distinguere} e di
\textbf{selezionare} le molecole ricombinanti. I marcatori di selezione
più utilizzati nei batteri sono i geni di resistenza agli antibiotici.

Per esempio il gene per la \(\beta\)-lattamasi codifica per un enzima
capace di idrolizzare l'anello lattamico degli antibiotici di tipo
penicillinico (es. l'ampicillina). I batteri che contengono un plasmide
con questo gene, quindi,(simboleggiato con Amp o Ap) possono crescere in
terreni di coltura che contengono l'antibiotico ampicillina.

\paragraph{I siti di restrizione
unici}\label{i-siti-di-restrizione-unici}

Per effettuare un clonaggio molecolare è necessario avere sempre
\emph{almeno un sito di riconoscimento} per una \textbf{endonucleasi di
restrizione}.

Il sito di riconoscimento per una endonucleasi di restrizione deve
essere \emph{presente} nel vettore \emph{una sola volta} per non
distruggere l'integrità fisica del plasmide e \emph{non deve essere
presente in regioni cis essenziali} (es. ori o promotori) o \emph{in
geni che codificano per funzioni essenziali} (es. geni di resistenza).

\begin{itemize}
\itemsep1pt\parskip0pt\parsep0pt
\item
  Un esempio di plasmide di clonaggio è il \textbf{PBR322}.
\end{itemize}

Questo è un vettore primitivo con un numero limitato di siti di
restrizione unici (20) distribuiti su tutta la molecola di DNA. È di
piccole dimensioni, 4363 bp.

Contiene \textbf{due geni per la resistenza agli antibiotici},
\textbf{Amp e Tet} (insieme di geni che codificano per enzimi
detossificanti la tetraciclina), al cui interno sono presenti siti di
restrizione unici utilizzabili per il clonaggio.

Per es. la clonazione in PstI inattiva il gene Amp, mentre la clonazione
in BamHI inattiva il gene Tet.

\begin{itemize}
\itemsep1pt\parskip0pt\parsep0pt
\item
  Un vettore plasmidico più evoluto è il \textbf{pUC19}.
\end{itemize}

Questo vettore presenta un sito di clonaggio introdotto con la tecnica
delle clonazione di oligonucleotidi sintetici.

È una serie plasmidica che differisce per la lunghessa e l'orientamento
del \textbf{polylinker}.

Un polylinker o \textbf{multiple cloning site (MCS)}, è un corto
segmento di DNA contenente molti siti di restrizione (circa 21). I siti
di restrizione all'interno di un MCS sono solitamenti unici, ovvero sono
presenti una sola volta all'interno dello stesso plasmide.

Il polylinker è una regione del plasmide nella quale può essere inserito
un DNA esterno.

Il sito di policlonaggio è inserito nel \textbf{gene lacZ} in modo da
tenere la cornice di lettura della proteina; questo permette di
individuare i cloni ricombinanti con un saggio enzimatico
(\(\alpha\)-complementazione) e di esprimere la proteina corrispondente
se l'inserto è inserito in fase.

Presenta resistenza all'Amp.

È un plasmide di espressione.

\paragraph{Evoluzione dei vettori di
clonaggio}\label{evoluzione-dei-vettori-di-clonaggio}

Da questi due vettori di clonaggio sono derivati decine di nuovi altri
vettori.

La tendenza è quella di creare vettori più piccoli e sempre più
funzionali.

Ci sono numerosi vantaggi, infatti, a ridurre la dimensione di un
plasmide:

\begin{enumerate}
\def\labelenumi{\arabic{enumi}.}
\itemsep1pt\parskip0pt\parsep0pt
\item
  è \textbf{più maneggevole}. Per esempio è più difficile danneggiarlo o
  introdurvi interruzzioni a singola elica durante le manipolazioni
  sperimentali;
\item
  è \textbf{più facile estrarlo}. I principali metodi di separazione dei
  plasmidi dal cromosoma batterico si basano sulla denaturazione degli
  acidi nucleici (per es. mediante calore o basi diluite) e sulla loro
  successiva rinaturazione. Mentre i plasmidi, di piccole dimensioni,
  rinaturano rapidamente, il grosso cromosoma batterico non riesce a
  rinaturare velocemente e viene selettivamente eliminato.
\end{enumerate}

La \emph{velocità di rinaturazione plasmidica è} \textbf{inversamente
proporzionale} \emph{alla dimensione}. Quanto più piccoli sono, tanto
più facile é il loro isolamento;

\begin{enumerate}
\def\labelenumi{\arabic{enumi}.}
\setcounter{enumi}{2}
\itemsep1pt\parskip0pt\parsep0pt
\item
  è \textbf{più facile introdurlo dentro un batterio}. I metodi di
  ``trasformazione'' sono essenziali nella tecnologia del DNA
  ricombinante. Esistono varie tecniche, come la \emph{trasformazione
  con CaCl 2 o l'elettroporazione}.
\end{enumerate}

In tutti i casi \emph{l'efficienza di trasformazione è}
\textbf{inversamente proporzionale} \emph{alla dimensione plasmidica}.

Un'ulteriore tendenza è quella di \textbf{sostituire i siti di
restrizione unici con ``Multi Cloning Sites''} sempre più completi.
Questa caratteristica (in genere) facilita il lavoro di clonaggio
permettendo di utilizzare l'enzima di restrizione più conveniente.

Questo problema é particolarmente sentito quando si devono clonare
inserti di grosse dimensioni in cui possono essere presenti numerosi
siti di restrizione.

Numerosi altri vettori più o meno ``specializzati'' sono reperibili per
gli utilizzi più disparati: ``trascrizione in vitro'', inserzioni di
trasposoni, selezione di mutazioni, clonaggio di frammenti amplificati
con PCR, vettori ``shuttle'' che contengono più origini di replicazione
ecc.

\begin{itemize}
\itemsep1pt\parskip0pt\parsep0pt
\item
  Il \textbf{pET system}
\end{itemize}

Un vettore pET consiste in un plasmide batterico ``disegnato'' per
attivare una rapida produzione di una grande quantità di una
qualsivoglia proteina, quando attivata.

Questo plasmide contiene diversi elementi importanti:

\begin{itemize}
\itemsep1pt\parskip0pt\parsep0pt
\item
  un \textbf{gene lacI} che codifica per il repressore proteico di
  \emph{lac};
\item
  un \textbf{promotore di T7}, specifico solo per la T7 RNA polimerasi
  (non per le RNA polimerasi batteriche) e che non è presente ovunque
  nel genoma procariotico;
\item
  un \emph{operatore che può bloccare la trascrizione};
\item
  un \textbf{polylinker};
\item
  un'\textbf{origine di replicazione f1} (così che un plasmide a singolo
  filamento possa essere prodotto quando co-infettato con un fago M13)
\item
  un \textbf{gene di resistenza all'ampicillina};
\item
  un'\textbf{origine di replicazione ColE1}.
\end{itemize}

Per iniziare il processo, un gene a scelta che chiamereno \textbf{YFG},
viene clonato nel sito del polylinker all'interno di un plasmide pET.

Sia il promotore T7 che l'operatore lac sono localizzati all'estremità
5' dell'YFG. Quando lA T7 RNA polimerasi è presente e l'operatore lac
non è represso, la trascrizione di YFG procede rapidamente.

Poichè T7 è un promotore virale, viene trascritto rapidamente e in
abbondanza finchè la T7 RNA polimerasi è presente.

L'espressione di YFP (la proteina codificata dal gene da noi scelto)
cresce rapidamente insieme all'aumento dell'mRNA trascritto da YFG. In
poche ore, YFP diventa uno the componenti maggiormente presenti nella
cellula.

\section{Isolamento e purificazione di DNA e
RNA}\label{isolamento-e-purificazione-di-dna-e-rna}

Il primo passo di qualunque tecnica di biologia molecolare consiste
nell'isolare e purificare gli acidi nucleici.

I dettagli sperimentali variano a seconda degli organismi, del tipo di
acido nucleico che si vuole separare, dal tipo di esperimento che si
deve effettuare, ecc.

In tutti i casi dovremo:

\begin{itemize}
\itemsep1pt\parskip0pt\parsep0pt
\item
  rompere la parete o la membrana cellulare;
\item
  separare gli acidi nucleici dagli altri componenti cellulari;
\item
  separare gli acidi nucleici tra loro (per es. il DNA dal RNA).
\end{itemize}

\subsection{Rottura della parete o della membrana
cellulare}\label{rottura-della-parete-o-della-membrana-cellulare}

Sebbene esistano metodi differenti per estrarre acidi nucleici dalle
cellule, tutti hanno in comune alcune caratteristiche di base.

Per prima cosa è necessario procurarsi il materiale biologico di
partenza, separandolo per centrifugazione dal terreno di coltura, come
nel caso dei batteri, o frazionandolo ed omogenizzandolo, in caso di
tessuti più complessi.

Il secondo passo consiste nella lisi delle cellule, affinchè queste
rilascino i loro componenti cellulari (in funzione di diversi tipi
cellulari vengono utilizzati metodi diversi). Nel caso delle cellule
procariotiche oltre alle membrane cellulari bisogna distruggere la
parete cellulare di peptidoglicano. In genere si utilizzano miscele di
\emph{lisozima}, \emph{detergenti ionici} (tipicamente SDS) ed
\emph{EDTA}.

Il lisozima serve per indebolire la parete di peptidoglicano, l'SDS
provvede a solubilizzare i lipidi delle membrane, mentre l'EDTA è un
agente chelante che sequestra cationi bivalenti necessari per la
stabilizzazione delle membrane e per l'attività di molti enzimi tra cui
la DNasi.

Per cellule animali in genere bastano omogenizzazioni in tamponi
ipoosmolari e/o detergenti ionici e non ionici.

Nel caso di cellule protette da pareti cellulari più resistenti, come i
lieviti o le cellule vegetali, è necessario rompere la parete con metodi
fisici (cicli di congelamento-scongelamento, biglie di vetro,
sonicazione, utilizzo di mortaio e pestello ecc.) oppure ricorrere a
metodi enzimatici capaci di digerire la parete cellulare.

Dopo la rottura delle pareti cellulari e della membrana plasmatica, e la
separazione della frazione solubile da quella insolubile, si otterà una
miscela complessa costituita da varie componenti cellulari come DNA,
RNA, lipidi, mono e polisaccaridi, proteine e sali.

La rottura delle cellule induce quasi sempre una parziale frammentazione
del DNA cromosomale. Il cromosoma batterico, in particolare, che nella
sua forma nativa si trova sempre in forma circolare, verrà ridotto in
frammenti lineari più o meno lunghi in funzione del tipo di trattamento.
In tutti i casi è opportuno utilizzare il metodo più blando possibile
per minimizzare eventuali danni al DNA, particolarmente a quello
genomico.

\subsection{Separazione degli acidi nucleici da altri componenti
cellulari}\label{separazione-degli-acidi-nucleici-da-altri-componenti-cellulari}

Una volta lisata la membrana (ed eventualmente la parete) bisogna
separare gli acidi nucleici dagli altri componenti cellulari.

A questo scopo è pratica comune precipitare gli acidi nucleici con alcol
da soluzioni preventivamente deproteinizzate.

La rimozione delle proteine dal lisato cellulare è particolarmente
importante sia perché tra le proteine sono presenti enzimi capaci di
degradare gli acidi nucleici, sia per la presenza di proteine capaci di
legarsi agli acidi nucleici impedendone la funzione e/o la
purificazione.

Una tecnica comunemente utilizzata per rimuovere il grosso delle
proteine dagli acidi nucleici consiste nel trattare la soluzione acquosa
contenente gli acidi nucleici con solventi organici immiscibili (quasi)
in acqua, tipicamente fenolo/cloroformio.

Emulsionando i due componenti si formano due fasi distinte,
all'interfaccia delle quali si denaturano e si stratificano la maggior
parte delle proteine. Poiché l'acqua ed il fenolo sono solo parzialmente
immiscibili, il fenolo deve essere equilibrato con una soluzione
tampone. Se si usa un fenolo equilibrato con un tampone a pH neutro o
alcalino, sia il DNA che llRNA si ripartiranno nella fase acquosa
superiore.

Se si effetua un'estrazione con un fenolo equilibrato con un tampone
acido o con acqua (visto che il fenolo è naturalmente acido), il solo
RNA ripartirà nella fase acquosa superiore mentre il DNA verrà
trattenuto nella fase organica inferiore.

\{img/estrazione-con-fenolo\}

Poiché il fenolo è parzialmente solubile in acqua, alcune sue tracce
possono rimanere in soluzione e inibire successivi trattamenti
enzimatici (denaturando gli enzimi!). È importante, quindi, rimuovere
ogni traccia di fenolo estraendolo con cloroformio e, in qualche caso
estraendo anche eventuali tracce di cloroformio con etere che, infine
verrà eliminato per evaporazione.

\subsection{Separazione degli acidi nucleici tra
loro}\label{separazione-degli-acidi-nucleici-tra-loro}

Dopo aver rotto l'involucro cellulare e separato gli acidi nucleici
dagli altri componenti solubili (proteine, carboidrati ecc.) avremo una
miscela costituita prevalentemente da DNA genomico, DNA plasmidico ed
RNA (rRNA, tRNA, mRNA).

Per separare il DNA si compie un trattamento con Rasi seguito da una
precipitazione selettiva con isopropanolo (a temperatura ambiente).

In questo modo si ottiene una miscela contenente DNA genomico e
plasmidico.

Per separare i 2 DNA si sfruttano le differenze di peso molecolare
mediante un ciclo di denaturazione e rinaturazione.

Per separare l'RNA, invece, si compie un trattamento con Dnasi seguito
da una precipitazione selettiva con LiCi e da un trattamento con
guanidina tiocinato o con inibitori delle RNasi.

A questo punto, per separare l'mRNA dal resto dell'RNA si compie una
purificazione per affinità su cellulosa.

\subsubsection{Precipitazione con alcol}\label{precipitazione-con-alcol}

Dopo l'estrazione fenolica e, in linea generale, per purificare
ulteriormente o per concentrare gli acidi nucleici, si ricorre alla
precipitazione con alcol.

In soluzioni alcoliche, infatti, gli acidi nucleici precipitano, insieme
a parte dei sali, e possono essere efficacemente separati da altri
componenti cellulari più solubili.

Anche la maggior parte delle proteine sono insolubili in alcol e co-
precipiterebbero in larga misura con gli acidi nucleici se non fossero
preventivamente rimosse.

Generalmente si usano 2 volumi di etanolo a (15' a -20°C) o 0,7 volumi
di isopranolo (5' a temperatura ambiente). \textbf{???}

Dopo la precipitazione, che avviene in presenza di cationi monovalenti
(Na\(^+\)) con funzione di carrier, la soluzione viene centrifugata,
seccata e risospesa in adeguati tamponi alla concentrazione desiderata.

L'alcool etilico determina modificazioni strutturali degli acidi
nucleici che ne inducono l'aggregazione e quindi la precipitazione.

Gli acidi nucleici possono essere completamente ed efficacemente
separati anche mediante centrifugazioni differenziali.

La purificazione del DNA su gradiente isopicnico di saccarosio in
bromuro di etidio è stata molti anni il metodo di purificazione più
usato. Nonostante questo metodo fornisca DNA di grande purezza, a causa
della sua laboriosità e potenziale tossicità, è stato in gran parte
soppiantato da metodi basati su resine a scambio ionico.

\end{document}
