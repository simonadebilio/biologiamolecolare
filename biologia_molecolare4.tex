\documentclass[]{article}
\usepackage{lmodern}
\usepackage{amssymb,amsmath}
\usepackage{ifxetex,ifluatex}
\usepackage{fixltx2e} % provides \textsubscript
\ifnum 0\ifxetex 1\fi\ifluatex 1\fi=0 % if pdftex
  \usepackage[T1]{fontenc}
  \usepackage[utf8]{inputenc}
\else % if luatex or xelatex
  \ifxetex
    \usepackage{mathspec}
    \usepackage{xltxtra,xunicode}
  \else
    \usepackage{fontspec}
  \fi
  \defaultfontfeatures{Mapping=tex-text,Scale=MatchLowercase}
  \newcommand{\euro}{€}
\fi
% use upquote if available, for straight quotes in verbatim environments
\IfFileExists{upquote.sty}{\usepackage{upquote}}{}
% use microtype if available
\IfFileExists{microtype.sty}{%
\usepackage{microtype}
\UseMicrotypeSet[protrusion]{basicmath} % disable protrusion for tt fonts
}{}
\ifxetex
  \usepackage[setpagesize=false, % page size defined by xetex
              unicode=false, % unicode breaks when used with xetex
              xetex]{hyperref}
\else
  \usepackage[unicode=true]{hyperref}
\fi
\hypersetup{breaklinks=true,
            bookmarks=true,
            pdfauthor={},
            pdftitle={},
            colorlinks=true,
            citecolor=blue,
            urlcolor=blue,
            linkcolor=magenta,
            pdfborder={0 0 0}}
\urlstyle{same}  % don't use monospace font for urls
\setlength{\parindent}{0pt}
\setlength{\parskip}{6pt plus 2pt minus 1pt}
\setlength{\emergencystretch}{3em}  % prevent overfull lines
\setcounter{secnumdepth}{0}

\date{}

\begin{document}

\section{Il clonaggio del DNA}\label{il-clonaggio-del-dna}

Il termine clonaggio di un segmento di DNA indica làintroduzione di un
gene )o un suo segmento= allòinterno di una cellula, in modo che tale
frammento venga copiato quando la cellula si replica, determinando la
produzione di numerose cellule tutte contenenti almeno una copia di quel
frammento.

Per gli esperimenti di clonaggio viene generata una molecola di
\textbf{DNA ricombinante}, cioè isolata dal suo contesto e introdotta in
un \textbf{replicone} (una unità di DNA capace di replicarsi detta
vettore).

La molecola di DNA ricombinante viene introdotta in un sistema cellulare
appropriato. Tutti i discendenti di questa singola cellula, detta
\textbf{clone}, conterranno la medesima molecola di DNA ricombinante
originariamente isolata, permettendo di produrne quantità praticamente
illimitate.

Il clonaggio molecolare di un frammento di DNA parte d auna miscela
complessa di frammenti generati dalla rottura meccanica o dal taglio con
ER (enzimi di restrizione) del genoma di un certo organismo e usa poi
cellule viventi per riprodurre molte copie di uno specifico frammento
che è stato inserito all'interno di ``trasportatori specializzati''
chiamati \textbf{vettori di clonaggio}.

Esistono diversi tipi di vettori di clonaggio.

La principale considerazione da fare è relativa alle \textbf{dimensioni}
dell'inserto di DNA che ogni vettore può accettare.

Dai plasmidi batterici naturali sono derivati \textbf{vettori di
clonaggio plasmidici}. Tra i vari vettori disponibili, i plasmidi hanno
acquistato un posto privilegiato per la loro \emph{affidabilità} e
\emph{facilità di manipolazione}.

\subsection{Caratteristiche ideali di un
vettore}\label{caratteristiche-ideali-di-un-vettore}

Per svolgere la sua funzione, un vettore plasmidico \emph{deve
possedere} le seguenti proprietà:

\begin{itemize}
\itemsep1pt\parskip0pt\parsep0pt
\item
  un \textbf{origine di replicazione} (\emph{ori}) che gli permetta di
  replicarsi autonomamente una volta introdotto all'interno della
  cellula ospite;
\item
  uno o più \textbf{marcatori genetici} o \textbf{``marcatori di
  selezione''} che permettano di individuare le cellule che contengono
  il vettore e qualsiasi frammento di DNA in esso inserito. I marcatori
  genetici possono essere \emph{resistenze ad antibiotici} o
  \emph{marcatori nutrizionali} (es. auxotrofie). In questo ultimo caso
  si devono usare ceppi batterici con mutazioni adatte;
\item
  contenere il maggior numero possibile di \textbf{siti di restrizione
  unici}, cioè presenti una sola volta nel loro DNA (siti di taglio che
  presentano il più alto numero possibili di enzimi di restrizione.
  Questi siti devono stare \textbf{al di fuori} delle regioni essenziali
  (replicone e marcatore).
\end{itemize}

Non è essenziale, ma molto utile:

\begin{itemize}
\itemsep1pt\parskip0pt\parsep0pt
\item
  \emph{conoscere l'intera sequenza nucleotidica del vettore}, in modo
  da poter analizzare al computer tutti gli elementi di interesse (siti
  di taglio, prodotti di digestione..);
\item
  che il \emph{sito di clonaggio sia fiancheggiato da sequenze per i}
  \textbf{primer universali} usati nella PCR e nel sequenziamento;
\item
  che il sito di clonaggio sia fiancheggiato da elementi di controllo
  che consentano la trascrizione dell'inserto di DNA in RNA
  (\textbf{vettori di trascrizione}) o la sua espressione in cellule di
  procarioti o di eucarioti (\textbf{vettori di espressione}).
\end{itemize}

\subsubsection{I plasmidi come vettori}\label{i-plasmidi-come-vettori}

I plasmidi sono degli \textbf{elementi genetici extracrosomali} che si
\emph{replicano autonomamente} e che \emph{segregano autonomamente}
rispetto al DNA cromosomale batterico.

Variano da 1 a 200 Kb e sono molto diffusi tra i procarioti. Esempi di
plasmidi batterici naturali sono i \textbf{plasmidi pcolE1} di E.coli
(colina), i \textbf{plasmidi F} di E.coli (coniugazione), i
\textbf{plasmidi Ti} o \textbf{Ri} di Agrobacterium (galla del colletto
nelle dicotiledoni)\ldots{}

I plasmidi possono essere \emph{lineari} o \emph{integrati} nel
cromosoma batterico ma, nella maggior parte dei casi, sono molecole di
DNA \emph{circolari}.

Nell'ospite batterico i plasmidi si presentano come \textbf{molecole
circolari superavvolte} che, durante le manipolazioni sperimentali,
possono rilassarsi o linearizzarsi in seguito a rotture a singolo o a
doppio filamento.

La caratteristica più importante dei plasmidi è quella di essere dei
\textbf{repliconi}, cioè molecole capaci di replicazione autonoma. Un
replicone è costituito da un **origine di replicazione, chiamata ori e
da elementi di controllo.

Si conoscono circa 30 repliconi, ma la maggior parte dei plasmidi di
clonaggio possiede il replicone di \textbf{pMB1} (un plasmide naturale)
che è \emph{identico al replicone di pcolE1}.

I plasmidi si replicano per replicazione \(\theta\) (uni o
bi-direzionale) o per circolo rotante.

Richiedono proteine plasmidiche e/o dell'ospite batterico.

Oltre ad essere essenziale per la replicazione, l'origine di
replicazione controlla:

\begin{itemize}
\itemsep1pt\parskip0pt\parsep0pt
\item
  il numero di copie del plasmide;
\item
  la specificità d'ospite.
\end{itemize}

\paragraph{I marcatori di selezione}\label{i-marcatori-di-selezione}

I plasmidi naturali a volte codificano per \emph{funzioni non
essenziali}, mentre a volte \emph{conferiscono un vantaggio selettivo}
in alcune situazioni. Per esempio possono codificare per le tossine
batteriche o per geni di resistenza agli antibiotici.

In alcuni casi, tuttavia, nessun vantaggio competitivo sembra essere
associato alla presenza di geni di resistenza.

Tutti i vettori di clonaggio includono \emph{almeno} un marcatore di
selezione. Lo scopo essenziale è di \textbf{distinguere} e di
\textbf{selezionare} le molecole ricombinanti. I marcatori di selezione
più utilizzati nei batteri sono i geni di resistenza agli antibiotici.

Per esempio il gene per la \(\beta\)-lattamasi codifica per un enzima
capace di idrolizzare l'anello lattamico degli antibiotici di tipo
penicillinico (es. l'ampicillina). I batteri che contengono un plasmide
con questo gene, quindi,(simboleggiato con Amp o Ap) possono crescere in
terreni di coltura che contengono l'antibiotico ampicillina.

\paragraph{I siti di restrizione
unici}\label{i-siti-di-restrizione-unici}

Per effettuare un clonaggio molecolare è necessario avere sempre
\emph{almeno un sito di riconoscimento} per una \textbf{endonucleasi di
restrizione}.

Il sito di riconoscimento per una endonucleasi di restrizione deve
essere \emph{presente} nel vettore \emph{una sola volta} per non
distruggere l'integrità fisica del plasmide e \emph{non deve essere
presente in regioni cis essenziali} (es. ori o promotori) o \emph{in
geni che codificano per funzioni essenziali} (es. geni di resistenza).

\begin{itemize}
\itemsep1pt\parskip0pt\parsep0pt
\item
  Un esempio di plasmide di clonaggio è il \textbf{PBR322}.
\end{itemize}

Questo è un vettore primitivo con un numero limitato di siti di
restrizione unici (20) distribuiti su tutta la molecola di DNA. È di
piccole dimensioni, 4363 bp.

Contiene \textbf{due geni per la resistenza agli antibiotici},
\textbf{Amp e Tet} (insieme di geni che codificano per enzimi
detossificanti la tetraciclina), al cui interno sono presenti siti di
restrizione unici utilizzabili per il clonaggio.

Per es. la clonazione in PstI inattiva il gene Amp, mentre la clonazione
in BamHI inattiva il gene Tet.

\begin{itemize}
\itemsep1pt\parskip0pt\parsep0pt
\item
  Un vettore plasmidico più evoluto è il \textbf{pUC19}.
\end{itemize}

Questo vettore presenta un sito di clonaggio introdotto con la tecnica
delle clonazione di oligonucleotidi sintetici.

È una serie plasmidica che differisce per la lunghessa e l'orientamento
del \textbf{polylinker}.

Un polylinker o \textbf{multiple cloning site (MCS)}, è un corto
segmento di DNA contenente molti siti di restrizione (circa 21). I siti
di restrizione all'interno di un MCS sono solitamenti unici, ovvero sono
presenti una sola volta all'interno dello stesso plasmide.

Il polylinker è una regione del plasmide nella quale può essere inserito
un DNA esterno.

Il sito di policlonaggio è inserito nel \textbf{gene lacZ} in modo da
tenere la cornice di lettura della proteina; questo permette di
individuare i cloni ricombinanti con un saggio enzimatico
(\(\alpha\)-complementazione) e di esprimere la proteina corrispondente
se l'inserto è inserito in fase.

Presenta resistenza all'Amp.

È un plasmide di espressione.

\paragraph{Evoluzione dei vettori di
clonaggio}\label{evoluzione-dei-vettori-di-clonaggio}

Da questi due vettori di clonaggio sono derivati decine di nuovi altri
vettori.

La tendenza è quella di creare vettori più piccoli e sempre più
funzionali.

Ci sono numerosi vantaggi, infatti, a ridurre la dimensione di un
plasmide:

\begin{enumerate}
\def\labelenumi{\arabic{enumi}.}
\itemsep1pt\parskip0pt\parsep0pt
\item
  è \textbf{più maneggevole}. Per esempio è più difficile danneggiarlo o
  introdurvi interruzzioni a singola elica durante le manipolazioni
  sperimentali;
\item
  è \textbf{più facile estrarlo}. I principali metodi di separazione dei
  plasmidi dal cromosoma batterico si basano sulla denaturazione degli
  acidi nucleici (per es. mediante calore o basi diluite) e sulla loro
  successiva rinaturazione. Mentre i plasmidi, di piccole dimensioni,
  rinaturano rapidamente, il grosso cromosoma batterico non riesce a
  rinaturare velocemente e viene selettivamente eliminato.
\end{enumerate}

La \emph{velocità di rinaturazione plasmidica è} \textbf{inversamente
proporzionale} \emph{alla dimensione}. Quanto più piccoli sono, tanto
più facile é il loro isolamento;

\begin{enumerate}
\def\labelenumi{\arabic{enumi}.}
\setcounter{enumi}{2}
\itemsep1pt\parskip0pt\parsep0pt
\item
  è \textbf{più facile introdurlo dentro un batterio}. I metodi di
  ``trasformazione'' sono essenziali nella tecnologia del DNA
  ricombinante. Esistono varie tecniche, come la \emph{trasformazione
  con CaCl 2 o l'elettroporazione}.
\end{enumerate}

In tutti i casi \emph{l'efficienza di trasformazione è}
\textbf{inversamente proporzionale} \emph{alla dimensione plasmidica}.

Un'ulteriore tendenza è quella di \textbf{sostituire i siti di
restrizione unici con ``Multi Cloning Sites''} sempre più completi.
Questa caratteristica (in genere) facilita il lavoro di clonaggio
permettendo di utilizzare l'enzima di restrizione più conveniente.

Questo problema é particolarmente sentito quando si devono clonare
inserti di grosse dimensioni in cui possono essere presenti numerosi
siti di restrizione.

Numerosi altri vettori più o meno ``specializzati'' sono reperibili per
gli utilizzi più disparati: ``trascrizione in vitro'', inserzioni di
trasposoni, selezione di mutazioni, clonaggio di frammenti amplificati
con PCR, vettori ``shuttle'' che contengono più origini di replicazione
ecc.

\begin{itemize}
\itemsep1pt\parskip0pt\parsep0pt
\item
  Il \textbf{pET system}
\end{itemize}

Un vettore pET consiste in un plasmide batterico ``disegnato'' per
attivare una rapida produzione di una grande quantità di una
qualsivoglia proteina, quando attivata.

Questo plasmide contiene diversi elementi importanti:

\begin{itemize}
\itemsep1pt\parskip0pt\parsep0pt
\item
  un \textbf{gene lacI} che codifica per il repressore proteico di
  \emph{lac};
\item
  un \textbf{promotore di T7}, specifico solo per la T7 RNA polimerasi
  (non per le RNA polimerasi batteriche) e che non è presente ovunque
  nel genoma procariotico;
\item
  un \emph{operatore che può bloccare la trascrizione};
\item
  un \textbf{polylinker};
\item
  un'\textbf{origine di replicazione f1} (così che un plasmide a singolo
  filamento possa essere prodotto quando co-infettato con un fago M13)
\item
  un \textbf{gene di resistenza all'ampicillina};
\item
  un'\textbf{origine di replicazione ColE1}.
\end{itemize}

Per iniziare il processo, un gene a scelta che chiamereno \textbf{YFG},
viene clonato nel sito del polylinker all'interno di un plasmide pET.

Sia il promotore T7 che l'operatore lac sono localizzati all'estremità
5' dell'YFG. Quando lA T7 RNA polimerasi è presente e l'operatore lac
non è represso, la trascrizione di YFG procede rapidamente.

Poichè T7 è un promotore virale, viene trascritto rapidamente e in
abbondanza finchè la T7 RNA polimerasi è presente.

L'espressione di YFP (la proteina codificata dal gene da noi scelto)
cresce rapidamente insieme all'aumento dell'mRNA trascritto da YFG. In
poche ore, YFP diventa uno the componenti maggiormente presenti nella
cellula.

\section{Isolamento e purificazione di DNA e
RNA}\label{isolamento-e-purificazione-di-dna-e-rna}

Il primo passo di qualunque tecnica di biologia molecolare consiste
nell'isolare e purificare gli acidi nucleici.

I dettagli sperimentali variano a seconda degli organismi, del tipo di
acido nucleico che si vuole separare, dal tipo di esperimento che si
deve effettuare, ecc.

In tutti i casi dovremo:

\begin{itemize}
\itemsep1pt\parskip0pt\parsep0pt
\item
  rompere la parete o la membrana cellulare;
\item
  separare gli acidi nucleici dagli altri componenti cellulari;
\item
  separare gli acidi nucleici tra loro (per es. il DNA dal RNA).
\end{itemize}

\subsection{Rottura della parete o della membrana
cellulare}\label{rottura-della-parete-o-della-membrana-cellulare}

Sebbene esistano metodi differenti per estrarre acidi nucleici dalle
cellule, tutti hanno in comune alcune caratteristiche di base.

Per prima cosa è necessario procurarsi il materiale biologico di
partenza, separandolo per centrifugazione dal terreno di coltura, come
nel caso dei batteri, o frazionandolo ed omogenizzandolo, in caso di
tessuti più complessi.

Il secondo passo consiste nella lisi delle cellule, affinchè queste
rilascino i loro componenti cellulari (in funzione di diversi tipi
cellulari vengono utilizzati metodi diversi). Nel caso delle cellule
procariotiche oltre alle membrane cellulari bisogna distruggere la
parete cellulare di peptidoglicano. In genere si utilizzano miscele di
\emph{lisozima}, \emph{detergenti ionici} (tipicamente SDS) ed
\emph{EDTA}.

Il lisozima serve per indebolire la parete di peptidoglicano, l'SDS
provvede a solubilizzare i lipidi delle membrane, mentre l'EDTA è un
agente chelante che sequestra cationi bivalenti necessari per la
stabilizzazione delle membrane e per l'attività di molti enzimi tra cui
la DNasi.

Per cellule animali in genere bastano omogenizzazioni in tamponi
ipoosmolari e/o detergenti ionici e non ionici.

Nel caso di cellule protette da pareti cellulari più resistenti, come i
lieviti o le cellule vegetali, è necessario rompere la parete con metodi
fisici (cicli di congelamento-scongelamento, biglie di vetro,
sonicazione, utilizzo di mortaio e pestello ecc.) oppure ricorrere a
metodi enzimatici capaci di digerire la parete cellulare.

Dopo la rottura delle pareti cellulari e della membrana plasmatica, e la
separazione della frazione solubile da quella insolubile, si otterà una
miscela complessa costituita da varie componenti cellulari come DNA,
RNA, lipidi, mono e polisaccaridi, proteine e sali.

La rottura delle cellule induce quasi sempre una parziale frammentazione
del DNA cromosomale. Il cromosoma batterico, in particolare, che nella
sua forma nativa si trova sempre in forma circolare, verrà ridotto in
frammenti lineari più o meno lunghi in funzione del tipo di trattamento.
In tutti i casi è opportuno utilizzare il metodo più blando possibile
per minimizzare eventuali danni al DNA, particolarmente a quello
genomico.

\subsection{Separazione degli acidi nucleici da altri componenti
cellulari}\label{separazione-degli-acidi-nucleici-da-altri-componenti-cellulari}

Una volta lisata la membrana (ed eventualmente la parete) bisogna
separare gli acidi nucleici dagli altri componenti cellulari.

A questo scopo è pratica comune precipitare gli acidi nucleici con alcol
da soluzioni preventivamente deproteinizzate.

La rimozione delle proteine dal lisato cellulare è particolarmente
importante sia perché tra le proteine sono presenti enzimi capaci di
degradare gli acidi nucleici, sia per la presenza di proteine capaci di
legarsi agli acidi nucleici impedendone la funzione e/o la
purificazione.

Una tecnica comunemente utilizzata per rimuovere il grosso delle
proteine dagli acidi nucleici consiste nel trattare la soluzione acquosa
contenente gli acidi nucleici con solventi organici immiscibili (quasi)
in acqua, tipicamente fenolo/cloroformio.

Emulsionando i due componenti si formano due fasi distinte,
all'interfaccia delle quali si denaturano e si stratificano la maggior
parte delle proteine. Poiché l'acqua ed il fenolo sono solo parzialmente
immiscibili, il fenolo deve essere equilibrato con una soluzione
tampone. Se si usa un fenolo equilibrato con un tampone a pH neutro o
alcalino, sia il DNA che llRNA si ripartiranno nella fase acquosa
superiore.

Se si effetua un'estrazione con un fenolo equilibrato con un tampone
acido o con acqua (visto che il fenolo è naturalmente acido), il solo
RNA ripartirà nella fase acquosa superiore mentre il DNA verrà
trattenuto nella fase organica inferiore.

\{img/estrazione-con-fenolo\}

Poiché il fenolo è parzialmente solubile in acqua, alcune sue tracce
possono rimanere in soluzione e inibire successivi trattamenti
enzimatici (denaturando gli enzimi!). È importante, quindi, rimuovere
ogni traccia di fenolo estraendolo con cloroformio e, in qualche caso
estraendo anche eventuali tracce di cloroformio con etere che, infine
verrà eliminato per evaporazione.

\subsection{Separazione degli acidi nucleici tra
loro}\label{separazione-degli-acidi-nucleici-tra-loro}

Dopo aver rotto l'involucro cellulare e separato gli acidi nucleici
dagli altri componenti solubili (proteine, carboidrati ecc.) avremo una
miscela costituita prevalentemente da DNA genomico, DNA plasmidico ed
RNA (rRNA, tRNA, mRNA).

\subsubsection{Separazione del DNA plasmidico da quello
genomico}\label{separazione-del-dna-plasmidico-da-quello-genomico}

Per separare il DNA si compie un trattamento con Rasi seguito da una
precipitazione selettiva con isopropanolo (a temperatura ambiente).

In questo modo si ottiene una miscela contenente DNA genomico e
plasmidico.

Per separare i 2 DNA si sfruttano le differenze di peso molecolare
mediante un ciclo di denaturazione e rinaturazione.

Per separare il DNA plasmidico da quello cromosomale si sfruttano le
differenze chimico-fisiche tra questi due tipi di macromolecole.

In particolare si utilizza la capacità che hanno i plasmidi, specie
quelli di piccole dimensioni, di riassumere una conformazione
covalentemente chiusa, in seguito a una moderata denaturazione.

In condizioni normali i plasmidi si trovano prevalentemente in forma
\textbf{circolare superavvolta (supercoiled)} e, in misura minore, in
forma circolare rilassata, caratteristica delle molecole che hanno
perduto il superavvolgimento a causa di un interruzione in uno dei due
filamenti (nick). Possono trovarsi anche in forma \textbf{lineare},
caratteristica delle molecole con una interruzione in entrambi i
filamenti.

Sottoponendo la soluzione contenente gli acidi nucleici a condizioni
moderatamente denaturanti (per esempio in condizioni alcaline intorno a
pH 12), il DNA cromosomale si frammenta in lunghi frammenti lineari,
denaturandosi irreversibilmente, mentre i plasmidi, pur ridotti in
condizioni denaturate, manterranno i due filamenti fisicamente
concatenati (\textbf{denaturazione alcalina}).

Dopo aver riportato la soluzione in condizioni normali (pH neutro)
neutralizzando con acido, mentre i frammenti cromosomali rimarranno
prevalentemente denaturati (si aggregano in un reticolo), i plasmidi
saranno in grado di rinaturarsi velocemente riassumendo una forma
circolare covalentemente chiusa.

Per facilitare la separazione tra DNA plasmidico e cromosomale, si usano
spesso tamponi contenenti \textbf{SDS} e ioni **K\(^+\) che, legandosi
alle proteine cromosomali, contribuiscono a creare un reticolo
insolubile formato da un complesso nucleo-proteico cromosomale inglobato
in micelle di SDS-K, che precipita.

\subsubsection{Separazione del'mRNA dai restanti
RNA}\label{separazione-delmrna-dai-restanti-rna}

Per separare l'RNA, invece, si compie un trattamento con Dnasi seguito
da una precipitazione selettiva con LiCi e da un trattamento con
guanidina tiocinato o con inibitori delle RNasi.

A questo punto, per separare l'mRNA dal resto dell'RNA si compie una
purificazione per affinità su cellulosa.

\subsubsection{Precipitazione con alcol}\label{precipitazione-con-alcol}

Dopo l'estrazione fenolica e, in linea generale, per purificare
ulteriormente o per concentrare gli acidi nucleici, si ricorre alla
precipitazione con alcol.

In soluzioni alcoliche, infatti, gli acidi nucleici precipitano, insieme
a parte dei sali, e possono essere efficacemente separati da altri
componenti cellulari più solubili.

Anche la maggior parte delle proteine sono insolubili in alcol e co-
precipiterebbero in larga misura con gli acidi nucleici se non fossero
preventivamente rimosse.

Generalmente si usano 2 volumi di etanolo a (15' a -20°C) o 0,7 volumi
di isopranolo (5' a temperatura ambiente). \textbf{???}

Dopo la precipitazione, che avviene in presenza di cationi monovalenti
(Na\(^+\)) con funzione di carrier, la soluzione viene centrifugata,
seccata e risospesa in adeguati tamponi alla concentrazione desiderata.

L'alcool etilico determina modificazioni strutturali degli acidi
nucleici che ne inducono l'aggregazione e quindi la precipitazione.

Gli acidi nucleici possono essere completamente ed efficacemente
separati anche mediante centrifugazioni differenziali.

La purificazione del DNA su gradiente isopicnico di saccarosio in
bromuro di etidio è stata molti anni il metodo di purificazione più
usato. Nonostante questo metodo fornisca DNA di grande purezza, a causa
della sua laboriosità e potenziale tossicità, è stato in gran parte
soppiantato da metodi basati su resine a scambio ionico.

\subsubsection{Kit di purificazione}\label{kit-di-purificazione}

In alternativa ai classici metodi di purificazione degli acidi nucleici
esistono in commercio numerosi kit di isolamento e purificazione degli
acidi nucleici che rappresentano un'alternativa sempre più utilizzata.

Esistono molti prodotti commerciali che garantiscono facilità d'uso,
riproducibilità ed elevato livello di purificazione.

Si basano essenzialmente sull'utilizzo di:

\begin{itemize}
\itemsep1pt\parskip0pt\parsep0pt
\item
  resine a scambio ionico (scambiatori anionici come la DEAE cellulosa);
\item
  matrici silicee;
\item
  gel filtration;
\item
  ultrafiltrazione.
\end{itemize}

Il principio sul quale si basa la purificazione degli acidi nucleici con
matrice silicea è semplice: gli acidi nucleici si legano alle particelle
della matrice in presenza di sali caotropici. Dopo avere lavato la
matrice silicea, gli acidi nucleici vengono eluiti in tamponi a basso
contenuto di sali e sono pronti per le successive reazioni (clonaggio,
digestione con enzimi di restrizione, blotting, sequenziamento manuale
ed automatico, amplificazione, trascrizione inversa ecc.).

\section{Tecniche di ibridazione del
DNA}\label{tecniche-di-ibridazione-del-dna}

Esistono diverse tecniche di ibridazione, tra cui:

\begin{itemize}
\itemsep1pt\parskip0pt\parsep0pt
\item
  la \textbf{southern blotting};
\item
  la \textbf{northern blotting};
\item
  la \textbf{western blotting}.
\end{itemize}

\subsection{La Southern blot}\label{la-southern-blot}

La Southern blot è una tecnica usata in biologia molecolare per
\emph{rivelare la presenza di specifiche sequenze di DNA} in una miscela
complessa.

Procedimento:

\begin{itemize}
\itemsep1pt\parskip0pt\parsep0pt
\item
  Si estrae il DNA genomico da una cellula e lo si tratta con enzimi di
  restrizione;
\item
  successivamente il campione eterogeneo di DNA viene sottoposto ad
  \textbf{elettroforesi} su gel d'agarosio o di poliacrilammide. Nel gel
  sarà possibile osservare uno \textbf{smear}, ossia una striscia
  continua; non si vedranno bande nette perché il DNA genomico digerito
  con l'enzima di restrizione ha tantissimi punti di taglio, quindi sul
  gel si troveranno tantissimi frammenti che migreranno con velocità
  diverse in base al diverso peso molecolare;
\item
  il gel viene coperto da un \textbf{foglio di nitrocellulosa o nylon} a
  \emph{carica positiva} e sopra di questo viene posta una pila di
  \textbf{fogli assorbenti}. Per \textbf{capillarità} la soluzione
  tenderà ad attraversare il gel e il foglio di nitrocellulosa risalendo
  nei fogli assorbenti. I sali trascinano i segmenti di DNA
  perfettamente in verticale, depositandoli sullo strato di
  nitrocellulosa con il quale i segmenti instaurano legami
  elettrostatici (dovute alle cariche negative dei gruppi fosfato del
  DNA che si legano alle positive della membrana). Da notare che in
  questo passaggio il DNA non sale grazie a una forza elettrica come
  nella prima elettroforesi ma solo per capillarità;
\item
  il foglio di nitrocellulosa viene quindi immerso in una soluzione
  contenente una \textbf{sonda marcata} in vario modo (fluorescenza,
  radioattività, ecc..) che ibridizza con sequenze di DNA complementari
  presenti sul foglio, identificandole. La sonda viene creata con
  tecniche di amplificazione del DNA. L'ibridizzazione della sonda con
  il DNA può avvenire con diversi gradi di ``stringenza'' a seconda
  delle finalità dell'esperimento, consentendo una ibridizzazione con
  specificità anche inferiore al 100\%.
\item
  a seguito di lavaggio della nitrocellulosa per eliminare le sonde non
  ibridate, si fa una autoradiografia o chemioluminescenza che metta in
  evidenza dove la sonda ha legato il DNA genomico.
\end{itemize}

La \textbf{``stringenza''} determina la selettività dell'ibridazione e
dipende dalla temperatura e dalla concentrazione salina.

Con una stringenza minore si identificano molecole non uguali
(complementari) al campione.

Il lavaggio a differenti temperature (stringenza) delle membrane
ibridate in Southern blot produce risultati differenti. Il lavagaiio ad
\textbf{alte temperature} rimuove tutte le molecole della sonda ad
eccezione di quelle legate con maggiore forza (che identificano sul
filtro le sequenze più simili o uguali a quella della sonda). I lavaggi
a \textbf{temperature inferiori} possono permette di rivelare sulla
membrana le sequenze che sono simili, ma non identiche, alla sequenza
della sonda. L'effettiva temperatura a cui si effettuano i lavaggi
dipendenrà dalla lunghezza della sonda e dalla probabilità che essa
trovi sulla membrana una sequenza perfettamente complementare.

\subsection{Il Northern blot}\label{il-northern-blot}

Il northern blotting è una tecnica che permette di \emph{visualizzare ed
identificare l'RNA purificato} da un campione, in particolare per
studiare l'espressione genica.

Tecnicamente, è simile al Southern Blotting, con la differenza
sostanziale che l'RNA tende a formare strutture secondarie stabili in
soluzione; per fare in modo che la mobilità elettroforetica sia solo
dipendente dalla lunghezza del frammento, l'RNA deve essere
preventivamente denaturato (solitamente esponendolo ad alte
temperature). Inoltre, la corsa elettroforetica deve essere eseguita in
presenza di agenti denaturanti, solitamente formaldeide e formammide.

\subsection{Il western blot}\label{il-western-blot}

Il western blot è una tecnica biochimica che permette di
\emph{identificare una determinata proteina in una miscela di proteine},
mediante il riconoscimento da parte di \emph{anticorpi specifici}.

In generale, per facilitare il riconoscimento, la miscela di proteine
viene prima separata \emph{in base alle loro dimensioni} (o peso
molecolare) utilizzando un gel di poliacrilammide; successivamente le
proteine vengono trasferite su di un supporto, che comunemente è una
membrana di nitrocellulosa, e quindi si procede al riconoscimento vero e
proprio della proteina mediante l'utilizzo di un \textbf{anticorpo
specifico}.

\subsection{Ibridazione in situ fluorescente
(FISH)}\label{ibridazione-in-situ-fluorescente-fish}

La FISH è una tecnica citogenetica che può essere utilizzata per
rilevare e localizzare la presenza o l'assenza di specifiche sequenze di
DNA su preparati fissati di cromosomi, nuclei interfasici e sezioni di
tessuto ottenuti da qualsiasi tipo di materiale biologico, sia esso
fresco, conservato o paraffinato.

Essa utilizza delle sonde a fluorescenza che si legano in modo
estremamente selettivo ad alcune specifiche regioni del cromosoma. Per
individuare il sito di legame tra sonda e cromosoma si utilizzano
tecniche di microscopia a fluorescenza.

\textbf{Processo:} innanzi tutto bisogna preparare la sonda, che deve
essere abbastanza lunga per ibridare esattamente il suo obiettivo (e non
un'altra sequenza simile del genoma), ma non deve essere tanto grande da
impedire il processo. La sonda può essere marcata con metodi diretti o
indiretti.

Successivamente si produce un preparato cromosomico. I cromosomi sono
attaccati saldamente al substrato, di solito di vetro. Dopo la
preparazione si applica la sonda al DNA del cromosoma e si inizia
l'ibridazione. In molti passaggi di lavaggio tutte le sonde non ibridate
o parzialmente ibridate vengono rimosse.

Se l'amplificazione del segnale è necessaria a superare il limite della
sensibilità del microscopio (che dipende da molti fattori come
l'efficienza della sonda, il tipo di sonda e la tinta fluorescente), gli
anticorpi fluorescenti o la streptavidina si legano alle molecole
marcate, per amplificarne la fluorescenza.

Infine, il campione è messo in un composto non-imbiancante e osservato
al microscopio a fluorescenza.

La FISH rappresenta un indispensabile complemento della citogenetica
tradizionale in quanto ha un maggiore potere di risoluzione. Consente
indatti di \textbf{caratterizzare anomalie cromosomiche di numero e di
struttura} non definibili attraverso le tecniche di citogenetica
classica e di identificare \textbf{riarrangiamenti criptici}, non
visibile neppure dopo bandeggio ad alta risoluzione.

\section{La produzione di librerie}\label{la-produzione-di-librerie}

Una \textbf{genoteca} è una collezione di geni o di frammenti genici o
genomici, ciascuno clonato in un vettore di clonaggio. Un aspetto
importante di una libreria genica è la sua rappresentatività: infatti si
dice rappresentativa quando contiene almeno una copia di tutte le
possibili sequenze.

Quando un DNA genomico viene estratto dalle cellule di un organismo,
tagliato con un enzima di restrizione e la popolazione dei frammenti di
DNA ottenuti viene clonata in vari vettori, si ottiene una collezione di
cloni, contenente almeno una copia di tutte le sequenze di DNA presenti
nel genoma. Tale collezione è definita libreria genomica o genoteca.

Le librerie di DNA sono \emph{collezioni di cloni ricombinanti ottenuti
da una popolazione complessa di frammenti di DNA genomico o di}
\textbf{cDNA (DNA complementare)}.

Ogni clone ricombinante ospita un frammento di DNA diverso. Una buona
libreria contiene \textbf{tutta l'informazione} presente nel DNA
genomico o cDNA di partenza.

Le librerie possono essere:

\begin{itemize}
\itemsep1pt\parskip0pt\parsep0pt
\item
  \textbf{genomiche}, se sono una collezione di cloni che rappresentano
  l'intero genoma di un organismo;
\item
  \textbf{di cDNA}, se sono una collezione di cloni che rappresentano i
  geni che vengono espressi in un organismo.
\end{itemize}

Le \emph{librerie di DNA genomico} sono utili quando si vuole
\textbf{determinare l'intera sequenza di un genoma}. Inoltre i cloni di
DNA genomico contengono le sequenze che regolano l'espressione dei geni
(promotore, siti di potenziamento o silenziamento della trascrizione).
Queste librerie servono per studiare la struttura e la regolazione
genica.

Nelle \emph{librerie di cDNA} l'RNA messaggero viene copiato in DNA
complementare (cDNA) dall'enzima \textbf{trascrittasi inversa}. Le
librerie di cDNA sono utili quando si vogliono \textbf{studiare le
proteine prodotte da un gene}: la sequenza degli aminoacidi può essere
dedotta rapidamente a partire dalla sequenza dei nucleotidi nel cDNA
corrispondente.

In tali genoteche è clonato quindi il DNA privo delle regioni introniche
e permettono quindi di analizzare la porzione codificante dei vari geni
analizzati, nonché di diminuire notevolmente le dimensioni dei
frammenti.

\textbf{Concetti importanti}:

\begin{itemize}
\itemsep1pt\parskip0pt\parsep0pt
\item
  \textbf{rappresentatività}: una libreria si dice rappresentativa
  quando contiene, in \textbf{almeno una copia}, tutte le possibili
  sequenze;
\item
  \textbf{ridondanza}: una libreria si dice ridondante quando contiene
  molte copie di alcune sequenze, complicando la ricerca e selezione
  (screening) della sequenza che vogliamo analizzare.
\end{itemize}

Poichè è virtualmente impossibile essere sicuri di avere tutti i
possibili cloni in una libreria genica, possiamo solo aumentare la
probabilità di avere un dato clone aumentando il numero totale di cloni
indipendenti (N). Purtroppo in questo modo aumenteremo anche la
\textbf{ridondanza} della nostra libreria.

La seguente formula, proposta da Clarke e Carbon, può stimare con
ragionevole approssimazione il \textbf{numero di cloni indipendneti} che
la nostra libreria deve contenere per avere una probabilità P (0,99) di
contenere il clone di nostro interesse.

Il valore di questo rapporto è dato dalla relazione tra
\textbf{frequenza} e \textbf{probabilità}: la probabilità \textbf{(P)}
di identificare un clone ricombinante in una libreria è una funzione
della sua frequenza \textbf{(f)}.

\textbf{N = ln(1-P)/ln(1-f)}

Dove:

\begin{itemize}
\itemsep1pt\parskip0pt\parsep0pt
\item
  \textbf{N} rappresenta il numero di cloni analizzati;
\item
  \textbf{f} è la frazione del genoma rappresentata da un clone medio.
  \emph{f} viene calcolata dividendo la media delle dimensioni degli
  inserti (dipende dal tipo di vettore usato) per la dimensione totale
  del genoma).
\end{itemize}

Esistono diversi tipi di vettori di clonaggio, ciascuno con vantaggi e
svantaggi.

La principale considerazione da fare è relativa alle dimensioni
dell'inserto di DNA che ogni vettore può accettare.

\{img/80\_dimensione-frammento-vettore\}

Se il vettore lambda può contenere un frammento di 17 kbp, N sarà dato
da:

N = ln(1-0,99) / ln(1-1,7x10\(^4\) bp) = 8,22 x 10\(^5\) numeor di
frammenti necessari nella libreria

\{img/81\_dimensioni-ottimali-libreria\}

\subsection{Costruzione di una libreria
genomica}\label{costruzione-di-una-libreria-genomica}

La costruzione di una libreria genomica si suddivide in 4 fasi
principali:

\begin{enumerate}
\def\labelenumi{\arabic{enumi}.}
\itemsep1pt\parskip0pt\parsep0pt
\item
  \textbf{frammentazione del genoma} tramite digestione con endonucleasi
  di restrizione, o tramite frammentazione meccanica;
\item
  \textbf{ligazione con il vettore}, ligazione con estremità coesive o
  piatte;
\item
  \textbf{introduzione nella cellula ospite}, trasformazione con DNA
  plasmidico ricombinante, tresfezione con DNA fagico, ecc\ldots{};
\item
  \textbf{screening dei cloni di interesse} tramite ibridazione, PCR,
  ecc\ldots{}
\end{enumerate}

Il DNA genomico viene frammentato in pezzi di dimensioni clonabili
mediante:

\begin{itemize}
\itemsep1pt\parskip0pt\parsep0pt
\item
  \textbf{digestione parziale} (genera frammenti sovrapposti);
\item
  \textbf{digestione totale} (non genera frammenti sovrapposti);
\item
  \textbf{rottura meccanica}.
\end{itemize}

I frammenti vengono clonati nel vettore.

Il prodotto della ligazione viene utilizzato per la trasformazione di
batteri o lieviti (a seconda del vettore utilizzato).

Si determina la \emph{qualità} della libreria valutando la percentuale
dei cloni che contengono l'inserto e la dimensione degli stessi.

\subsection{Costruzione di una libreria di
cDNA}\label{costruzione-di-una-libreria-di-cdna}

Una buona libreria di cDNA deve possedere tutti gli mRNA presenti nel
tessuto di partenza, compresi quelli più rari.

Per raggiungere questo scopo, è necessario partire da almeno 1-5
\(\mu\)g di poly(A)\(^+\) -RNA.

I vantaggi della libreria di cDNA sono che:

\begin{itemize}
\itemsep1pt\parskip0pt\parsep0pt
\item
  vengono valutate solo le porzioni codificanti del genoma;
\item
  nella sequenza genica non vi sono gli introni (eucarioti).
\end{itemize}

La costruzione di una libreria di cDNA si divide principalmente in
queste fasi:

\begin{itemize}
\itemsep1pt\parskip0pt\parsep0pt
\item
  \textbf{estrazione dell'mRNA};
\item
  \textbf{sintesi del cDNA} (retrotrascrizione) a partire da mRNA;
\item
  \textbf{sintesi del secondo frammento di DNA};
\item
  \textbf{screening dei cloni} di interesse.
\end{itemize}

I frammenti vengono CLONATI nel vettore (plasmide o fago).

Il prodotto della ligazione viene utilizzato per la trasformazione di
batteri.

Si determina la qualità della libreria valutando la percentuale dei
cloni che contengono l'inserto e la dimensione degli stessi.

\subsubsection{Lo screening delle
librerie}\label{lo-screening-delle-librerie}

Ci sono tre possibili metodi per identificare un clone di DNA genomico e
di cDNA:

\begin{enumerate}
\def\labelenumi{\arabic{enumi}.}
\itemsep1pt\parskip0pt\parsep0pt
\item
  \textbf{ibridazione con acidi nucleici};
\item
  \textbf{reattività immunologica di specifici antigeni};
\item
  \textbf{produzione di molecole biologicamente attive} (quando esistono
  saggi funzionali per la proteina di interesse o il substrato su cui
  agisce).
\end{enumerate}

Le Sonde che riconoscono il DNA possono essere formate da:

\begin{itemize}
\itemsep1pt\parskip0pt\parsep0pt
\item
  \textbf{acidi nucleici} (DNA o RNA a singolo filamento);
\item
  \textbf{oligonucleotidi di sintesi} (se è nota la sequenza del
  prodotto di espressione).
\end{itemize}

Le sonde sono marcate con un \textbf{isotopo radioattivo}
(P\(^3\)\(^2\), P\(^3\)\(^3\),H\(^3\) e S\(^3\)\(^5\)) o coniugate con
un enzima che permette un saggio colorimetrico.

Le sonde che riconoscono le proteine (solo per librerie di cDNA
costruite in vettori di espressione) possono essere formate da
\textbf{anticorpi} (monoclonali o policlonali).

Per identificare i cloni cui si è legato l'anticorpo specifico(primario)
si può usare:

\begin{itemize}
\itemsep1pt\parskip0pt\parsep0pt
\item
  un \textbf{anticorpo secondario} marcato con radioattivo (es.
  I\(^1\)\(^2\)\(^5\));
\item
  la \textbf{proteina A marcata con radioattivo}. La proteina A
  purificata da Staphylococcus aureus ha elevata affinità per le catene
  pesanti delle immunoglobuline;
\item
  un anticorpo secondario coniugato con un enzima che permette un saggio
  colorimetrico (es. fosfatasi o perossidasi).
\end{itemize}

Fasi per lo screening di una genoteca cDNA:

\begin{enumerate}
\def\labelenumi{\arabic{enumi}.}
\itemsep1pt\parskip0pt\parsep0pt
\item
  la genoteca è divisa in 20-30 piastre di grande diametro, ciascuna con
  diverse centinaia di cloni da analizzare;
\item
  si fanno crescere i batteri finché le colonie/placche sono di
  dimensioni accettabili;
\item
  si applica un filtro su ogni piastra. Per le proteine si utilizza un
  filtro di nitrocellulosa, mentre per gli acidi nucleici si utilizza un
  filtro di nylon carico positivamente;
\item
  il filtro è rimosso ed i batteri sono lisati, mentre le proteine o gli
  acidi nucleici restano legati al filtro;
\item
  ogni filtro è incubato con una sonda specifica per il cDNA di
  interesse;
\item
  la sonda non legata è allontanata mediante lavaggi;
\item
  la sonda legata in modo specifico ai cloni di cDNA viene rivelata con
  opportune tecniche (es. autoradiografia).
\end{enumerate}

\{img/82\_screening)

\subsubsection{Il sequenziamento del
DNA}\label{il-sequenziamento-del-dna}

Il sequenziamento del DNA è una tecnica che consente di determinare in
modo preciso l'ordine dei nucleotidi in un tratto di DNA.

Per sequenziare il DNA esistono due metodi diversi:

\begin{enumerate}
\def\labelenumi{\arabic{enumi}.}
\itemsep1pt\parskip0pt\parsep0pt
\item
  \textbf{frammentazione chimica}, metodo Maxam e Gilbert;
\item
  \textbf{sintesi enzimatica} o terminazione della catena, metodo di
  Sanger.
\end{enumerate}

Questo due tecniche presentano degli aspetti comuni:

\begin{itemize}
\itemsep1pt\parskip0pt\parsep0pt
\item
  generano set di filamenti singoli marcati di tutte le possibili
  lunghezze (da 1 a n nucleotidi);
\item
  il set completo di frammenti viene suddiviso in 4 collezioni separate;
\item
  ciascuna collezione contiene i frammenti che iniziano da una estremità
  della catena fino ad una specifica delle 4 basi;
\item
  i frammenti vengono separati per elettroforesi, correndo le 4
  collezioni in corsie parallele;
\item
  le collezioni possono essere visualizzare per autoradiografia.
\end{itemize}

\paragraph{Metodo di Maxam e Gilbert}\label{metodo-di-maxam-e-gilbert}

Questo metodo è basato sulla degradazione chimica di frammenti di DNA.

Questo metodo è poco utilizzato perché impiega composti chimici dannosi
alla salute e perché non adattabile al sequenziamento automatico.

In questo caso, una molecola di DNA a doppio filamento viene marcata con
\(^3\)\(^2\)P all'estremità 5' o 3'.

Successivamente si denaturano e separano i 2 filamenti.

Il DNA a singola elica viene suddiviso in 4 campioni, ognuno dei quali
viene trattato con un reagente chimico che demolisce una o due delle 4
basi del DNA:

\begin{itemize}
\itemsep1pt\parskip0pt\parsep0pt
\item
  per la base G i reagenti sono il DMA e la piperidina;
\item
  per le basi G+A i reagenti sono il DMA, la piperidina e l'acido
  formico;
\item
  per le basi C+T i reagenti sono l'idrazina e la piperidina;
\item
  per la base C i reagenti sono l'idrazina e la piperidina in NaCl 1,5M.
\end{itemize}

Sulla molecola marcata, si eseguono dunque quattro reazioni separate,
che portano alla rottura del DNA in corrispondenza di una base specifica
modificata o in corrispondenza di gruppi di basi modificate.

Le reazioni vengono condotte in difetto del reagente, in modo che ogni
molecola di DNA venga tagliata in media una volta sola.

Le reazioni sono controllate in modo da avere una frammentazione
parziale: statisticamente tutte le possibili basi saranno degradate
producendo una serie di frammenti la cui lunghezza dipenderà dalla
distanza tra l'estremità marcata e il sito di taglio.

I frammenti prodotti dalle 4 reazioni vengono poi separati con
elettroforesi ad alta risoluzione su gel di poliacrilamide, capace di
separare (risolvere) due molecole di DNA che differiscono per lunghezza
anche di un solo nucleotide.

\paragraph{Il metodo di Sanger}\label{il-metodo-di-sanger}

Il metodo Sanger è un metodo cosiddetto enzimatico, poiché richiede
l'utilizzo di un enzima.

Il principio sul quale questa tecnica si basa è l'utilizzo di nucleotidi
modificati (\textbf{dideossitrifosfato, ddNTPs}) per interrompere la
reazione di sintesi in posizioni specifiche.

I nucleotidi dideossitrifosfato sono molecole artificiali corrispondenti
ai nucleotidi naturali, ma si differenziano per l'\emph{assenza del
gruppo idrossilico sul carbonio 2' e 3'} della molecola.

I dideossinucleotidi, a causa della loro conformazione, impediscono che
un altro nucleotide si leghi ad essi, in quanto non si possono formare
legami fosfodiesterici.

Il protocollo classico richiede:

\begin{itemize}
\itemsep1pt\parskip0pt\parsep0pt
\item
  un templato di DNA a singolo filamento;
\item
  un primer per iniziare la reazione di polimerizzazione;
\item
  una DNA polimerasi;
\item
  deossinucleotidi;
\item
  dideossinucleotidi per terminare la reazione di polimerizzazione.
\end{itemize}

I nucleotidi modificati (ddNTPs) o il primer devono essere marcati
(radioattivamente o per fluorescenza) in modo da poter visualizzare le
bande dei frammenti di DNA neosintetizzato dopo aver effettuato
l'elettroforesi.

Il campione di DNA da sequenziare viene diviso in quattro reazioni
separate, ognuna delle quali contiene la DNA polimerasi e tutti e 4 i
deossiribonucleotidi (dATP, dCTP, dGTP, dTTP).

Ad ognuna di queste reazioni viene poi aggiunto solo uno dei 4
nucleotidi dideossi (ddATP, ddCTP, ddGTP, ddTTP) in quantità
stechiometricamente inferiore per permettere una elongazione del
filamento sufficiente per l'analisi. L'incorporazione di un
dideossinucleotide lungo il filamento di DNA in estensione ne causa la
terminazione prima del raggiungimento della fine della sequenza di DNA
stampo; questo dà origine ad una serie di \emph{frammenti di DNA di
lunghezza diversa interrotti in corrispondenza dell'incorporazione del
dideossinucleotide}, che avviene casualmente quando esso è utilizzato
dalla polimerasi in luogo di un nucleotide deossi.

I frammenti generati da queste reazioni vengono poi fatti correre su gel
di poliacrilammide-urea che permette la separazione dei vari frammenti
con una risoluzione di un nucleotide.

Ognuna delle 4 reazioni è corsa su pozzetti vicini, dopodiché le bande
sono visualizzate su lastra autoradiografica o sotto luce UV, e la
sequenza viene letta direttamente sulla lastra o sul gel, a seconda del
tipo di marcatura dei nucleotidi dideossi.

Per la lettura dei frammenti, per prima cosa si individua la banda a
peso molecolare più basso (corrisponde al frammento più corto) e si
prende nota della corsia in cui si trova. Successivamente si individua
la banda più mobile dopo la prima (questa avrà \textbf{un} nucleotide in
più della prima banda).

\paragraph{Sequenziamento automatico}\label{sequenziamento-automatico}

Nel caso del sequenziamento automatico il principio è quello del metodo
enzimatico, ma la reazione viene eseguita in un unico tubo che contiene,
oltre al frammento di DNA che si vuole sequenziare:

\begin{itemize}
\itemsep1pt\parskip0pt\parsep0pt
\item
  l'enzima DNA polimerasi;
\item
  l'oligonucleotide sintetico che serve come punto d'appoggio per la DNA
  polimerasi;
\item
  i 4 nucleotidi;
\item
  i ddNTP, marcati ciascuno con un \emph{composto fluorescente
  differente}.
\end{itemize}

All'interno del sequenziatore automatico la reazione passa attraverso un
capillare che contiene una matrice con la stessa funzione di separazione
dei frammenti normalmente svolta dal gel di poliacrilamide.

In questo caso la separazione è simile a quella della cromatografia su
colonna: i frammenti di diverse dimensioni si separano e sono rivelati,
uno dopo l'altro, da luce laser di lunghezza d'onda tale da eccitare la
fluorescenza dei ddNTP.

Le misure del rivelatore permettono di riconoscere i diversi picchi di
fluorescenza, ciascuno con emissione a una determinata lunghezza d'onda,
generati dai quattro diversi ddNTPs.

In questo modo la sequenza nucleotidica del frammento di interesse può
essere ricostruita da un computer. I sequenziatori automatici possono
lavorare su frammenti di 200-1000 nucleotidi. Il sequenziamento di un
frammento di DNA con 500 basi azotate dura poco più di mezz'ora.

\begin{itemize}
\itemsep1pt\parskip0pt\parsep0pt
\item
  \textbf{Il colorante BigDye}
\end{itemize}

Contengono un sito donatore fluorescinato legato ad uno dei 4 coloranti
accettori diclororodamine (dRhodamine).

La massima eccitazione di ogni colorante è quella della fluorescina
(donatore) e lo spettro di emissione è quello della rodamina
(accettore).

Hanno dei vantaggi rispetto alle rodamine usate in precedenza. Danno
picchi più ristretti e più alti con riduzione delle sovrapposizioni
degli spettri di emissione e quindi meno rumore di fondo nella lettura.

Le emissioni fluorescenti vengono captate da un rilevatore e le
informazioni vengono integrate e trasformate in picchi di colore
diverso, con aree proporzionali all'intensità di emissione.

\end{document}
