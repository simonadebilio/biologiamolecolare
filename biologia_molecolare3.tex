\documentclass[]{article}
\usepackage{lmodern}
\usepackage{amssymb,amsmath}
\usepackage{ifxetex,ifluatex}
\usepackage{fixltx2e} % provides \textsubscript
\ifnum 0\ifxetex 1\fi\ifluatex 1\fi=0 % if pdftex
  \usepackage[T1]{fontenc}
  \usepackage[utf8]{inputenc}
\else % if luatex or xelatex
  \ifxetex
    \usepackage{mathspec}
    \usepackage{xltxtra,xunicode}
  \else
    \usepackage{fontspec}
  \fi
  \defaultfontfeatures{Mapping=tex-text,Scale=MatchLowercase}
  \newcommand{\euro}{€}
\fi
% use upquote if available, for straight quotes in verbatim environments
\IfFileExists{upquote.sty}{\usepackage{upquote}}{}
% use microtype if available
\IfFileExists{microtype.sty}{%
\usepackage{microtype}
\UseMicrotypeSet[protrusion]{basicmath} % disable protrusion for tt fonts
}{}
\ifxetex
  \usepackage[setpagesize=false, % page size defined by xetex
              unicode=false, % unicode breaks when used with xetex
              xetex]{hyperref}
\else
  \usepackage[unicode=true]{hyperref}
\fi
\hypersetup{breaklinks=true,
            bookmarks=true,
            pdfauthor={},
            pdftitle={},
            colorlinks=true,
            citecolor=blue,
            urlcolor=blue,
            linkcolor=magenta,
            pdfborder={0 0 0}}
\urlstyle{same}  % don't use monospace font for urls
\setlength{\parindent}{0pt}
\setlength{\parskip}{6pt plus 2pt minus 1pt}
\setlength{\emergencystretch}{3em}  % prevent overfull lines
\setcounter{secnumdepth}{0}

\date{}

\begin{document}

\section{Reazione a catena della polimerasi
(PCR)}\label{reazione-a-catena-della-polimerasi-pcr}

La tecnica della PCR fu ideata nel 1983 da Kar B. Mullis. Questa tenica
permette di selezionare e amplificare, da una preparazione eterogenea di
DNA, un particolare tratto che ci interessa, producendone un grandissimo
numero di copie (può amplificare un tratto di DNA per più di un milione
di volte).

Il processo di PCR prevede un certo numero di cicli (30-35) ed ogbu
ciclo cinsiste di 3 passaggi:

\begin{enumerate}
\def\labelenumi{\arabic{enumi}.}
\itemsep1pt\parskip0pt\parsep0pt
\item
  \textbf{denaturazione} del DNA, avviene alla temperatura di 95°. Il
  DNA stampo viene denaturato, i due filamenti si separano;
\item
  \textbf{appaiamento}, avviene alla temperatura di 55° circa. In questo
  passaggio i primer si appaiano con il DNA stampo.
\item
  \textbf{sintesi}, avviene alla temperatura di 72°.
\end{enumerate}

La PCR consiste sostanzialmente in una replicazione in viste di un
frammento di DNA ottenuta mediante una DNA polimerasi e 2
oligonucleotidi lunghi 20-30 nt che fungono da inneschi
(\textbf{primer}) per la replicazione e che sono responsabili della
specificità dell'amplificazione.

Gli elementi necessari per la PCR sono:

\begin{itemize}
\itemsep1pt\parskip0pt\parsep0pt
\item
  \textbf{2 oligonucleotidi} complementari alle 2 regioni che si trovano
  sui filamenti opposti del DNA stampo, ai lati della regione che si
  vuole amplificare;
\item
  un \textbf{DNA stampo} che contenga la regione da amplificare;
\item
  una \textbf{polimerasi termostabile} (non deve denaturarsi se portata
  a 95°);
\item
  i 4 deossinucleotidi trifosfati.
\end{itemize}

Il DNA di partenza viene denaturato e poi ibridato con i 2 inneschi che
vanno a legarsi ciascuno su un filamento a un estremo della regione da
amplificare. Utilizzando i nucleosidi trifosfato che sono aggiunti alla
miscela, la DNA polimerasi allunga gli inneschi in direzione 5'
\(\rightarrow\) 3'.

I primi tentativi vennero fatti utilizzando DNA polimerasi I di E.coli,
che però veniva inattivata alla temperatura di denaturazione del DNA.

Mullis introdusse l'utilizzo di polimerasi estratte da organismi
termofili per effettuare il protocollo PCR. In questo modo non era più
necessario ripreparare la miscela di amplificazione alla fine di ogni
ciclo.

La polimerasi termostabile maggiormente usata è quella di
\emph{Thermophilus aquaticus}, la \textbf{Taq polimerasi}.

Tutti questi passaggi vengono effettuati in appositi strumenti che sono,
in sostanza, dei piccoli termostati in cui è possibile programmare in
maniera opportuna innalzamenti e abbassamenti della temperatura.

Un esempio di ciclo di amplificazine è:

\begin{enumerate}
\def\labelenumi{\arabic{enumi}.}
\itemsep1pt\parskip0pt\parsep0pt
\item
  \textbf{denaturazione}: 30 sec a 95°C per denaturare il DNA;
\item
  \textbf{associazione} (annealing): 30 sec a 40°-60°C per favorire
  làassociaizone specifica degli inneschi;
\item
  \textbf{sintesi}: 2 min a 72°C, che è la temperatura ottimale per la
  Taq pol.
\end{enumerate}

Esistono attualmente in commercio un gran numero di DNA polimerasi
termostabili con caratteristiche diverse:

La \emph{Taq} è priva di attività esonucleasi 3' \(\rightarrow\) 5'
(correzione bozze) ed incorpora nucleotidi sbagliati con la frequenza
piuttosto alta di 1 ogni (circa) 10 Kb. Nello stesso tempo però, è
sufficientemente processiva per amplificare senza problemi DNA stampo
lunghi 3-5 Kb.

Caratteristiche della Taq polimerasi: + attività \textbf{DNA
polimerasica} 5' \(\rightarrow\) 3'; + attività \textbf{esonucleasi} 5'
\(\rightarrow\) 3'; + attività di \textbf{Terminal transferasi}
(debole).

Viceversa, gli enzimi con attività esonucleasica 3' \(\rightarrow\) 5'
come Pwo e Pfu sono molto più fedeli (1-3 errori ogni 1000 Kb), non
presentano attività di terminal transferasi, ma hanno una bassa
processività.

Enzimi come la \emph{Taq} vanno bene per la maggior parte delle
applicaizoni in PCR.

Enzimi con attività esonucleasica 3' \(\rightarrow\) 5' come \emph{Pfu}
invece, sono necessari quando è molto importante amplificare in modo
fedele il DNA stampo.

Esistono pure in commercio muschele che sommano le proprietà più
vantaggiose di entrambi gli enzimi.

Esistono delle ``Golden Rules'' per disegnare delle buone coppie di
primer:

\begin{enumerate}
\def\labelenumi{\arabic{enumi}.}
\itemsep1pt\parskip0pt\parsep0pt
\item
  la lunghezza del primer deve essere compresa tra \textbf{18 e 22 paia
  di basi};
\item
  se possibile, il primer deve avere una sequenza che contenga una
  \textbf{percentuale di GC vicina al 50-60\%};
\item
  la temperatura di annealing deve essere compresa tra 50°C e 65°C, o
  comunque non deve mai essere troppo bassa (per evitare problemi di
  aspecificità);
\item
  il primer non deve contenere lunghi tratti di nucleotidi ripetuti (es.
  GGGGGG);
\item
  la temperatura di annealing teorica (T\(_a\)) del \textbf{Primer
  Forward} deve essere la stessa di quella del \textbf{Primer Revers} o,
  almeno, molto simile. I due primer utilizzati per la PCR sono definiti
  reverse e forward, a seconda che siano complementari al filamento 5'
  \(\rightarrow\) 3' o a quello inverso 3' \(\rightarrow\) 5'.
\item
  si devono evitare i tratti che potrebbero dare ``self-annealing''
  (riassociazione).
\end{enumerate}

Come si possono disegnano le coppie di primer?

La probabilità associata indica, per una data lunghezza di un primer, la
probabilità di riscontrare casualmente nel campione di DNA una sequenza
identica:

\textbf{P.A. = 4\(^n\)}

Dove:

\begin{itemize}
\itemsep1pt\parskip0pt\parsep0pt
\item
  \textbf{4} è il numero dei diversi nucleotidi del DNA (A, G, C, T);
\item
  \textbf{n} è la lunghezza dell'oligonucleotide.
\end{itemize}

Ad esempio, per un oligonucleotide di 16 basi, la probabilità di trovare
una identica sequenza nel campione di DNA è inferiore a 1 su 4 miliardi,
un numero più grande del numero di basi che compone l'intero genoma
umano.

Calcolare la \textbf{temperatura di ``annealing''} di un
oligonucleotide.

Per oligonucleotidi corti (meno di 20 bp) vale la regola empirica:

\textbf{T\(_a\) = 2 (A+T) + 4 (G+C)}

\{img/temperatura-annealing\}

La PCR è applicabile in diverse occasioni:

\begin{itemize}
\itemsep1pt\parskip0pt\parsep0pt
\item
  in \textbf{medicina forense e antropologia} per l'identificazione
  personale di tracce biologiche, l'identificazione personale di
  cadaveri non altrimenti riconoscibili, ecc\ldots{};
\item
  \textbf{analisi degli alimenti} (identificazione organismi OGM,
  contaminazioni batteriche, ecc\ldots{});
\item
  \textbf{genetica medica} (screening malattie ereditarie, diagnosi
  prenatale, ecc\ldots{});
\item
  \textbf{biologia molecolare} (clonaggio e mutagenesi sito specifica,
  ecc\ldots{}).
\end{itemize}

Uno degli svantaggi di questo metodo è che bisogna avere qualche
informazione sulla sequenza del pezzo di DNA che si vuole amplificare
per sintetizzare primer specifici appropriati su entrambe le estremità.

\textbf{Esempio}

Voglio amplificare la seguente sequenza

\{img/sequenza\}

Per prima cosa disegno i 2 oligonucleotidi complementari alle 2 regioni
che si trovano ai lati della regione che voglio amplificare. In questo
caso il Primer Forward (5' \(\rightarrow\) 3') è: \textbf{5'ATG GAG ACT
ACC AAT GGA ACG3'}

Dopodichè individuo la sequenza al 3' su cui disegno il Primer Reverse,
in questo caso: \textbf{5'AAC CGT GTG TCT ATG ATC TAA3'}

Il Primer Revers è però antisenso, ovvero si trova sul filamento
complementare (direzione 3' \(\rightarrow\) 5'). Risulata essere quindi:
\textbf{3'TTG GCA CAC AGA TAC TAG ATT5'}

Siccome però, per convenzione, tutti gli oligonucleotidi vengono scritti
in direzione 5' \(\rightarrow\) 3', bisogna disegnare la sequenza
``reverse'': \textbf{5'TTA GAT CAT AGA CAC ACG GTT3'}

\subsection{La Real-Time PCR}\label{la-real-time-pcr}

Questa consiste in una reazione PCR che può essere \emph{seguita in
tempo reale}, e la quantità di DNA sintetizzata può essere misurata ad
ogni ciclo PCR.

La differenza tra questo tipo di PCR e quello analizzato precedentemente
è che la PCR classica misura il DNA end-point, ovvero solo una volta che
questo ha saturato la soluzione, mentre la Real-Time PRC misura il DNA
alla fine di ogni ciclo di amplificazione.

Nella Real-Time PCR la crescite presenta 3 ``momenti'':

\begin{enumerate}
\def\labelenumi{\arabic{enumi}.}
\itemsep1pt\parskip0pt\parsep0pt
\item
  un primo momento di crescita esponenziale (limitata nel tempo);
\item
  una crescita lineare;
\item
  un plateau (non è correlato alla quantità iniziale di DNA presente).
\end{enumerate}

Il processo di duplicazione non procede ``all'infinito'', esso è
limitato dalla quantità dei primers, dall'attività della Taq pol e dal
reannialing dei filamenti.

Queste fasi possono essere individuate mettendo in relazione su un
grafico il numero di cicli effettuati (x) e il log{[}DNA{]} (y).

Per individuare il prodotto di PCR bisogna utilizzare i dati ottenuti
durante la fase esponenziale (prodotto PCR proporzionale al template
iniziale).

Questo è reso possibile mediante il rilevamento di una fluorescenza
proporzionale al prodotto di PCR.

La fluorescenza, durante ogni ciclo di amplificazione può essere
rilevata utilizzando uno strumento quantitativo che segue la cinetica
della reazione di PCR.

Perchè Real-Time? Perchè misura l'amplificazione in tempo reale durante
la fase esponenziale della PCR, quando cioè l'efficienza di
amplificazione è influenzata minimamente dalle variabili di reazione,
permettendo di ottenere risultati molto più accurati rispetto alla PCR
tradizionale ``endpoint''.

I componenti necessari per la reazione di Real-Time PCR sono:

\begin{itemize}
\itemsep1pt\parskip0pt\parsep0pt
\item
  DNA target;
\item
  DNA polimerasi;
\item
  due oligonucleotidi;
\item
  dNTPs.
\item
  sonda (probe) fluorescente
\end{itemize}

\subsubsection{Chimiche fluorescenti per PCR
Real-Time}\label{chimiche-fluorescenti-per-pcr-real-time}

La fluorescenza si genera durante la PCR per effetto di diverse
possibili reazioni chimiche.

Le chimiche principali sono basate:

\begin{itemize}
\itemsep1pt\parskip0pt\parsep0pt
\item
  sul legame di \textbf{coloranti fluorescenti} che si intercalano in
  modo aspecifico nella doppia elica di DNA (\emph{SYBR Green});
\item
  sull'\textbf{ibridazione di sonde} specifiche per il frammento di
  interesse \textbf{marcate con molecole fluorescenti}
  (\emph{Dual-labeled} come le \emph{sonde TaqMan}, Molecular beacons,
  Scorpion, e le sonde FRET = Fluorescence Resonance Energy Transfer).
\end{itemize}

\paragraph{SYBR Green}\label{sybr-green}

Questo metodo uhtilizza una molecola fluorescente non specifica che si
lega al solco minore del DNA.

Per questa colorazione si possono utilizzare primers in uso in
qualitativa. Non è costosa e non-specifica.

La molecola fluorescente si lega random a tutte le doppie eliche,
includendo i dimeri di primers; per questo motivo è necessario
ottimizzare la metodica per evitare la formazione di prodotti
aspecifici.

All'inizio del processo di amplificazione, la miscela di reazione
contiene DNA denaturato, primers e la molecola fluorescente.

\{img/SBRY1\}

Dopo l'anneling dei primer si legano poche molecole fluorescenti alla
doppia elica. Durante l'elongazione si verica un aumento di fluorescenza
che corrisponde all'aumento del numero di copie dell'\textbf{amplicone}
(un amplicone è un pezzo di DNA o RNA che è la sorgente e/o il prodotto
di eventi di amplificazione o di replicazione).

\paragraph{Sonde TaqMan}\label{sonde-taqman}

Le sonde di tipo TaqMan sono formate da un oligonucleotide che, come i
primers della PCR, viene disegnato per essere complementare alla
sequenza bersaglio da amplificare. La sonda è disegnata in modo da
ibridarsi all'interno del frammento amplificato nella reazione di PCR.

Presenta all'estremità 5' un \textbf{fluoroforo} ad alta energia che
emette fluorescenza detto \textbf{``Reporter'' (R)}, ed all'estremità 3'
una molecola a bassa energia che spegne la fluorescenza del reporter
detto \textbf{``Quencher'' (Q)}.

Se R e Q si trovano vicini, Q spegne l'effetto di R perchè i fotoni di R
vengono assorbiti da Q. L'aumento di fluorescenza di R è direttamente
proporzionale al numero di ampliconi generati.

\paragraph{Sonde FRET (Fluorescence Resonance Energy
Transfer)}\label{sonde-fret-fluorescence-resonance-energy-transfer}

Questo genere di sonde sono simili alle sonde TaqMan perché si legano al
DNA bersaglio e vengono idrolizzate. Ci sono però due sonde, ognuna
marcata con un solo fluorocromo (accettore e donatore).

Quando le sonde non sono legate alle sequenze target il segnale
fluorescente proveniente dall'accettore non è rilevato.

Durante lo step di annealing PCR, entrambe le sonde FRET ibridizzano
alle sequenze target: ciò avvicina il fluoroforo donatore all'accettore
permettendo il trasferimento di energia tra i due fluorofori e la
produzione di un segnale fluorescente da parte dell'accettore che viene
rilevato.

\paragraph{Molecular Beacons}\label{molecular-beacons}

I ``molecular beacons'' contengono un fluoroforo e un quencher non
fluorescente alle estremità opposte di un oligonucleotide. Le estremità
oppeste di questo oligonucleotide sono disegnate in modo da essere
complementari tra loro e formare una \textbf{struttura stem-loop}

Nella struttura stem-loop il fluoroforo e il quencher si trovano vicini,
e la loro vicinanza impedisce l'emissione di fluorescenza.

Il loop è complementare ad una sequenza all'interno del prodotto
amplificato.

Durante lo step di annealing PCR, la sonda ibridizza alla sua sequenza
target; ciò separa il colorante fluorescente dal reporter, producendo un
segnale fluorescente.

La quantità di fluorescenza prodotta ad ogni ciclo, o dopo la PCR,
dipende dalla quantità di prodotto specifico in quel dato momento.

A differenza delle sonde TaqMan, le molecular beacons non vengono
distrutte durante la reazione di amplificaizone per cui possono
reibridizzarsi durante il successivo ciclo.

\{img/probes\}

Se raccogliamo in un grafico i risultati ottenuti tramite la nostra R-T
PCR vedremo che, per ogni campione, si ottiene una curva di
amplificazione il cui \textbf{Threshold Cycle (C\(_T\))} è inversamente
proporzionale alla quantità di template iniziale.

\{img/diagramma-RT-PCR\}

Dopo ogni rilevamento, i segnli di fluorescenza, sono processati da un
software e la cinetica di formazione dell'amplificato è visualizzata
graficamente come incremento dei segnali di fluorescenza in ordinata
(\(\Delta\)Rn) per cicli di reazone in ascissa.

Il \textbf{ciclo threshol (C\(_T\))} è il ciclo della reazione di
amplificaizone nelquali il segnale di fluorescenza supera il valore di
threshold.

Il \textbf{threshold} rappresenta quela valore di fluorescenza pari a 10
volte la ds della fluorescenza registrata nei primi cicli di
amplificazione (background).

La quantificazione della fluorescenza può essere:

\begin{itemize}
\itemsep1pt\parskip0pt\parsep0pt
\item
  \textbf{quantificazione assoluta}, tramite l'uso di DNA a
  concentrazione nota (utilizzo di una \emph{standard curve});
\item
  \textbf{quantificaizone relativa}. Questa necessita di controlli
  endogeni tramite \emph{geni housekeeping} (o reference interni). Non
  viene utilizzata una standar curve, ma la quantificazione viene
  effettuata paragonando i C\(_T\). I geni housekeeping sono
  comunemente:

  \begin{itemize}
  \itemsep1pt\parskip0pt\parsep0pt
  \item
    ubiquitari;
  \item
    espressi in maniera costitutiva;
  \item
    non risentono dei trattamenti sperimentali;
  \item
    mantengono la loro espressione costante nei diversi tipi cellulari.
  \end{itemize}
\end{itemize}

Nella \textbf{quantificazione assoluta} si ha la necessita di utilizzare
una standard curve. Per costruirla è necessario produrre almeno 5
diluizione dello stesso standar. I campioni da determinare vengono
affiancati agli standar ed ai controlli negativi fatto con H\(_2\)O.

È necessario produrre 3 repliche sia per ogni campione sia per gli
standard per aumentare l'affidabilità del risultato.

Per ogni esperimento in quantificazione relativa è necessario:

\begin{itemize}
\itemsep1pt\parskip0pt\parsep0pt
\item
  un \textbf{target}, ovvero la sequenza di DNA da analizzare;
\item
  un \textbf{calibratore}, ovvero il campione da usare come riferimento
  per l'analisi comparativa;
\item
  un \textbf{controllo endogeno}, ovvero un gene espresso in maniera
  costitutiva in tutti i campioni analizzati necessario per normalizzare
  i dati rispetto alla quantità di DNA caricato e a variazioni di
  efficienza della reazione.
\end{itemize}

Per analizzare i dati ottenuti con la quantitativa relativa è
necessario:

\begin{enumerate}
\def\labelenumi{\arabic{enumi}.}
\itemsep1pt\parskip0pt\parsep0pt
\item
  normalizzare il target tramite un controllo endogeno (r) espresso
  costitutivamente (\(\Delta\)C\(_T\));
\item
  comparare ciascun \(\Delta\)C\(_T\) così ottenuto con il
  \(\Delta\)C\(_T\) di un calibratore (cb, \(\Delta\)\(\Delta\)C\(_T\))
\item
  inserire i dati nella seguente formula:
\end{enumerate}

\{img/formula1\}

Il valore così ottenuto permette di determinare la concentrazione
relativa del target.

Il primo step consiste nella normalizzazione tramite un controllo
endogeno (HK)

\{img/formula2\}

Il secondo consiste nella normalizzazione tramite il calibratore

\{img/formula3\}

Infine si applica la formula

\{img/formula4\}

\textbf{Esempio}

\{img/esempio\}

\end{document}
